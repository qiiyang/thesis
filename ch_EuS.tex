As discussed in Sec.~\ref{sec:TI-FM}, an insulating ferromagnet is necessary to implement an experiment to study proximity effects between topological insulators and ferromagnets. Among various materials quoted as ``insulating magnets'', many are ferrimagnets. Examples of these include magnetite (Fe$_3$O$_4$, $T_C\approx850~\mathrm{K}$)~\cite{Neel1948} and yittrium-iron garnet (``YIG'', Y$_3$Fe$_2$(FeO$_4$)$_3$, $T_C\approx550~\mathrm{K}$)~\cite{YIG}. In such case the net spontaneous magnetization is formed by opposite but differently sized magnetic lattices, therefore the 

Europium (II) Sulfide (EuS) thin films were fabricated by pulsed laser deposition (PLD).