As discussed in Sec.~\ref{sec:TI-FM}, an insulating ferromagnet is necessary to implement an experiment to study the interplay between topological insulators and ferromagnetism. Materials with both spontaneous magnetization and insulating behavior are often interchangeably referred to as ``magnetic insulators'' or ``magnetic semiconductors''. Among the variety of these materials, many are ferrimagnets. Such examples include magnetite (Fe$_3$O$_4$, Curie temperature $T_C\approx850~\mathrm{K}$)~\cite{Neel1948} and yittrium-iron garnet (``YIG'', Y$_3$Fe$_2$(FeO$_4$)$_3$, $T_C\approx550~\mathrm{K}$)~\cite{YIG}. In such case the net spontaneous magnetization is formed by opposite but unequal magnetic lattices, therefore the surface of these material may consist of alternating opposite magnetic domains, which would complicate the interface to a topological insulator. Insulating true ferromagnets, on the other hand, only have a small number of candidate materials available, despite being actively sought after by the spintronics community due to potential technological applications \cite[][p.~329]{kittel}~\cite{Yi2014, Wolf2001, EuS_spin_filter, EuS_app1, EuS_spin_filter2}. In addition to the studies presented in this dissertation, a class of these applications, such as $\pi$-Josephson junctions for quantum qubits~\cite{pi_qubit, pi_junction, Jing}, also require fabrication of high-quality thin films with both robust magnetic properties and highly insulating behavior.

Two of the first known insulating ferromagnets are the europium(II) chalcogenides EuO and EuS. Divalent compounds of europium with elements of the sixth group (O, S, Se, Te) exhibit a rock-salt (NaCl) type crystal structure with ordered magnetic states at low temperatures. As the lattice constant increases from EuO to EuTe, a ferromagnetic ordered state of moments localized on Eu$^{2+}$ ions appear in EuO ($T_C\approx69~\mathrm{K}$) and in EuS ($T_C\approx16.7~\mathrm{K}$)~\cite{EuO_TC, EuS_Shafer}, while EuSe and EuTe show collinear antiferromagnetic ordering with N\'eel temperatures $T_N\approx4.2~\mathrm{K}$, $T_N\approx9.8~\mathrm{K}$ respectively~\cite{EuSe_AF, EuTe_AF}. In these chalcogenide compounds, the \textit{S} ground state of Eu$^{2+}$ ions and their simple face centered cubic (FCC) magnetic make them prime candidates to test of the Heisenberg model of ferromagnetism and theories of critical phenomena~\cite{divalent_Eu,  EuS_neighbor_exchange, EuS_critical, EuS_neutron, EuS_spin_wave}. Between the two ferromagnets, for our purposes, EuS is favored over EuO. Firstly, the properties of EuS has been more thoroughly studied experimentally due to better sample qualities achieved~\cite{EuS_band_th2}. Furthermore, the relatively low Curie temperature of EuS would allow switching on and off its ferromagnetism at low temperatures, when quantum phenomena are more likely observable due to reduced thermal phenomena.

