\footnote[2]{A part of this chapter is adapted from ref.~\cite{EuS_PLD}~\fullcite{EuS_PLD}, with the permission of AIP Publishing.}%
As discussed in Sec.~\ref{sec:TI-FM}, an insulating ferromagnet is necessary to implement an experiment to study the interplay between topological insulators and ferromagnetism. Materials with both spontaneous magnetization and insulating behavior are often interchangeably referred to as ``magnetic insulators'' or ``magnetic semiconductors''. Among the variety of these materials, many are ferrimagnets. Such examples include magnetite (Fe$_3$O$_4$, Curie temperature $T_C\approx850~\mathrm{K}$)~\cite{Neel1948} and yittrium-iron garnet (``YIG'', Y$_3$Fe$_2$(FeO$_4$)$_3$, $T_C\approx550~\mathrm{K}$)~\cite{YIG}. In such case the net spontaneous magnetization is formed by opposite but unequal magnetic lattices, therefore the surface of these material may consist of alternating opposite magnetic domains, which would complicate the interface to a topological insulator. Insulating true ferromagnets, on the other hand, only have a small number of candidate materials available, despite being actively sought after by the spintronics community due to potential technological applications \cite[][p.~329]{kittel}~\cite{Yi2014, Wolf2001, EuS_spin_filter, EuS_app1, EuS_spin_filter2}. In addition to the studies presented in this dissertation, a class of these applications, such as $\pi$-Josephson junctions for quantum qubits~\cite{pi_qubit, pi_junction, Jing}, also require fabrication of high-quality thin films with both robust magnetic properties and highly insulating behavior.

Two of the first known insulating ferromagnets are the europium(II) chalcogenides EuO and EuS. Divalent compounds of europium with elements of the sixth group (O, S, Se, Te) exhibit a rock-salt (NaCl) type crystal structure with ordered magnetic states at low temperatures. As the lattice parameter increases from EuO to EuTe, a ferromagnetic ordered state of moments localized on Eu$^{2+}$ ions appear in EuO ($T_C\approx69~\mathrm{K}$) and in EuS ($T_C\approx16.5~\mathrm{K}$)~\cite{EuO_TC, EuS_Shafer, EuS_specific_heat}, while EuSe and EuTe show collinear antiferromagnetic ordering with N\'eel temperatures $T_N\approx4.2~\mathrm{K}$, $T_N\approx9.8~\mathrm{K}$ respectively~\cite{EuSe_AF, EuTe_AF}. In these chalcogenide compounds, the localized isotropic $L=0$ ground state ($^8S_{7/2}$) of Eu$^{2+}$ ions and their simple face centered cubic (FCC) magnetic lattice make them prime candidates to test the Heisenberg model of local-moment ferromagnetism and theories of critical phenomena~\cite{divalent_Eu, EuX_indirect_exchange, EuS_neighbor_exchange, EuS_critical, EuS_neutron, EuS_spin_wave}. Between the two ferromagnets, EuS is favored over EuO for our purposes. Firstly, the properties of EuS has been more thoroughly studied experimentally due to better sample qualities achieved~\cite{EuS_band_th2}. Furthermore, the relatively low Curie temperature of EuS would allow switching on and off its ferromagnetism at low temperatures, when changes in quantum phenomena are more likely observable due to reduced thermal motions. EuS therefore will be the focus of this chapter.

\section{Reported Properties of Crystals and Thin Films}
EuS has a crystal parameter $a_0=5.967~\mathrm\AA$~\cite{EuS_Shafer}, and exhibits an indirect gap of 1.65~eV between the valence band maximum at the $\Gamma$ point (Brillouin zone center) and the conduction band minimum at the X point ([100] Brillouin zone boundary)~\cite{EuX_absorption, EuS_band_th1, EuS_band_th2}. High quality stoichiometric single crystals of EuS exhibit Curie temperatures around $T_C\approx16.5~\mathrm{K}$, and good electric insulation with resistivity values $\rho\approx10~\Omega\cdot\mathrm{cm}$ at room temperature and $\rho\approx10^4~\Omega\cdot\mathrm{cm}$ at $T=135~\mathrm{K}$~\cite{EuS_Shafer}. However, when excessive electrons are present, either by intentional doping or due to difficulties in material fabrication, such as sulfur deficiency and crystal defects, the resistivity may be drastically reduced by n-type conduction to as low as $\rho\approx10^{-2}~\Omega\cdot\mathrm{cm}$ both at room temperature and at low temperatures~\cite{EuS_LaDoped, EuS_ntype}. Such reduction in resistivity was found to be universally accompanied by increased Curie temperatures (up to $T_C\approx30~\mathrm{K}$) due to interactions between charge carriers and the Eu$^{2+}$ ions~\cite{EuS_TC_doping, EuS_ntype, EuX_doped_transport, EuX_RKKY}. On the other hand, cases of p-type conduction in EuS has not been reported, presumably due to the readily available Eu$^{3+}$ ionic state, cancelling the effect of any acceptor doping~\cite{EuX_doped_transport}.

In one and two dimensions, N. D. Mermin and H. Wagner predicted that neither ferromagnetism nor antiferromagnetism can exist at any non-zero temperature within an isotropic Heisenberg model~\cite{Mermin1966}. From 3D crystals to thin films at finite temperatures, namely crossing over from three to two dimensions, one may therefore expect the Curie temperature of an isotropic Heisenberg ferromagnet to decrease as the thickness decreases. Indeed, such effect of dimensionality reduction has been investigated theoretically and observed experimentally~\cite{thickness_Tc_theory, thickness_Tc_exp}. Particularly for EuS, lower-than-bulk Curie temperatures were observed in thin films then the thicknesses $t < 6~\mathrm{nm}$~\cite{EuS_MBE_Muller}. However, above such thickness threshold, thin films fabricated by a variety of methods, such as electron beam evaporation, MBE, and PLD, often exhibited $T_C$ higher than single crystal values, which indicate significant n-type carrier doping~\cite{EuS_MBE_Muller, EuS_thin_film_Keller, EuS_PLD1, EuS_PLD2}. In addition, these reported growth methods often resulted in samples with multiple crystal orientations, which might give rise to fractured magnetic domains given the considerable magnetocrystalline anisotropy of EuS~\cite{EuS_anisotropy}, that might in turn complicate further experiments.

In the following sections of this chapter, I present a PLD procedure that yield EuS thin films with significantly improved qualities. Specifically, characterization results indicating excellent electric insulation, significant and uniform out-of-plane component of the magnetization, a single lattice orientation and a near-ideal surface topography, all of critical importance for applications involving interfacing the EuS film with another system.

