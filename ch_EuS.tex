\footnote[2]{A part of this chapter is adapted from ref.~\cite{EuS_PLD}~\fullcite{EuS_PLD}, with the permission of AIP Publishing.}%
As discussed in Sec.~\ref{sec:TI-FM}, an insulating ferromagnet is necessary to implement an experiment to study the interplay between topological insulators and ferromagnetism. Materials with both spontaneous magnetization and insulating behavior are often interchangeably referred to as ``magnetic insulators'' or ``magnetic semiconductors''. Among the variety of these materials, many are ferrimagnets. Such examples include magnetite (Fe$_3$O$_4$, Curie temperature $T_C\approx850~\mathrm{K}$)~\cite{Neel1948} and yittrium-iron garnet (``YIG'', Y$_3$Fe$_2$(FeO$_4$)$_3$, $T_C\approx550~\mathrm{K}$)~\cite{YIG}. In such case the net spontaneous magnetization is formed by opposite but unequal magnetic lattices, therefore the surface of these material may consist of alternating opposite magnetic domains, which would complicate the interface to a topological insulator. Insulating true ferromagnets, on the other hand, only have a small number of candidate materials available, despite being actively sought after by the spintronics community due to potential technological applications \cite[][p.~329]{kittel}~\cite{Yi2014, Wolf2001, EuS_spin_filter, EuS_app1, EuS_spin_filter2}. In addition to the studies presented in this dissertation, a class of these applications, such as $\pi$-Josephson junctions for quantum qubits~\cite{pi_qubit, pi_junction, Jing}, also require fabrication of high-quality thin films with both robust magnetic properties and highly insulating behavior.

Two of the first known insulating ferromagnets are the europium(II) chalcogenides EuO and EuS. Divalent compounds of europium with elements of the sixth group (O, S, Se, Te) exhibit a rock-salt (NaCl) type crystal structure with ordered magnetic states at low temperatures. As the lattice parameter increases from EuO to EuTe, a ferromagnetic ordered state of moments localized on Eu$^{2+}$ ions appear in EuO ($T_C\approx69~\mathrm{K}$) and in EuS ($T_C\approx16.5~\mathrm{K}$)~\cite{EuO_TC, EuS_Shafer, EuS_specific_heat}, while EuSe and EuTe show collinear antiferromagnetic ordering with N\'eel temperatures $T_N\approx4.2~\mathrm{K}$, $T_N\approx9.8~\mathrm{K}$ respectively~\cite{EuSe_AF, EuTe_AF}. In these chalcogenide compounds, the localized isotropic $L=0$ ground state ($^8S_{7/2}$) of Eu$^{2+}$ ions and their simple face centered cubic (FCC) magnetic lattice make them prime candidates to test the Heisenberg model of local-moment ferromagnetism and theories of critical phenomena~\cite{divalent_Eu, EuX_indirect_exchange, EuS_neighbor_exchange, EuS_critical, EuS_neutron, EuS_spin_wave}. Between the two ferromagnets, EuS is favored over EuO for our purposes. Firstly, the properties of EuS has been more thoroughly studied experimentally due to better sample qualities achieved~\cite{EuS_band_th2}. Furthermore, the relatively low Curie temperature of EuS would allow switching on and off its ferromagnetism at low temperatures, when changes in quantum phenomena are more likely observable due to reduced thermal motions..

\section{Reported Properties of Crystals and Thin Films}\label{sec:EuS_previous}
EuS has a crystal parameter $a_0=5.967~\mathrm\AA$~\cite{EuS_Shafer}, and exhibits an indirect gap of 1.65~eV between the valence band maximum at the $\Gamma$ point (Brillouin zone center) and the conduction band minimum at the X point ([100] Brillouin zone boundary)~\cite{EuX_absorption, EuS_band_th1, EuS_band_th2}. High quality stoichiometric single crystals of EuS exhibit Curie temperatures around $T_C\approx16.5~\mathrm{K}$, and good electric insulation with resistivity values $\rho\approx10~\Omega\cdot\mathrm{cm}$ at room temperature and $\rho\approx10^4~\Omega\cdot\mathrm{cm}$ at $T=135~\mathrm{K}$~\cite{EuS_Shafer}. However, when excessive electrons are present, either by intentional doping or due to difficulties in material fabrication, such as sulfur deficiency and crystal defects, the resistivity may be drastically reduced by n-type conduction to as low as $\rho\approx10^{-2}~\Omega\cdot\mathrm{cm}$ both at room temperature and at low temperatures~\cite{EuS_LaDoped, EuS_ntype}. Such reduction in resistivity was found to be universally accompanied by increased Curie temperatures (up to $T_C\approx30~\mathrm{K}$) due to interactions between charge carriers and the Eu$^{2+}$ ions~\cite{EuS_TC_doping, EuS_ntype, EuX_doped_transport, EuX_RKKY}. On the other hand, cases of p-type conduction in EuS has not been reported, presumably due to the readily available Eu$^{3+}$ ionic state, cancelling the effect of any acceptor doping~\cite{EuX_doped_transport}.

In one and two dimensions, N. D. Mermin and H. Wagner predicted that neither ferromagnetism nor antiferromagnetism can exist at any non-zero temperature within an isotropic Heisenberg model~\cite{Mermin1966}. From 3D crystals to thin films at finite temperatures, namely crossing over from three to two dimensions, one may therefore expect the Curie temperature of an isotropic Heisenberg ferromagnet to decrease as the thickness decreases. Indeed, such effect of dimensionality reduction has been investigated theoretically and observed experimentally~\cite{thickness_Tc_theory, thickness_Tc_exp}. Particularly for EuS, lower-than-bulk Curie temperatures were observed in thin films then the thicknesses $t < 6~\mathrm{nm}$~\cite{EuS_MBE_Muller}. However, above such thickness threshold, thin films fabricated by a variety of methods, such as electron beam evaporation, MBE, and PLD, often exhibited $T_C$ higher than single crystal values, which indicate significant n-type carrier doping~\cite{EuS_MBE_Muller, EuS_thin_film_Keller, EuS_PLD1, EuS_PLD2}. In addition, these reported growth methods often resulted in samples with multiple crystal orientations, which might give rise to fractured magnetic domains given the considerable magnetocrystalline anisotropy of EuS~\cite{EuS_anisotropy}, that might in turn complicate further experiments.

In the following sections of this chapter, I present a PLD procedure that yield EuS thin films with significantly improved qualities. Specifically, characterization results indicating excellent electric insulation, significant and uniform out-of-plane component of the magnetization, a single lattice orientation and a near-ideal surface topography, all of critical importance for applications involving interfacing the EuS film with another system.

\section{Pulsed Laser Deposition of Thin Films}
For PLD targets, solid disks (approximately $19~\mathrm{mm}$ in diameter and $3~\mathrm{mm}$ thick) were prepared from high-purity (99.95\%) EuS powder by a fast consolidation technique popularly referred to as spark plasma sintering (SPS)\footnote{Target preparation was carried out by Jinfeng Zhao and Subhash H. Risbud at the Department of Chemical Engineering and Materials Science, University of California, Davis.}. This technique uses an electric discharge to activate the surface of the powder particles prior to rapid resistance heating aimed at achieving complete densification. SPS has been effectively used to make solid disk-like targets of a wide range of materials including chalcogenides~\cite{Jinfeng2, Subhash1} and its efficiency in forming clean grain boundaries in polycrystalline targets has been shown for nitrides and refractory high-temperature materials.\cite{Subhash2, Jinfeng1} The target surface was polished with a 800~grit diamond sandpaper before transferring to high vacuum. For final conditioning of the target surface and to deposit EuS thin films, the target was ablated by a 25~ns 248~nm KrF excimer pulsed laser beam in $p=6\times{}10^{-7}~\mathrm{Torr}$ high vacuum at 10~Hz repetition rate. The typical ablation spot size was $2.1\pm0.3~\mathrm{mm^2}$ and the measured fluences were $1.0\pm0.2~\mathrm{J\cdot{}cm^{-2}}$. Corundum Al$_2$O$_3$ (0001) and Si (100) substrates were cleaned \textit{ex situ} by solvent sonication prior to transfer to high vacuum. The substrates were heated to $650~^{\circ}\mathrm{C}$ and placed 5~cm away from the target at the center of the plasma plume. The growth rate was estimated to be 1.3~\AA{} per pulse. After each deposition, the substrates were cooled in vacuum to $60~^{\circ}\mathrm{C}$ with a rate slower than $15~^{\circ}\mathrm{C}/\mathrm{min}$.

%
\begin{figure}[ht]%
\subfloat{\label{fig:EuS_TEM}}%
\subfloat{\label{fig:EuS_AFM}}%
\centering%
\includegraphics[width=0.75\columnwidth]{figs_EuS/EuS_TEM_AFM}%
\caption[Micrographs of the cross-section and of the surface of an EuS thin film]{(a)~Cross-sectional TEM image of an EuS thin film, showing its interface to the Al$_2$O$_3$ (0001) substrate. (b)~AFM image of a $1~\mathrm{\mu{}m}\times{}1~\mathrm{\mu{}m}$ area showing the surface topography of a $20~\mathrm{nm}$ EuS film. The root-mean-square roughness of $\sigma=1.8~\mathrm\AA$ indicates smoothness to the atomic scale.}%
\label{fig:EuS_TEM_AFM}%
\end{figure}%
%
The resultant thin films with thicknesses $20~\mathrm{nm}<t<200~\mathrm{nm}$ have a translucent purple color on Al$_2$O$_3$ and are dark green on Si. Fig.~\ref{fig:EuS_TEM} shows a transmission electron micrograph (TEM) of a thin film cross-section where the FCC lattice of EuS can be clearly observed.\footnote{TEM work was done by Ann F. Marshall at the Stanford Nanocharacterization Laboratory.} The lattice constant is estimated from direct length measurements to be $a=6.0\pm0.2~\mathrm\AA$, consistent with the established results~\cite{EuS_Shafer}. Surface topography was measured with an atomic force microscope (AFM). Fig.~\ref{fig:EuS_AFM} shows a $1~\mathrm{\mu{}m}\times{}1\mathrm{\mu{}m}$ area on the surface of a 20~nm film on Al$_2$O$_3$. The difference between the minima and maxima in height is roughly twice the lattice constant. The root-mean-square roughness $\sigma=1.8~\mathrm\AA$ calculated from a randomly selected line profile indicates near-ideal smoothness. Similar smoothness were obtained on films with thicknesses up to 200~nm deposited on either Al$_2$O$_3$~(0001) or Si~(100).

Fig.~\ref{fig:EuS_XRD} shows the X-ray diffraction patterns on the PLD grown EuS thin films. %
\begin{figure}[ht]%
\subfloat{\label{fig:EuS_XRD_sap}}%
\subfloat{\label{fig:EuS_XRD_Si}}%
\centering%
\includegraphics[width=0.65\columnwidth]{figs_EuS/XRD}%
\caption[X-ray diffraction patterns of EuS thin films]{Semi-log X-ray diffraction patterns of 20~nm EuS thin films. (a)~On Al$_2$O$_3$ (0001) substrates, optimal growth conditions lead to a single (100) orientation, whereas multiple orientations were observed in non-optimal samples. Spikes near substrate peaks are due to the K-$\beta$ components of the X-ray source. (b)~On Si (100) substrates with native oxides, single (100) orientation was obtained at optimal growth conditions. To distinguish the EuS (400) peak, a monochromator was used in the Si (100) case to eliminate the K-$\beta$ components.}%
\label{fig:EuS_XRD}%
\end{figure}%
%
On both Al$_2$O$_3$ (0001) and Si (100) substrates, the optimal conditions described earlier produced samples with a single orientation where the [100] planes are parallel to the substrate surface. On Al$_2$O$_3$ (0001) substrates (fig.~\ref{fig:EuS_XRD_sap}), the (200) and (400) reflections are easily identified whereas the (600) reflection is discernible from the background. On Si (100) substrates (fig.~\ref{fig:EuS_XRD_Si}), all [100] reflections are clearly observable. For comparison, the diffraction pattern of a non-optimal sample deposited on Al$_2$O$_3$ (0001) at a lower temperature ($T=600~^{\circ}\mathrm{C}$) was plotted in fig.~\ref{fig:EuS_XRD_sap}. Reflections from both the [100] and the [111] orientations were observed with comparable weights. Similar multiple orientations were observed in samples deposited at higher-than-optimal temperatures ($T>700~^{\circ}\mathrm{C}$) or higher ambient pressures ($p>2\times{}10^{-6}~\mathrm{Torr}$).

The resistances of the EuS thin films were measured with a van der Pauw technique~\cite{VdP1958}. When deposited at the optimal conditions on either Al$_2$O$_3$ (0001) or Si (100) substrates, samples with thicknesses $20~\mathrm{nm}<t<200~\mathrm{nm}$ all show sheet resistances $R_\Box>20~\mathrm{M\Omega}$ at temperatures $T > 100~\mathrm{K}$, and immeasurably  high resistance at lower temperatures. This is equivalent to bulk resistivity exceeding  $\rho>400~\mathrm{\Omega\cdot{}cm}$, consistent to values obtained on high-purity single crystals~\cite{EuS_Shafer}. In contrast, films deposited at non-optimal conditions show sheet resistances as low as $R_\Box\sim~\mathrm{k\Omega}$ (fig.~\ref{fig:EuS_RvT}), %
%
\begin{figure}[ht]%
\subfloat{\label{fig:EuS_RvT}}%
\subfloat{\label{fig:EuS_RvH}}%
\centering%
\includegraphics[width=0.75\columnwidth]{figs_EuS/trans}%
\caption[Electrical properties of EuS thin films]{(a)~While samples grown at optimal conditions have sheet resistance $R_\Box>20~\mathrm{M\Omega}$ for $2~\mathrm{K}<T<300~\mathrm{K}$, samples grown under non-optimal conditions (N1--N3, with 200~nm thickness) show finite resistance, indicating high carrier densities. (b)~These non-optimal samples show negative giant magnetoresistance at $T=2~\mathrm{K}$, similar to that observed in n-type single crystals.}%
\label{fig:EuS_transport}
\end{figure}%
%
which corresponds to a bulk resistivity of $\rho\sim10^{-2}\mathrm{\Omega\cdot{}cm}$, consistent with the conductive r\'egime in doped single crystals~\cite{EuS_ntype}. Different from n-type single crystals results, where resistance anomalies were observed near $T_C$ and attributed to change in carrier concentrations~\cite{EuS_ntype, EuX_doped_transport}, monotonic increases in resistance were observed towards low temperatures in thin films. Such difference could be resulted from different natures of dopants or due to the effects of reduced dimensionality~\cite{2D_conduction}. Similar to n-type doped single crystals, negative giant magnetoresistance was observed in conducting samples at low temperatures (fig.~\ref{fig:EuS_RvH}).

Magnetizations of the thin films were measured in a superconducting quantum interference device (SQUID) magnetometer down to $T=2~\mathrm{K}$. A significant perpendicular component of the magnetization was observed (fig.~\ref{fig:EuS_MvH_z}), whereas the easy axes are in the sample plane (fig.~\ref{fig:EuS_MvH_x}). %
%
\begin{figure}[h]%
\subfloat{\label{fig:EuS_MvH_z}}%
\subfloat{\label{fig:EuS_MvH_x}}%
\subfloat{\label{fig:EuS_MvT_z}}%
\subfloat{\label{fig:EuS_Sagnac}}%
\centering%
\includegraphics[width=0.75\columnwidth]{figs_EuS/mag}%
\caption[Magnetic properties of EuS thin films]{Magnetization of a 20~nm EuS film on Al$_2$O$_3$ (0001), (a)~in perpendicular fields, (b)~in parallel fields, and (c)~its temperature dependence, plotted in the same arbitrary unit with a linear paramagnetic component of the substrate subtracted. The fitting to the Curie-Weiss Law indicates a low Curie temperature $T_C=15.9~\mathrm{K}$. (d)~Kerr effect measured with a scanning Sagnac interferometer at $T=10~\mathrm{K}$, showing uniform magnetization.}%
\label{fig:EuS_magnetic}%
\end{figure}%
%
While the coercive field of perpendicular magnetization may vary within the same order of magnitude for film thicknesses between 20~nm and 200~nm, the general hysteresis features are similar for all our samples on either Al$_2$O$_3$ (0001) or Si (100). By fitting to the Curie-Weiss law in the paramagnetic regime (fig.~\ref{fig:EuS_MvT_z}), an upper limit of the Curie temperature of an optimal 20~nm thin film on Al$_2$O$_3$ (0001) was estimated to be $T_C=15.9~\mathrm{K}$.~\cite{Eu_mag_compounds} As discussed earlier in sec.~\ref{sec:EuS_previous}, a low $T_C$ is expected for samples with diminishing carrier densities~\cite{EuS_TC_doping, EuS_ntype, EuX_doped_transport}. Furthermore, the $T_C$ of thin films are expected to be lower than the minimal single crystal value $T_C\approx16.5~\mathrm{K}$ due to dimensionality reduction~\cite{thickness_Tc_theory, thickness_Tc_exp}.