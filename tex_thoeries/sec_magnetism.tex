A spontaneous magnetic order is a particular alignment of the magnetic moments in a material in absence of an external magnetic field. It is among the first quantum phenomena utilized by mankind~\cite{mag_history}. Spontaneous magnetism can be formed either by the spins of localized electrons, or by that of itinerant electrons~\cite{moriya1984}. Since itinerant magnetic order involves finite electric conduction, which would complicate the measurements of electron transport when interfacing with a topological insulator, we will focus on the localized case for the scope of this thesis.

Electronic orbitals have two-fold degeneracy due to the spin degree of freedom. A full subshell has zero total spin, whereas a partially filled subshell exhibits a net spin due to the Hund's rule of maximal multiplicity. A magnetic order formed by localized spins is described approximately by the Heisenberg model. In the isotropic case, only considering nearest neighbor interactions, the potential energy due to the spin is%
\begin{equation}%
    U = -2J \sum_{<i, j>} \vec{S_i} \cdot \vec{S_j}~,\label{eq:heisenberg}%
\end{equation}%
where $<i, j>$ denotes nearest neightbors~\cite{kittel}.

When the coupling constant, $J$, is positive, the ground state resulted from equation~\ref{eq:heisenberg} involves all spins aligning with each other, hence a net spontaneous magnetization is observed. This is referred to as ferromagnetism (figure~\ref{fig:bg_mag}a). %
\begin{figure}[ht]%
    \centering%
    \includegraphics[width=0.80\columnwidth]{figs_theories/magnetism}%
    \caption[Spin alignments in ferromagnetism, antiferromagnetism, and ferrimagnetism]{\label{fig:bg_mag}Examples of spin alignments in (a)~ferromagnetism, (b)~simple antiferromagnetism, and (c)~ferrimagnetism.}%
\end{figure}%
%
At sufficiently high temperatures, such spontaneous magnetic order disappears, and the material becomes paramagnetic. The behavior of magnetic susceptibility $\chi$, relating the magnetization $M$ and the applied field strength $H$, in such paramagnetic r\'egime is described by the Curie-Weiss law:%
\begin{equation}%
    \chi = \frac{M}{H} = \frac{C}{T - T_C}~,\label{eq:curie_weiss}%
\end{equation}%
where $C$ is the Curie constant. The Curie temperature, $T_C$, is the temperature dividing the ferromagnetic and the paramagnetic r\'egimes.

When $J$ is negative, and all the spins are equal in size, the spins of two nearest neighbors are opposite to each other in the ground state. In such case, no net spontaneous magnetization is obtained. This is referred to as antiferromagnetism (figure~\ref{fig:bg_mag}b). Similarly to ferromagnetism, antiferromagnetic order disappears above a temperature $T_N$ (the N\'eel temperature). Above $T_N$ the material become paramagnetic, and the magnetic susceptibility is similarly described by the Curie-Weiss law:%
\begin{equation}%
    \chi = \frac{M}{H} = \frac{C}{T - \Theta}~,%
\end{equation}%
where $\Theta$ is a negative constant.

When $J$ is negative, and nearest neighbors consist of unequal spins, the spins are opposite to each other in the ground state, similarly to antiferromagnetism. However, the opposing spins form unequal magnetic moments, therefore a finite spontaneous magnetization remains. This is referred to as ferrimagnetism (figure~\ref{fig:bg_mag}c).

When a non-magnetic material is in close proximity to a magnetic material, the time-reversal symmetry may be broken in the non-magnetic material. Examples of such phenomena include the so-called inverse proximity effect between a ferromagnet and a superconductor~\cite{Xia2009, Kalcheim2015}. Since both ferromagnetism and ferrimagnetism produce spontaneous magnetization, a close proximity to either might introduce time-reversal symmetry breaking in a topological insulator. The work documented in this thesis focused on ferromagnetism for its simplicity: the opposite spins in the sub-lattices in a ferrimagnet might produce complex magnetic domains at its surface, whereas it is more likely to observe a uniform magnetization on the surface of a ferromagnet.