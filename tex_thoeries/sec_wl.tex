The weak localization and the weak antilocalization effects are quantum interference effects that lead to electric conductivity deviating from classical values predicted by the Drude model. In the context of this thesis, we will concentrate on the two-dimensional case.

In a two-dimensional medium where finite disorder is present, consider electron scattering paths around a loop back to the same point and invert the direction of travel: for every possible scattering path, there exists a path that is counter-propagating, but otherwise identical (figure~\ref{fig:bg_scattering}a).%
\begin{figure}[ht]%
    \centering%
    \includegraphics[width=0.95\columnwidth]{figs_misc/wl_scattering}%
    \caption[Electron scattering paths in two-dimensions.]{\label{fig:bg_scattering}Illustration of electron scattering paths in a two-dimensional medium. (a)~For every scattering path, there is another path that is counter propagating, but otherwise identical. (b) Each path in the real space involves a \ang{180} rotation of the momentum, with the two paths rotate towards the opposite directions.}%
\end{figure}%

In absence of strong spin-orbit coupling, the two paths introduce exactly the same phase change to the incidental electron wave function, therefore interfere constructively. Hence the probability for the electron to travel backwards is increased. The electric resistance increases as consequence of the constructive interference of all such pairs of paths. This is referred to as the weak localization effect. A magnetic field perpendicular to such two-dimensional medium introduces a phase difference $\Delta\varphi = 2\cdot\frac{eBA}{\hbar}$ between a pair of counter-propagating loop enclosing an area $A$. Since the loops formed by different pairs of paths have different sizes, a magnetic field through the loops effectively introduces random phase differences. Therefore the weak localization effect is rapidly suppressed by a perpendicular magnetic field, and manifests as a sharp negative magnetoresistance or equivalently a sharp positive magnetoconductance in addition to the parabolic classical magnetoresistance (\textit{e.g.}, ref.~\cite{bishop1982}).

When the two-dimensional medium exhibits strong spin-orbit coupling, the opposite effect occurs. Two counter-propagating but otherwise identical scattering paths that invert the direction of travel are represented in the momentum space as two paths that are counter-propagating but otherwise centrosymmetric to each other (figure~\ref{fig:bg_scattering}b). These two paths are topologically equivalent to two connected semicircles that encircle the origin. Because of the strong spin-orbit coupling, the two paths result in electron spin rotating \ang{180} towards the opposite directions. Since electrons are spin-half fermions, this results in a $\Delta\varphi=\pi$ phase difference between the two paths. Consequently, the electric resistance is reduced from its classical value. This is referred to as the weak antilocalization effect, and manifests as a sharp positive magnetoresistance or equivalently a sharp negative magnetoconductance, in addition to the parabolic classical magnetoresistance (\textit{e.g.}, ref.~\cite{TI_WAL_Hongkong}).

In absence of either spin-orbit coupling or magnetic scattering, the weak localization effect produces a correction to electric conductance in two dimensions, $\Delta\sigma = \sigma - \sigma_{classical}$, that is proportional to the logarithmic of temperature~\cite{anderson1979, dolan1979}:%
\begin{equation}\label{eq:wl_T}
    \Delta\sigma \propto \log(T)~.
\end{equation}%
This results in a reduction of conductivity, or an increase in sheet resistance, as the temperature decreases. When the spin-orbit coupling is strong, on the other hand, the correction term due to weak antilocalization results in an increase of conductivity, or a reduction in sheet resistance, as the temperature lowers (ref.~\cite{bergmann1984}: figure~4.1).

The magneto-transport properties due to weak localization and weak antilocalization effects are described by the Hikami-Larkin-Nagaoka formula~\cite{WL_HLN, bergmann1984}:%
\begin{equation}\label{eq:hln}
    \Delta\sigma(H) = \sigma_0\left[f\left(\frac{H_1}{H}\right) - \frac{3}{2}f\left(\frac{H_2}{H}\right) + \frac{1}{2}f\left(\frac{H_3}{H}\right)\right]~,
\end{equation}%
where%
\begin{equation}\label{eq:sigma0}
    \sigma_0 = \frac{e^2}{2\pi^2\hbar}~,
\end{equation}
\begin{equation}
    f(z) = \ln(z) - \psi(z)~,
\end{equation}
and $\psi(z)$ is the digamma function. The three scaling numerators for the applied magnetic field are related to the characteristic fields of different types of scatterings:%
\begin{align}
    H_1 &= H_e + H_{so} + H_s\nonumber\\
    H_2 &= \frac{4}{3}H_{so} + \frac{2}{3}H_s + H_\phi\\
    H_3 &= 2H_s + H_\phi~.\nonumber
\end{align}
Each of the characteristic magnetic fields $H_i$ is related to the mean-free-length of a scattering type, $l_i$, by
\begin{equation}
    H_i = \frac{\hbar}{4el_i^2}~.
\end{equation}
The subscripts, $i$, denote the scattering type:%
\begin{description}
    \item [$e:$] elastic scattering
    \item [$so:$] spin-orbit scattering
    \item [$s:$] magnetic (spin-flipping) scattering
    \item [$\phi:$] inelastic (dephasing) scattering
\end{description}

Since the weak localization and weak antilocalization effects are quantum corrections to the Drude conduction model, they are valid only when the Drude model is valid. Further more, since their contributions to the electric conductivity is scaled by $\sigma_0 = \frac{e^2}{2\pi^2\hbar} = \frac{1}{\pi}\frac{e^2}{h}$, the conductivity needs to be much greater than $\sigma_0$ in order for such contributions to be considered correction terms:%
\begin{equation}\label{eq:sigma0}
    \sigma_{2D} \gg \frac{e^2}{h}~.
\end{equation}%
Namely, the 2D conductivity needs to be well above the Mott-Ioffe-Regel limit.