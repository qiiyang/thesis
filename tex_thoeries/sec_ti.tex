A Dirac cone is an electronic dispersion relation that takes form of a conical surface. The apex of such conical surface is called the Dirac point. Dirac points are often unstable in nature. Such tendency is illustrated by the von Neumann--Wigner problem~\cite{vonNeumann1993, Asano2011}

A three-dimensional topological insulator is a band insulator in its bulk, and possesses a surface state with spin-polarized Dirac-like dispersion (figure~\ref{fig:bg_ti}). %
\begin{figure}[ht]%
    \centering%
    \includegraphics[width=0.55\columnwidth]{figs_theories/band_ti}%
    \caption[Band structure of a 3D topological insulator]{\label{fig:bg_ti}A one-dimensional slice of the band structure of a three-dimensional topological insulator. The $x$--$y$ plane is defined by the surface of the material, while looking from the outside. The arrows denote the directions of the spin on the surface state. The surface state has a spin-polarized Dirac dispersion, and connects to the bulk conduction band (BCB) and the bulk valence band (BVB) within a Brillouin zone.}%
\end{figure}%
%
