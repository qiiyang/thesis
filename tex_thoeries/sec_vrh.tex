Electron transport in semiconductors in the low-conductivity r\'egime are discussed in detail by \citeauthor{schklovskii_efros} in ref.~\cite{schklovskii_efros}; whereas here the most relevant results are outlined.

In a semiconductor, namely a band insulator with a small band gap $E_g$, without any dopant, the Fermi level is at the middle of the gap (figure~\ref{fig:band_semiconductor}a). %
\begin{figure}[ht]%
    \centering%
    \includegraphics[width=1.0\columnwidth]{figs_misc/band_semiconductor}%
    \caption[Band structures of an intrinsic semiconductor and an n-type doped semiconductor]{\label{fig:band_semiconductor}Band structures of (a)~an intrinsic semiconductor and (b)~an n-type doped semiconductor.}%
\end{figure}%
%
At finite temperatures, the conduction in such intrinsic semiconductor is determined by thermal activation. As the electrons at the top of the valence band are thermally excited and enters the conduction band, the densities of intrinsic electrons and holes, $n_e$ and $n_h$, are equally increased (ref.~\cite{schklovskii_efros}, equation 4.1.1):%
\begin{equation}
    n_e = n_h = \frac{\left(2\pi\sqrt{m_e m_h}k_B T\right)^{3/2}}{4\pi^3\hbar^3}e^{-\frac{E_g}{2 k_B T}}~.
\end{equation}

In a semiconductor with dopants, the Fermi level is closely related to the ionization energy $E_{ion}$ of an isolated dopant (figure~\ref{fig:band_semiconductor}b). The conductivity contributed by the thermal excitation of dopant charges (extrinsic carriers) is often%
\begin{equation}
    \sigma = \sigma_1 e^{-\frac{\varepsilon_1}{k_B T}}~,
\end{equation}%
where $\varepsilon_1$ is close to $E_{ion}$ (ref.~\cite{schklovskii_efros}, equation 4.1.4).

When $k_B T \ll E_{ion} < E_g$, neither of intrinsic carriers or the extrinsic carriers are excited. Electric conduction in this r\'egime is by electrons jumping from occupied dopant sites to unoccupied ones due to weak overlapping between the wave functions of electrons on nearby dopant sites. This is known as hopping conduction (figure~\ref{fig:hopping}).%
\begin{figure}[ht]%
    \centering%
    \includegraphics[width=0.8\columnwidth]{figs_misc/hopping}%
    \caption[Illustration of hopping conduction]{\label{fig:hopping}Illustration of hopping conduction. In an electric field $\vec{E}$, electrons may jump from occupied dopant sites to unoccupied ones due to weak overlapping between the wave functions of electrons on neighboring dopant sites.}%
\end{figure}%

%
\begin{figure}[ht]%
    \centering%
    \includegraphics[width=0.95\columnwidth]{figs_theories/mott}%
    \caption[Energy structure of Mott's Law]{\label{fig:mott}Energy structure of Mott's Law.}%
\end{figure}%