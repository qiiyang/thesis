Electron transport in semiconductors in the low-conductivity r\'egime are discussed in detail by \citeauthor{schklovskii_efros} in ref.~\cite{schklovskii_efros}; whereas here the most relevant results are outlined.

In a semiconductor, namely a band insulator with a small band gap $E_g$, without any dopant, the Fermi level is at the middle of the gap (figure~\ref{fig:band_semiconductor}a). %
\begin{figure}[ht]%
    \centering%
    \includegraphics[width=1.0\columnwidth]{figs_misc/band_semiconductor}%
    \caption[Band structures of an intrinsic semiconductor and an n-type doped semiconductor]{\label{fig:band_semiconductor}Band structures of (a)~an intrinsic semiconductor and (b)~an n-type doped semiconductor.}%
\end{figure}%
%
At finite temperatures, the conduction in such intrinsic semiconductor is determined by thermal activation. As the electrons at the top of the valence band are thermally excited and enters the conduction band, the densities of electrons and holes, $n_e$ and $n_h$ are equally increased~\cite[section 4.1]{schklovskii_efros}:%
\begin{equation}
    n_e = n_p = \frac{\left(2\pi\sqrt{m_e m_h}k_B T\right)^{3/2}}{4\pi^3\hbar^3}e^{-\frac{E_g}{2 k_B T}}
\end{equation}%