At finite temperatures, electric conduction in a conductor is characterized by scattering of electrons. This is sometimes referred to as ``diffusive electron transport''. The scattering of electrons in an electric field can be described classically by the Drude model, which treats electrons moving among scattering centers as non-interacting charged pinballs being scattered around infinitesimal pins~\cite{drude1900a}:%
\begin{align}%
    \sigma &= \left|\frac{\vec{J}}{\vec{E}}\right| = \frac{ne^2\tau}{m}\\
        &= \frac{ne^2l}{p}~,\label{eq:drude}
\end{align}%
where $n$ is the electron density, $e$ the elementary charge, $\tau$ the mean free time, $m$ the effective mass of the electron, $l$ the mean free path, and $p$ the mean magnitude of the electron momentum. Although the Drude model is purely classical, the result in equation \ref{eq:drude} remains valid even when Fermi-Dirac statistics is taken into account, namely in the free electron (Drude-Sommerfeld) model~\cite{sommerfeld1928}.

The Drude model can be related to the band structure in two dimensions by considering an n-type conductor, where the Fermi level intersects with the conduction band at a wave vector $k = k_F$ (figure~\ref{fig:band_conductor}a). %
\begin{figure}[ht]%
    \centering%
    \includegraphics[width=0.5\columnwidth]{figs_misc/band_conductor}%
    \caption[Band structure of an n-type conductor]{\label{fig:band_conductor}A conductor where the Fermi level intersects with the conduction band.}%
\end{figure}%
%
The mean magnitude of the electron momentum is $p = \hbar k_F$. Assuming the band structure is isotropic, the area of the Fermi surface is $A_F = \pi k_F^2$. The density of states in $k$-space per unit area in the real space is%
\begin{align}
    g(\vec{k}) &= \frac{\diff^4N}{\diff^2 \vec{k} \diff^2 \vec{x}}\nonumber\\
        & = \frac{1}{(2\pi)^2}~,
\end{align}%
