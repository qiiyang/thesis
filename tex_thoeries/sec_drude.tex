The electric conduction in a conductor is described quasi-classically by the Drude model, which treats electrons moving among scattering centers as charged pinballs being scattered around infinitesimal pins~\cite{drude1900}:%
\begin{align}%
    \sigma &= \left|\frac{\vec{J}}{\vec{E}}\right| = \frac{ne^2\tau}{m}\nonumber\\
        &= \frac{ne^2l}{p}~,\label{eq:drude}
\end{align}%
where $n$ is the electron density, $e$ the elementary charge, $\tau$ the mean free time, $m$ the effective mass of the electron, $l$ the mean free path, and $p$ the mean magnitude of the electron momentum.

In a two dimensional conductor where the Fermi level intersects with the conduction band at a wave vector $k = k_F$ (figure~\ref{fig:band_conductor}a), %
\begin{figure}[ht]%
    \centering%
    \includegraphics[width=0.5\columnwidth]{figs_misc/band_conductor}%
    \caption[Band structure of an n-type conductor]{\label{fig:band_conductor}A conductor where the Fermi level intersects with the conduction band.}%
\end{figure}%
%
the mean magnitude of the electron momentum is $p = \hbar k_F$. Assuming the band structure is isotropic, the area of the Fermi surface is $A_F = \pi k_F^2$.