At finite temperatures, electric conduction in a macroscopic conductor is characterized by the scattering of electrons. This is sometimes referred to as ``diffusive electron transport''. The scattering of electrons in an electric field can be described classically by the Drude model, which treats electrons moving among scattering centers as non-interacting charged pinballs being scattered around infinitesimal pins~\cite{drude1900a}:%
\begin{align}%
    \sigma &= \left|\frac{\vec{J}}{\vec{E}}\right| = \frac{ne^2\tau}{m}\\
        &= \frac{ne^2l}{p}~,\label{eq:drude}
\end{align}%
where $n$ is the electron density, $e$ the elementary charge, $\tau$ the mean free time, $m$ the effective mass of the electron, $l$ the mean free path, and $p$ the mean magnitude of electron momentum. Although the Drude model is purely classical, the result in equation \ref{eq:drude} remains valid even when Fermi-Dirac statistics is taken into account, namely in the free electron (Drude-Sommerfeld) model~\cite{sommerfeld1928}.

The Drude model can be related to the band structure in two dimensions by considering an n-type conductor, where the Fermi level intersects with the conduction band at a wave vector $k = k_F$ (figure~\ref{fig:band_conductor}a). %
\begin{figure}[ht]%
    \centering%
    \includegraphics[width=0.5\columnwidth]{figs_misc/band_conductor}%
    \caption[Band structure of an n-type conductor]{\label{fig:band_conductor}A conductor where the Fermi level intersects with the conduction band.}%
\end{figure}%
%
The mean magnitude of electron momentum is
\begin{equation}
    p = \hbar k_F~.\label{eq:drude_p}
\end{equation}%
Assuming the band structure is isotropic, the area of the Fermi surface is $A_F = \pi k_F^2$. Consider the two-fold degeneracy due to the spin degree of freedom, and a periodic boundary condition: $\vec{k} \cdot \vec{L} = 2N\pi$, where $\vec{L}$ is any vector connecting two points on the boundary of the sample, and $N$ is restricted to be integers, the density of states in $k$-space per unit real-space area is%
\begin{align}
    g(\vec{k}) &= 2\cdot\frac{\diff^4N}{\diff^2 \vec{k} \diff^2 \vec{L}}\nonumber\\
        & = \frac{2}{(2\pi)^2}~.
\end{align}%
At the limit where the Fermi surface is small, hence the entire Fermi surface contributes to electric conduction, the two-dimensional electron density is the total number of filled states in the conduction band:%
\begin{align}
    n_{2D} &= g(\vec{k}) A_F\nonumber\\
        &= \frac{k_F^2}{2\pi}~.\label{eq:drude_n}
\end{align}%
Combining equations \ref{eq:drude}, \ref{eq:drude_p} \& \ref{eq:drude_n}, the 2D conductivity therefore takes the form:
\begin{equation}
    \sigma_{2D} = \frac{e^2}{h} \cdot (k_F l)~.\label{eq:drude_2d}
\end{equation}%

The above discussion about an n-type conductor can be similarly applied to p-type conductors. In such case, the Fermi level intersects with the valence band; $n$ and $m$ refer to the density and effective mass of the holes rather than of electrons.

Equation \ref{eq:drude_2d} suggests a lower bound of $\sigma_{2D}$ for the Drude model to be applicable. Since equation \ref{eq:drude} requires the momentum $p$, hence the wave vector $k_F$ to be well-defined. When $k_F l \lesssim 1$, however, scatterings occur to an electron at a similar or smaller length scales compared to its wavelength, such requirement is no longer satisfied. In such case, localization plays an important role in electric conduction. Variable-range hopping, one of the conduction models for such localized r\'egime, is introduced in section~\ref{sec:vrh}. At $k_F l = 1$, the conductivity takes a quantum value, namely the inverse of the von Klitzing constant:%
\begin{equation}
    \sigma_{2D}(k_F l = 1) = \frac{1}{R_K} = \frac{e^2}{h}~.
\end{equation}%
Such lower bound on the 2D conductivity for the Drude model is sometimes referred to as the Mott-Ioffe-Regel limit \cite{mott_book, hussey2004}.