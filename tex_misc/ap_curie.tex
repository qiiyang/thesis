The critical temperature of a phase transition, such as the Curie temperature ($T_C$) for a ferromagnetic transition, is best determined by measuring thermodynamic properties. Examples of such techniques include specific heat measurements, and, specifically for magnetic transitions, the Arrott method~\cite{Arrott1957}. In practice, when the accuracy is not crucial, a variety of approximate methods have been used to estimate the $T_C$ of EuS thin films. In some studies, the $T_C$ was quoted as that above which the spontaneous magnetization disappears in zero field, \textit{e.g.}, refs.~\cite{EuS_MBE_Dauth, Moodera2013}; in others, the sharpest slope in the temperature dependence of in-field magnetization was taken, and the $T_C$ was quoted as the extrapolated intercept to zero magnetization, \textit{e.g.}, ref.~\cite{EuS_PLD1}.

For both ferromagnetic europium(II) chalcogenides, EuS and EuO, Curie temperatures estimated by fittings to the Curie-Weiss law near and above the ferromagnetic transition were shown to be consistently greater than the values measured by thermodynamic methods~\cite{Eu_mag_compounds}. Namely, a $T_C$ estimated by a fitting to the Curie-Weiss law serves as an upper bound to the thermodynamic value. Since one of the reasons to estimate $T_C$ in this thesis was to compare the qualities with previously reported samples, where a lower $T_C$ is an indication of better sample quality (\textit{q.v.}, section~\ref{sec:EuS_previous}), Curie-Weiss law fittings were chosen as the method to estimate $T_C$ specifically because of its overestimating nature, such that claims of sample qualities were made conservatively.

A disadvantage of fittings to the Curie-Weiss law is its dependence on the measurements taken above the $T_C$. As the temperature increases, the thin films rapidly stops being the dominant magnetic dipole compared to the surrounding material, thereby the technical limit of the SQUID measurements is encountered (\textit{q.v.} section~\ref{sec:squid}). A few Kelvins above $T_C$, as the dominant dipole shift in position relative to the SQUID coils, a jump was always observed in the signal (figure~\ref{fig:inverse_M}).\footnote{Errorbars represent 90\% confidence intervals.}%
\begin{figure}[ht]%
    \centering%
    \includegraphics[width=1.0\columnwidth]{communications/squid1104}%
    \caption[Inverse magnetization plots for EuS thin films]{\label{fig:inverse_M}(Color online) The inverse magnetization plots for (a)~figure~\ref{fig:EuS_MvT_z}, and (b)~figure~\ref{fig:bl2014_MvT}, with constant background values subtracted from the magnetizations. As the temperature increases above $T_C$, the EuS thin film rapidly stops being the dominant dipole compared to the surrounding material, a shift in SQUID center results in a jump in the signal.}%
\end{figure} %
%
Consequently, the fittings were performed only for a few Kelvins between the estimated $T_C$ and the jump in signals. While such small ranges of temperatures are not ideal, the $T_C$ obtained by such fitting were corroborated by the polar Kerr effect measurements (figure~\ref{fig:bl2014_SchemmT}), and the AC susceptibility measurements (figure~\ref{fig:bilayer2018_mi2t40}).

The Curie temperatures estimated by fittings to the Curie-Weiss law were referenced to in the temperature dependence of transport measurements. Perhaps most consequentially, the estimated $T_C$ was used to infer a correlation between the appearance of the negative magnetoresistance and the ferromagnetic transition. While the difference between the $T_C$ estimated in the paramagnetic r\'egime and that measured by thermodynamic techniques have not been studied systematically for thin films, such difference was found to be roughly \SI{2}{K} for single crystals. Should the value be similar for thin films, such difference should not affect the interpretation of the magnetoresistance measurements, for which the temperature steps were greater or similar to such difference (figure~\ref{fig:bl2014_quadratic}d).