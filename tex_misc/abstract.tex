To investigate the effects due to proximity between a three-dimensional topological insulator (TI) and an insulating ferromagnet (IF), TI--IF thin film bilayers were fabricated with pulsed laser deposition. Either of Bi$_2$Se$_3$ and (Bi$_x$Sb$_{1-x}$)$_2$Te$_3$ was used for the TI layer, whereas the IF layer was formed by the Heisenberg ferromagnet EuS. While a positive magnetoresistance was observed above the Curie temperature of EuS, as ubiquitously observed in high-quality TI thin films, an unusual negative magnetoresistance was observed below the Curie temperature in the variable-range hopping r\'egime. Specific to (Bi$_x$Sb$_{1-x}$)$_2$Te$_3$-EuS bilayers, when the bulk conduction is minimized, magnetic anomalies in AC susceptibility and resistive anomalies were observed concurrently at the same temperatures, suggesting an interface magnetic order. These phenomena together suggest a two-stage proximity effect between the topological insulator and the insulating ferromagnet.