An ideal three-dimensional topological insulator (3D TI) is an electric insulator in its bulk and conducting on its surface. In the bulk, similarly to a conventional insulator or semiconductor, a full energy gap is present at the Fermi level; whereas on the surface, the electronic dispersion relation contains an odd number of spin-polarized Dirac cones inside the bulk gap. On such a spin-polarized Dirac cone, an electron's energy relates linearly to its crystal momentum. The energy converges to a single value at the Dirac point in the momentum space. Two electrons with the same energy and opposite spins have equal and opposite momenta relative to the Dirac point~\cite{TI_Qi, TI_Col}.

Such spin-polarized Dirac dispersion is topologically protected by the time-reversal symmetry in a 3D TI, such that the degeneracy at the Dirac point is robust against any scattering mechanism that does not break time-reversal symmetry. Such protection can be removed, therefore a gap opened at the Dirac point, by breaking time-reversal symmetry

This thesis documents experimental work investigating the effects due to proximity between a three-dimensional topological insulator and an insulating ferromagnet.