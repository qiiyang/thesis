A three-dimensional topological insulator (3D TI) is an electric insulator in its bulk, and a conductor on its surface. In the bulk, similarly to a conventional insulator or semiconductor, a full energy gap is present at the Fermi level; whereas on the surface, the electronic dispersion relation contains an odd number of spin-polarized Dirac cones inside the bulk band gap within a Brillouin zone. On such a spin-polarized Dirac cone, an electron's energy relates linearly to its crystal momentum. The energy converges to a single value at the Dirac point in the momentum space. Two electrons with the same energy and opposite spins have equal and opposite momenta relative to the Dirac point~\cite{TI_Qi, TI_Col}.

The spin-polarized Dirac dispersion is topologically protected by the time-reversal symmetry (TRS) in a 3D TI, therefore the degeneracy at the Dirac point is robust against any scattering mechanism that does not break the TRS. Such protection can be removed, thereby a gap opened at the Dirac point, by breaking the TRS. A variety of novel phenomena have been predicted as consequence of such gap-opening, including image magnetic monoples~\cite{TI_birth, TI_monopole}, inverse spin-galvanic effect~\cite{ISG}, and integer or half-integer--quantized anomalous Hall effects (QAHE)~\cite{TI_Col, QAH_TI_Yu}. Two approaches have been attempted so far to break the TRS in a 3D TI with spontaneous magnetic orders:

The first approach is to introduce magnetic impurities, such as Mn, Fe, and Cr, into a 3D TI. This approach has the advantage of technical simplicity, and succeeded in producing the integer QAHE during the timespan of the projects documented in this thesis~\cite{Chang2013, Kou2014}. However, magnetic impurities in the bulk introduce disorder and inhomogeneity, and alter the band structure of the entire surface as well as the bulk. Consequently, a half-integer QAHE has not been observed so far, and a considerable remnant longitudinal conductance coexisted with the integer QAHE.

This thesis documents our experimental work on the second approach, that is to break the TRS in a 3D TI by close proximity to an insulating ferromagnet. Such approach involves more complex heterostructures, and traded bulk disorder with interface complications. However, it has the advantage of not altering the bulk band structure, and is potentially capable of breaking the TRS only on a part of the surface, hence to realize the half-integer QAHE~\cite{TI_Col}.

Topological insulator--insulating ferromagnet thin film bilayers were fabricated with pulsed laser deposition; unusual negative magnetoresistance was observed below the Curie temperature of the ferromagnet in the variable-range hopping r\'egime; magnetic and resistive anomalies were observed at the same temperatures above the Curie temperature, suggesting an interface magnetic order. Parallel efforts on similar bilayers were carried out in other laboratories~\cite{Moodera2013, Samarth2013, Moodera2016}, and have inspired a part of the work reported in this thesis.

The various theoretical topics relevant to the discussion will be introduced in chapter~\ref{ch:theory}. Experimental techniques employed in this work will be described in chapter~\ref{ch:methods}. The results of the experiments will be presented and discussed in chapters~\ref{ch:EuS}--\ref{ch:models}. Finally, the results will be summarized, and conclusions drawn, in chapter~\ref{ch:conclusion}.