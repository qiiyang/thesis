A three-dimensional topological insulator (3D TI) is an electric insulator in its bulk, and a conductor on its surface. In the bulk, similarly to a conventional insulator or semiconductor, a full energy gap is present at the Fermi level; whereas on the surface, the electronic dispersion relation contains an odd number of spin-polarized Dirac cones inside the bulk band gap within a Brillouin zone. On such a spin-polarized Dirac cone, an electron's energy relates linearly to its crystal momentum. The energy converges to a single value at the Dirac point in the momentum space. Two electrons with the same energy and opposite spins have equal and opposite momenta relative to the Dirac point (figure~\ref{fig:intro_ti}).%
\begin{figure}[ht]%
    \centering%
    \includegraphics[width=0.55\columnwidth]{figs_theories/band_ti}%
    \caption[Band structure of a 3D topological insulator]{\label{fig:intro_ti}A one-dimensional slice of the band structure of a three-dimensional topological insulator, modeled after Bi$_2$Se$_3$~\cite{TI_electronic_structure_zhang}. The $x$--$y$ plane is defined by the local surface of the material while looking from the outside. The arrows denote the directions of the spin on the surface state, measured on the $y$-axis. The surface state has a spin-polarized Dirac dispersion, and connects to the bulk conduction band (BCB) and the bulk valence band (BVB) within the Brillouin zone.}%
\end{figure}%

On a spin-polarized Dirac cone, the momentum and the spin of an electron are locked, resulting in a dominant Rashba-like term in the Hamiltonian:%
\begin{equation}%
    \op{H}_{surf} = A (\vec{\sigma} \times \vec{k}) \cdot \vu*{z}~,\label{eq:surf_Rashba}%
\end{equation}%
where $\vu*{z}$ is normal to the surface~\cite{TI_Qi, TI_electronic_structure_zhang, Liu2010}. Therefore traversing a closed path around the origin in the momentum space also rotates the spin by $360^\circ$. This results in the weak anti-localization effect, which manifests ubiquitously in high-quality TI thin films as sharp positive magnetoresistance~\cite{TI_WAL_Hongkong, TI_WAL_thickness, zhangli2012, zhangli2013}.

The Dirac points of a 3D TI's surface state situate at time-reversal momenta, namely Brillouin zone center or boundaries, their spin degeneracies are therefore protected by the time-reversal symmetry (TRS). When the Fermi level intersects a spin-polarized Dirac cone, each spin polarization crosses the Fermi level a single time. On the surface of a 3D TI, since there are odd number of Dirac cones within a Brillouin zone, no adiabatic deformation of the electronic dispersion can remove all crossings from the Fermi level, such crossings are therefore topologically protected~\cite{Kane2005}. The conducting surface state should be thereby robust against any non TRS--breaking scattering mechanisms~\cite{TI_Qi, TI_Col}.

\paragraph{}
Since the protection of the Dirac point in a 3D TI is due to the combination of the TRS and an odd number of Fermi level crossings, the degeneracy at the Dirac point can be lifted, thereby a gap opened at the Dirac point, by breaking the TRS. A variety of novel phenomena have been predicted as consequence of such gap-opening, including image magnetic monoples~\cite{TI_birth, TI_monopole}, inverse spin-galvanic effect~\cite{ISG}, integer or half-integer--quantized anomalous Hall effects (QAHE)~\cite{TI_Col, QAH_TI_Yu}, and observations of Majorana fermions~\cite{Akhmerov2009, TI_Majorana}. Two approaches have been attempted so far to break the TRS in a 3D TI with spontaneous magnetic orders:

The first approach is to introduce magnetic impurities, such as Mn, Fe, and Cr, into a 3D TI. This approach has the advantage of technical simplicity, and succeeded in producing the integer QAHE during the timespan of the projects documented in this thesis~\cite{Chang2013, Kou2014}. However, magnetic impurities in the bulk introduce disorder and inhomogeneity, and alter the band structure of the entire surface as well as the bulk. Consequently, a half-integer QAHE has not been observed so far, and a considerable remnant longitudinal conductance coexisted with the integer QAHE.

This thesis documents our experimental work on the second approach, that is to break the TRS in a 3D TI by close proximity to a ferromagnet. Such approach involves more complex heterostructures, and traded bulk disorder with interface complications. However, it has the advantage of not altering the bulk band structure, and is potentially capable of breaking the TRS only on a part of the surface, hence to realize the half-integer QAHE and the associated Majorana physics~\cite{TI_Qi, TI_Col}. In order to break the TRS and to observe the associated effects in the TI, the ferromagnet has to satisfy a few criteria. Firstly, its magnetization needs to have a component perpendicular to the interface to the TI~\cite{MnSe}. Secondly, such perpendicular magnetization should be uniformly distributed, in order to have well-controlled magnetic domains. Thirdly, the interface between the TI and the ferromagnet needs to be well-defined, in order to ensure good proximity between the two materials. Finally, the ferromagnet itself should be insulating, in order not to contribute to electric conduction itself.

To investigate the effects due to proximity between a 3D TI and an insulating ferromagnet (IF), we fabricated TI--IF thin film bilayers by pulsed laser deposition. The deposition and characterization of thin films of the IF, europium(II) sulfide (EuS), are presented in chapter~\ref{ch:EuS}. In chapter~\ref{ch:bilayer2014}, we report experiments on TI--IF bilayers with bismuth(III) selenide (Bi$_2$Se$_3$) as the TI layer, where an unusual negative magnetoresistance observed below the Curie temperature of the ferromagnet. Improving over Bi$_2$Se$_3$, in chapter~\ref{ch:bilayer2018}, bismuth-antimony(III) telluride ((Bi$_x$Sb$_{1-x}$)$_2$Te$_3$) was used as the TI layer to minimize bulk conduction and thereby to isolate the surface conduction. In addition to the negative magnetoresistance, magnetic and resistive anomalies were observed at the same temperatures above the Curie temperature, suggesting an interface magnetic order. The high sheet resistance of TI--IF bilayer samples which exhibited negative magnetoresistance, both in this thesis and in subsequent literature~\cite{Samarth2017, Tian2016}, suggest a strong localization r\'egime. Fittings to the variable-range hopping models were carried out and discussed in chapter~\ref{ch:models}. Finally, the results will be summarized, and conclusions drawn, in chapter~\ref{ch:conclusion}.

Supplemental to the results and discussions in chapters~\ref{ch:EuS}--\ref{ch:conclusion}, the theoretical background and the experimental techniques employed in this work are described in more detail in appendices~\ref{ch:theory}~\&~\ref{ch:methods}.