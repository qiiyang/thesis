In this thesis, we reported our investigations on the effects due to close proximity between three dimensional topological insulators (TIs) and insulating ferromagnets (IF). Pulsed laser deposition was employed to deposit thin films of the insulating ferromagnet, EuS, and TI-IF thin film bilayers: Bi$_2$Se$_3$--EuS and (Bi$_x$Sb$_{1-x}$)$_2$Te$_3$--EuS. Electron transport was measured with the van der Pauw method, and the magnetic properties by SQUID magnetometry and AC susceptibility measurements. Overall, we observed two stages of such proximity effects: the negative magnetoresistance that emerges below the Curie temperature of bulk EuS, and coincidental magnetic and resistive anomalies that occurred above the Curie temperature around $T\approx30$--$60~\mathrm{K}$.

\paragraph{}
Instead of the sharp positive magnetoresistance that is observed ubiquitously in thin film TIs, often interpreted in terms of the weak anti-localization effect, sharp negative magnetoresistance emerges below the Curie temperature of EuS, in both Bi$_2$Se$_3$--EuS bilayers (figure~\ref{fig:bl2014_MR_thickness}) and (Bi$_x$Sb$_{1-x}$)$_2$Te$_3$--EuS bilayers (figure~\ref{fig:bilayer2018_mr}), suggesting an influence of the ferromagnetism in the EuS on the electron transport in the TI layer.

By compiling existing reports on the transport properties of TI-IF bilayers in literature, it was observed that the negative magnetoresistance only arises when the sheet resistance exceeds the von Klitzing constant $h / e^2$, namely the Ioffe-Regel-Mott limit in two-dimensions (figure~\ref{fig:bilayer2018_wl_trend}). The analyses of the temperature dependence of resistance suggest that the negative magnetoresistance occurs well inside the variable-range hopping r\'egime in presence of Coulomb gaps (chapter~\ref{ch:models}).

Such variable-range hopping r\'egime, however, does not seem to be the sole cause of the negative magnetoresistance in bilayers. In the studies on \SI{2}{nm} (Bi$_x$Sb$_{1-x}$)$_2$Te$_3$ thin films~\cite{liao2015}, \citeauthor{liao2015} observed a negative magnetoresistance, that is attributed to variable-range hopping alone, only when the sheet resistance exceeds $0.5~\textrm{M}\Omega$. The bilayer samples in this thesis, on the other hand, exhibits negative resistance when the sheet resistance is as low as $20~\textrm{k}\Omega$ (figures~\ref{fig:bilayer2018_rvt_s1}~\&~\ref{fig:bilayer2018_mr_s1}), and the appearance of such negative magnetoresistance is closely correlated to the ferromagnetic transition in the EuS layer (figures~\ref{fig:bl2014_MR_temperature}~\&~\ref{fig:bilayer2018_mr_s1}). This suggests that the negative magnetoresistance discussed in this thesis is due to the proximity to the ferromagnet, with the variable-range hopping r\'egime serving as a prerequisite.

\paragraph{}
In (Bi$_x$Sb$_{1-x}$)$_2$Te$_3$--EuS bilayers, above the Curie temperature of bulk EuS, magnetic anomalies were observed at $T\approx30~\mathrm{K}$ and $T\approx60~\mathrm{K}$, and resistive transitions were observed at the same temperatures. Such phenomenon was so far observed exclusively in the thin and optimally-doped (Bi$_x$Sb$_{1-x}$)$_2$Te$_3$--EuS bilayers, where the bulk conduction is minimized, hence the surfaces should dominate the electric conduction. The magnetic anomalies were observed only with AC susceptibility measurements, which are sensitive to surface and interface effects, but not in SQUID measurements, which are dominated by the signal from bulk EuS. This seems to support the conclusion of \citeauthor{Moodera2016} that an interface ferromagnetic order exists above the bulk Curie temperature of EuS~\cite{Moodera2016}.