\section{Summary of Results}
In this thesis, we reported our investigations on the effects due to close proximity between three dimensional topological insulators (TIs) and insulating ferromagnets (IF). Pulsed laser deposition was employed to deposit thin films of the insulating ferromagnet, EuS, and TI--IF thin film bilayers: Bi$_2$Se$_3$--EuS and (Bi$_x$Sb$_{1-x}$)$_2$Te$_3$--EuS. Electron transport was measured with the van der Pauw method, and the magnetic properties by SQUID magnetometry and AC susceptibility measurements. Overall, we observed two stages of such proximity effects: the negative magnetoresistance that emerges below the Curie temperature of bulk EuS, and coincidental magnetic and resistive anomalies that occurred above the Curie temperature around $T\approx30$--$60~\mathrm{K}$.

Instead of the sharp positive magnetoresistance that is observed ubiquitously in thin film TIs, often interpreted in terms of the weak antilocalization effect, sharp negative magnetoresistance emerges below the Curie temperature of EuS, in both Bi$_2$Se$_3$--EuS bilayers (chapter~\ref{ch:bilayer2014}) and (Bi$_x$Sb$_{1-x}$)$_2$Te$_3$--EuS bilayers (chapter~\ref{ch:bilayer2018}). Such correlation between the emergence of the negative magnetoresistance and the Curie temperature of EuS strongly suggests an influence of the ferromagnetism in the EuS on the electron transport in the TI layer.

In (Bi$_x$Sb$_{1-x}$)$_2$Te$_3$--EuS bilayers, above the Curie temperature of bulk EuS, magnetic anomalies were observed at $T\approx30~\mathrm{K}$ and $T\approx60~\mathrm{K}$, and resistive transitions were observed at the same temperatures. Such phenomenon was observed exclusively in the thin and optimally-doped (Bi$_x$Sb$_{1-x}$)$_2$Te$_3$--EuS bilayers, where the bulk conduction is minimized, hence the surfaces should dominate the electric conduction. The magnetic anomalies were observed only with AC susceptibility measurements, which are sensitive to surface and interface effects, but not in SQUID measurements, which are dominated by the signal from the bulk of EuS. These combined seem to suggest an interface origin of the anomalies, consistent with the conclusion of \citeauthor{Moodera2016}~\cite{Moodera2016}.

Since the conclusion of the experiments in chapter~\ref{ch:bilayer2014} and the publication of \cite{bilayer2014}, crossovers from positive to negative magnetoresistance have been observed in bilayer structures with different TIs and ferromagnets~\cite{Samarth2017, Tian2016}. However, such crossover is absence in most of the available reports on TI--IF bilayers, \textit{e.g.}, refs.~\cite{Shi2014, Petta2014, Wang2014, Tian2016, Qiu2017}. By compiling existing reports on the transport properties of TI--IF bilayers in literature, it was observed that the negative magnetoresistance only arises when the sheet resistance exceeds $h / e^2$, namely the Ioffe-Regel-Mott limit in two-dimensions (figure~\ref{fig:bilayer2018_wl_trend}). The analyses of the temperature dependence of resistance suggest that the negative magnetoresistance occurs in the variable-range hopping r\'egime in presence of Coulomb gaps (chapter~\ref{ch:models}). Such variable-range hopping, however, does not seem to be the sole cause of the negative magnetoresistance in bilayers, for it was similarly observed in bilayer samples that did not show negative magnetoresistance (\textit{e.g.}, figure~\ref{fig:models_bl0}), as well as Bi$_2$Se$_3$ thin films without the EuS layer (figure~\ref{fig:models_ti3}). This suggests that the variable-range hopping r\'egime is necessary, but not sufficient, to observe the negative magnetoresistance.
%
\section{Towards the Half-Integer Quantized Anomalous Hall Effect}
A primary motivation of the research on the interplay between TIs and magnetism have been to realize and to study the integer and half-integer quantized anomalous Hall effects (QAHE), namely transverse conductance in zero applied magnetic field quantized to $\sigma_{xy} = \frac{e^2}{h}$ and $\sigma_{xy} = \frac{e^2}{2h}$, respectively. These effects were predicted for a TI when a gap is opened at the Dirac point by TRS-breaking, and the Fermi level is inside the gap. They are associated with one-dimensional (1D) chiral edge states formed by one of the two spin polarizations, either at the edge of a 2D TI (also known as a quantum spin Hall insulator), or at the boundary of a magnetic domain coupled to a 3D TI's surface state~\cite{TI_Qi, TI_Col, QAH_TI_Yu}. The half-integer case is of particular interest, due to its potential in further experiments studying Majorana fermions~\cite{Akhmerov2009}.

The integer QAHE has been observed so far in TIs doped with magnetic impurities in their bulk, in particular, in chromium (Cr) doped (Bi$_x$Sb$_{1-x}$)$_2$Te$_3$ (BST) thin films~\cite{Chang2013, Kou2014}. In both cases, Cr doping was introduced to break the time-reversal symmetry (TRS) in the BST, with a side effect of altering the bulk properties and Fermi levels.

In the case of \citeauthor{Chang2013}~\cite{Chang2013}, both the Cr:Bi:Sb ratios and back-gating were tuned to control the Fermi levels in a \SI{5}{nm} film. The results were interpreted as a 2D TI case resulted from the hybridization between the top and bottom surfaces of the 3D TI~\cite{QAH_TI_Yu, 2d_crossover}. In such case, in presence of a spontaneous magnetization perpendicular to the film, one of the spin polarization in the edge states forming the quantum spin Hall effect is pushed into the interior of the film, whereas the other polarization remains at the edge and exhibits an integer QAHE with $\sigma_{xy} = \frac{e^2}{h}$.

In the \citeauthor{Chang2013} case~\cite{Chang2013}, the thickness of the film $t=\SI{10}{nm}$ was above the hybridization limit~\cite{ARPES_thickness}, and the Cr:Bi:Sb ratios were precisely tuned to control the Fermi levels. While the authors did not give a complete interpretation of the results, they stated that such configuration is distinctive from the 2D case. Here we note that the direction of the spontaneous magnetization and the sample surface is topologically equivalent to ref.~\cite{TI_Col}: figure~18, that at one side of the film the magnetization is into the bulk, whereas on the opposite side the magnetization is pointing out of the bulk. Since the chiral states at the edges of top and bottom surfaces were only separated by the $t=\SI{10}{nm}$ thickness, and were shorted by the Ohmic contacts, both surfaces were measured, consequently the half-integer quantization of each chiral edge state was doubled to give $\sigma_{xy} = \frac{e^2}{h}$.

The half-integer quantization, and the associated Majorana fermion experiments, are not easily implemented with the bulk doping method. The bilayer approach documented in this thesis, that is to utilize the proximity between the TI and IF layers, in contrast to the magnetic doping approach, is the preferred configuration in the aforementioned theoretical proposals~\cite{TI_Col, Akhmerov2009}. While the negative magnetoresistance presented in this thesis demonstrated a clear proximity effect, and the Hall effect measured sometimes showed non-linearity~\cite{Samarth2017, Petta2014}, a quantization has not been so far observed.

In order to observe a quantized Hall effect, the electric conduction should be dominated by the associated edge current. However, in all of the bilayer configurations, any current at the edge of the TI surface in contact of the IF layer is always in parallel with the opposite surface that is further away from the TI. Due to the aspect ratio between the surface and the edge, it is likely that the electric conduction is mostly though the surface that is less influenced by the IT layer. Hence the Hall effect measured showed limited or no signature of ferromagnetism. To circumvent such limitation, and to fully realize the configuration proposed by ref.~\cite{TI_Col}: figure~18, an improved bilayer design (figure~\ref{fig:improved_bilayer}) may be considered for future experiment.%
%
\begin{figure}[ht]%
    \centering%
    \includegraphics[width=0.95\columnwidth]{figs_misc/improved_bilayer}%
    \caption[Schematic of an improved bilayer device]{\label{fig:improved_bilayer}(Color online)~Schematic of an improved bilayer device. A non-magnetic impurity is sprayed on the top surface of the TI to reduce its longitudinal conductance, hence to increase the proportion of the current in the edge states. During a magnetic field sweep, a domain boundary across a narrow bridge is energetically favored, hence current contacts across such bridge increase the chance of measuring the intended edge state.}%
\end{figure} %
%
While a TI surface state is robust against non-magnetic scatterings, an increase of such scatterings nonetheless reduces the longitudinal conductance. By introducing non-magnetic impurities on the top surface, \text{e.g}, by depositing disordered gold atoms~\cite{TI_WAL_Hongkong}, the edge states at the interface might be made the favorable conduction channel. By having a narrow bridge in the device shape, it is energetically favorable to have the domain boundary across the bridge during a magnetic field sweep. The edge current and the Hall effect thereof can be measured by electrodes across the bridge and across the magnetic domain (figure~\ref{fig:improved_bilayer}b).