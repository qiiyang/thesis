In this thesis, we reported our investigations on the effects due to close proximity between three dimensional topological insulators (TIs) and insulating ferromagnets (IF). Pulsed laser deposition was employed to deposit thin films of the insulating ferromagnet, EuS, and TI-IF thin film bilayers: Bi$_2$Se$_3$--EuS and (Bi$_x$Sb$_{1-x}$)$_2$Te$_3$--EuS.

Instead of the sharp positive magnetoresistance that is observed ubiquitously in thin film TIs, and often interpreted in terms of the weak anti-localization effect, sharp negative magnetoresistance emerges below the Curie temperature of EuS, in both Bi$_2$Se$_3$--EuS bilayers (figure~\ref{fig:bl2014_MR_thickness}) and (Bi$_x$Sb$_{1-x}$)$_2$Te$_3$--EuS bilayers (figure~\ref{fig:bilayer2018_mr}), suggesting an influence of the ferromagnetism in the EuS on the electron transport in the TI layer.

By compiling existing reports on the transport properties of TI-IF bilayers in literature, it was observed that the negative magnetoresistance only arises when the sheet resistance exceeds the von Klitzing constant $h / e^2$, namely the Ioffe-Regel-Mott limit in two-dimensions (figure~\ref{fig:bilayer2018_wl_trend}). The analyses of the temperature dependence of resistance suggest that the negative magnetoresistance occurs well inside the variable-range hopping r\'egime in presence of Coulomb gaps (chapter~\ref{ch:models}).