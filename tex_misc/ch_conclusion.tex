In this thesis, we reported our investigations on the effects due to close proximity between three dimensional topological insulators (TIs) and insulating ferromagnets (IF). Pulsed laser deposition was employed to deposit thin films of the insulating ferromagnet, EuS, and TI--IF thin film bilayers: Bi$_2$Se$_3$--EuS and (Bi$_x$Sb$_{1-x}$)$_2$Te$_3$--EuS. Electron transport was measured with the van der Pauw method, and the magnetic properties by SQUID magnetometry and AC susceptibility measurements. Overall, we observed two stages of such proximity effects: the negative magnetoresistance that emerges below the Curie temperature of bulk EuS, and coincidental magnetic and resistive anomalies that occurred above the Curie temperature around $T\approx30$--$60~\mathrm{K}$.

\paragraph{}
Instead of the sharp positive magnetoresistance that is observed ubiquitously in thin film TIs, often interpreted in terms of the weak antilocalization effect, sharp negative magnetoresistance emerges below the Curie temperature of EuS, in both Bi$_2$Se$_3$--EuS bilayers (chapter~\ref{ch:bilayer2014}) and (Bi$_x$Sb$_{1-x}$)$_2$Te$_3$--EuS bilayers (chapter~\ref{ch:bilayer2018}). Such correlation between the emergence of the negative magnetoresistance and the Curie temperature of EuS strongly suggests an influence of the ferromagnetism in the EuS on the electron transport in the TI layer.

By compiling existing reports on the transport properties of TI--IF bilayers in literature, it was observed that the negative magnetoresistance only arises when the sheet resistance exceeds $h / e^2$, namely the Ioffe-Regel-Mott limit in two-dimensions (figure~\ref{fig:bilayer2018_wl_trend}). The analyses of the temperature dependence of resistance suggest that the negative magnetoresistance occurs in the variable-range hopping r\'egime in presence of Coulomb gaps (chapter~\ref{ch:models}). Such variable-range hopping, however, does not seem to be the sole cause of the negative magnetoresistance in bilayers, for it was similarly observed in bilayer samples that did not show negative magnetoresistance (\textit{e.g.}, figure~\ref{fig:models_bl0}), as well as Bi$_2$Se$_3$ thin films without the EuS layer (figure~\ref{fig:models_ti3}). This suggests that the variable-range hopping r\'egime is necessary, but not sufficient, to observe the negative magnetoresistance.

\paragraph{}
In (Bi$_x$Sb$_{1-x}$)$_2$Te$_3$--EuS bilayers, above the Curie temperature of bulk EuS, magnetic anomalies were observed at $T\approx30~\mathrm{K}$ and $T\approx60~\mathrm{K}$, and resistive transitions were observed at the same temperatures. Such phenomenon was observed exclusively in the thin and optimally-doped (Bi$_x$Sb$_{1-x}$)$_2$Te$_3$--EuS bilayers, where the bulk conduction is minimized, hence the surfaces should dominate the electric conduction. The magnetic anomalies were observed only with AC susceptibility measurements, which are sensitive to surface and interface effects, but not in SQUID measurements, which are dominated by the signal from the bulk of EuS. These combined seem to suggest an interface origin of the anomalies, consistent with the conclusion of \citeauthor{Moodera2016}~\cite{Moodera2016}.

\paragraph{}
While the available data indicate the existence of two stages of proximity effects between the TI and IF, they are not sufficient to conclude the exact nature of such effects. Nevertheless, the increases in resistance when the samples are cooled through the two stages, and the important role of localization in the emergence of the negative magnetoresistance, broadly agree with a general picture that a ferromagnetic order, in proximity to a topological insulator, has the effect of eliminating the topologically protection of the surface Dirac-dispersion.