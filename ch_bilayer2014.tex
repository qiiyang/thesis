\footnote[2]{A part of this chapter is adapted from ref.~\cite{bilayer2014}~\fullcite{bilayer2014}, with permission of the publisher. Copyright (2013) by the American Physical Society.}%
%
At the conclusion of the experiments presented in this chapter and the publication of ref.~\cite{bilayer2014}, probably the most extensively studied 3D-TI hasd been bismuth-selenide (Bi$_2$Se$_3$)~\cite{TI_electronic_structure_zhang, Zhanybek3, TI_other1}, exhibiting crystal structure that consists of atomic quintuple layers (QLs), with three QLs forming a unit cell. As made, uncompensated samples typically have a Fermi level above the Dirac point and intersecting the bulk conduction band~\cite{TI_ARPES1, ARPES_thickness}. In particular, low temperature transport measurements on ungated and uncompensated TI films show positive magnetoresistance (MR) at low magnetic fields and in a wide range of film thicknesses~\cite{ TI_WAL_Hongkong, TI_WAL_thickness, zhangli2012}. This was explained in terms of weak antilocalization (WAL) that results from spin-momentum locking on the surface state Dirac cone~\cite{TI_WAL_Hongkong, WL_theory}. While the inability to account for the bulk bands (presumably because of their low mobility) has challenged this simple assignment, the discovery of weak localization (WL) effects at higher fields~\cite{zhangli2013} and the ability to accurately separate quantum oscillation effects~\cite{Ando_PRL} in high-quality films may provide a first step towards a more comprehensive understanding of transport in these systems.

In this chapter, procedure of pulsed laser deposition of Bi$_2$Se$_3$ on EuS~(100) and Al$_2$O$_3$~(0001) are described, and magneto-transport measurement results are presented, demonstrating a negative magnetoresistance that emerges below the Curie temperature of EuS, in contrast to the ubiquitously observed positive magnetoresistance in topological insulator thin films.

\section{Sample Fabrication and Characterization}
All bilayer (BL) samples presented in this chapter were grown using a pulsed laser deposition (PLD) method. For the bottom layer, EuS thin films of thicknesses 20--200nm were grown on Al$_2$O$_3$ (0001) using the optimal growth procedure outlined in chapter~\ref{ch:EuS}. Titanium Ohmic contacts were subsequently evaporated {\it ex situ} on each EuS thin film through a shadow mask. Electric insulation ($R_\Box > 20~\mathrm{M\Omega}$) was verified for $2~\mathrm{K}<T<300~\mathrm{K}$. Finally, the Bi$_2$Se$_3$ layer was deposited overlapping with the titanium contacts at their inner corners in a van der Pauw configuration (fig.~\ref{fig:bl2014_sketch}).%
%
\begin{figure}[h]%
\centering%
\subfloat{\label{fig:bl2014_sketch}}%
\subfloat{\label{fig:bl2014_TEM}}%
\subfloat{\label{fig:bl2014_Sagnac}}%
\includegraphics[width=0.75\columnwidth]{figs_bilayer2014/char.pdf}%
\caption[Schematic and characterization of Bi$_2$Se$_3$ / EuS thin film bilayers]{\label{fig:bl2014_char}(a)~Schematic of a bilayer device: an EuS film was deposited on an Al$_2$O$_3$(0001) substrate, followed by 15nm Titanium contacts with gradual height on the edges, finally a Bi$_2$Se$_3$ layer were deposited. (b)~Transmission electron micrograph of a cross-section of a bilayer sample, showing the quintuple layers (QL) of Bi$_2$Se$_3$ and a smooth TI-IF interface. (c)~Kerr Angles measured at $T = 10~\mathrm{K}$ by a scanning Sagnac interferometer across the edge of the Bi$_2$Se$_3$ layer, where $x<0$ region is bare EuS and $x>0$ is Bi$_2$Se$_3$ on top of EuS.}%
\end{figure}%

Similarly to the EuS target in section~\ref{sec:EuSPLD}, PLD targets of Bi$_2$Se$_3$ were prepared by SPS~\cite{Jinfeng2, Subhash1},\footnote{Target preparation was carried out by Jinfeng Zhao and Subhash H. Risbud at the Department of Chemical Engineering and Materials Science, University of California, Davis.} and the target surface was similarly polished with a 800~grit diamond sandpaper before transferring to the vacuum chamber. Unlike EuS, the plasma plumes generated by ablating Bi$_2$Se$_3$ in high vacuum are narrow in comparison to the substrate size ($\sim$5~mm). To diffuse the plasma plumes and to prevent potential oxidation from residual gases, 200~millitorr of argon gas mixed with 2\% hydrogen was introduced after pumping to $10^{-6}$ high vacuum. For final target surface treatment and to deposit the Bi$_2$Se$_3$ thin films, the target was ablated by a 25~ns 248~nm KrF excimer pulsed laser beam at 5~Hz repetition rate and 0.54~$\mathrm{J\cdot{}cm^{-2}}$ fluence, while spinning at 18 rpm. The growth rate was estimated to be 0.15~\AA{} per pulse.

