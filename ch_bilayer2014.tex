\footnote[2]{A part of this chapter is adapted from ref.~\cite{bilayer2014}~\fullcite{bilayer2014}, with permission of the publisher. Copyright (2013) by the American Physical Society.}%
%
At the conclusion of the experiments presented in this chapter and the publication of ref.~\cite{bilayer2014}, probably the most extensively studied 3D-TI hasd been bismuth-selenide (Bi$_2$Se$_3$)~\cite{TI_electronic_structure_zhang, Zhanybek3, TI_other1}, exhibiting crystal structure that consists of atomic quintuple layers (QLs), with three QLs forming a unit cell. As made, uncompensated samples typically have a Fermi level above the Dirac point and intersecting the bulk conduction band~\cite{TI_ARPES1, ARPES_thickness}. In particular, low temperature transport measurements on ungated and uncompensated TI films show positive magnetoresistance (MR) at low magnetic fields and in a wide range of film thicknesses~\cite{ TI_WAL_Hongkong, TI_WAL_thickness, zhangli2012}. This was explained in terms of weak antilocalization (WAL) that results from spin-momentum locking on the surface state Dirac cone~\cite{TI_WAL_Hongkong, WL_theory}. While the inability to account for the bulk bands (presumably because of their low mobility) has challenged this simple assignment, the discovery of weak localization (WL) effects at higher fields~\cite{zhangli2013} and the ability to accurately separate quantum oscillation effects~\cite{Ando_PRL} in high-quality films may provide a first step towards a more comprehensive understanding of transport in these systems.

In this chapter, procedure of pulsed laser deposition of Bi$_2$Se$_3$ on EuS~(100) and Al$_2$O$_3$~(0001) are described, and magneto-transport measurement results are presented, demonstrating a negative magnetoresistance that emerges below the Curie temperature of EuS, in contrast to the ubiquitously observed positive magnetoresistance in topological insulator thin films.

\section{Sample Fabrication and Characterization}\label{sec:bilayer2014_char}
All bilayer (BL) samples presented in this chapter were grown using a pulsed laser deposition (PLD) method. For the bottom layer, EuS thin films of thicknesses 20--200nm were grown on Al$_2$O$_3$ (0001) using the optimal growth procedure outlined in Chap.~\ref{ch:EuS}. Titanium Ohmic contacts were subsequently evaporated {\it ex situ} on each EuS thin film through a shadow mask. Electric insulation ($R_\Box > 20~\mathrm{M\Omega}$) was verified for $2~\mathrm{K}<T<300~\mathrm{K}$. Finally, the Bi$_2$Se$_3$ layer was deposited, overlapping with the titanium contacts at their inner corners, forming a van der Pauw configuration (fig.~\ref{fig:bl2014_sketch}).%
%
\begin{figure}[h]%
\centering%
\subfloat{\label{fig:bl2014_sketch}}%
\subfloat{\label{fig:bl2014_TEM}}%
\includegraphics[width=0.75\columnwidth]{figs_bilayer2014/schetch_tem.pdf}%
\caption[Schematic and cross-section TEM of Bi$_2$Se$_3$ / EuS thin film bilayers]{(a)~Schematic of a bilayer device: an EuS film was deposited on an Al$_2$O$_3$(0001) substrate, followed by 15nm Titanium contacts with gradual height on the edges, finally a Bi$_2$Se$_3$ layer were deposited. (b)~Transmission electron micrograph of a cross-section of a bilayer sample, showing the quintuple layers (QL) of Bi$_2$Se$_3$ and a smooth TI-IF interface.}%
\end{figure}%

Similarly to the EuS target in section~\ref{sec:EuSPLD}, PLD targets of Bi$_2$Se$_3$ were prepared by SPS~\cite{Jinfeng2, Subhash1},\footnote{Target preparation was carried out by Jinfeng Zhao and Subhash H. Risbud at the Department of Chemical Engineering and Materials Science, University of California, Davis.}\footnote{PLD recipe for Bi$_2$Se$_3$ was developed together with Li Zhang.} and the target surface was similarly polished with a 800~grit diamond sandpaper before transferring to the vacuum chamber. Unlike EuS, the plasma plumes generated by ablating Bi$_2$Se$_3$ in high vacuum are narrow in comparison to the substrate size ($\sim$5~mm). To diffuse the plasma plumes and to prevent potential oxidation from residual gases, 200~millitorr of argon gas mixed with 2\% hydrogen was introduced after pumping to $10^{-6}$ high vacuum. For final target surface treatment and to deposit the Bi$_2$Se$_3$ thin films, the target was ablated by a 25~ns 248~nm KrF excimer pulsed laser beam at 5~Hz repetition rate and 0.54~$\mathrm{J\cdot{}cm^{-2}}$ fluence, while spinning at 18 rpm. Bare Al$_2$O$_3$(0001) substrates or that with previously deposited EuS were heated to $\SI{150}{\degreeCelsius}$ $\SI{5}{cm}$ away from the target. The growth rate was estimated to be 0.15~\AA{} per pulse.

To compensate for selenium deficiencies that are typical for as-grown Bi$_2$Se$_3$ thin films \cite{Zhanybek3, zhangli2012, zhangli2013, TI_ARPES1, ARPES_thickness}, a selenium capping layer was deposited immediately following the Bi$_2$Se$_3$ layer by ablating a selenium sputtering target (Kurt J. Lesker, 99.999\% pure) for 300 pulses. The sample was annealed with the capping layer at $\SI{150}{\degreeCelsius}$ for roughly 15 minutes before the heater was switched off. When the sample was cooled to below $\SI{80}{\degreeCelsius}$, an additional selenium capping layer was applied before venting to prevent exposure to air.

The cross-section transmission electron micrographs (fig.~\ref{fig:bl2014_TEM}) indicate smooth interface and excellent layering of the Bi$_2$Se$_3$. While the brightness is heightened at the interface in fig.~\ref{fig:bl2014_TEM}, individual atoms are still discernible when the micrograph is presented in digital format. Such heightened brightness is likely artefact resulted from the preparation of the cross section, which involves mechanical polishing followed by argon ion milling. Very different mechanical properties of EuS (cubic lattice with ionic bonds) and Bi$_2$Se$_3$ (quintuple layers weakly bound by van der Waals force) likely result in different response to mechanical polishing, which might be further compounded by different etching rate during ion milling. Similar heightened brightness is observed at the interface between EuS and Al$_2$O$_3$ to a lesser extent (sec.~\ref{sec:EuSPLD}, fig.~\ref{fig:EuS_AFM}).

{\color{red} work later} The crucial component to open a gap at the Dirac point of a topological insulator's surface state is to break the protection of time-reversal symmetry on its surface. In context of  The polar Kerr effect measured by a Sagnac interferometer is a direct probe of time-reversal symmetry breaking~\cite{Xia2006}. Fig.~\ref{fig:bl2014_Fried} shows the Kerr angles at $T=\SI{10}{K}$ measured by a scanning Sagnac interferometer tracing a line across the boundary between a bare EuS region and one covered with a Bi$_2$Se$_3$ layer.\footnote{Sagnac interferometry measurements were carried out by Alexander Fried and Elisabeth Schemm.} %
%
%
\begin{figure}[ht]%
\centering%
\subfloat{\label{fig:bl2014_Fried}}%
\subfloat{\label{fig:bl2014_SchemmT}}%
\subfloat{\label{fig:bl2014_SchemmH}}%
\includegraphics[width=0.75\columnwidth]{figs_bilayer2014/fried.pdf}\\
\includegraphics[width=\columnwidth]{figs_bilayer2014/schemm.pdf}
\caption[Polar Kerr effect in Bi$_2$Se$_3$ / EuS bilayers]{Polar Kerr effect measured by Sagnac interferometers. Kerr angles were measured (a)~at $T=\SI{10}{K}$ as functions of applied magnetic fields and in-plane distance across the edge of the Bi$_2$Se$_3$ layer, where $x<0$ region is bare EuS and $x>0$ is Bi$_2$Se$_3$ on top of EuS; and at a fixed point on Bi$_2$Se$_3$ / EuS bilayer: (b) in zero magnetic field as a function of increasing temperature; (c) at $T=\SI{310}{mK}$ as functions of sweeping magnetic field, where the arrows denote the direction of field sweeps.}%
\end{figure}%
%
The Kerr angle is reduced by the presence of the Bi$_2$Se$_3$ while remains finite in size. At a fixed point on Bi$_2$Se$_3$ covered EuS area, the polar Kerr effect exhibits a transition at the Curie temperature of EuS $T_C\approx\SI{16}{K}$ (fig.~\ref{fig:bl2014_SchemmT}) and hysteresis in a sweeping magnetic field (fig.~\ref{fig:bl2014_SchemmH}), consistent with an origin in ferromagnetism.

\section{Emergent Negative Magnetoresistance}\label{sec:bl2014_negtive_MR}
Hitherto, the data presented in this dissertation demonstrated that the important features necessary for a reliable study of the TI-IF proximity effect, which are an insulating ferromagnet with a perpendicular anisotropy component, and a well defined interface to the TI material, are all met in our bilayer thin films. The transport properties of the bilayer samples were studied using the van der Pauw technique. For shaping the layers in fig.~\ref{fig:bl2014_sketch}, we used shadow masks to produce gradual height profiles and smooth overlapping near the edges of Titanium contacts. The effectiveness was confirmed with scanning electron microscopy (SEM). For comparison, Bi$_2$Se$_3$-only samples were made in each of the PLD sessions in which the Bi$_2$Se$_3$ of bilayers were deposited, with Ohmic contacts between Bi$_2$Se$_3$ and bare Al$_2$O$_3$ (0001) substrates, i.e. identical configuration to fig.~\ref{fig:bl2014_sketch} except for the absence of an EuS layer. To examine the relevance of surface conduction, samples with different thicknesses for the Bi$_2$Se$_3$ layer were fabricated and compared. Table~\ref{tab:bl2014_samples} lists all the samples to be discussed in this chapter. The Bi$_2$Se$_3$ layers on the bilayer (BL) and Bi$_2$Se$_3$-only (TI) samples with the same numerical suffix were deposited in the same batch. %
%
\begin{table}[ht]
    \centering
    \begin{tabularx}{0.6\columnwidth}[t]{l|l|X}
		\hline\hline
        Samples & Configurations & TI Thicknesses\\
        \hline
        TI0 & Bi$_2$Se$_3$ only & 5~nm\\
        TI3 & Bi$_2$Se$_3$ only & 3~nm\\
        BL0 & Bi$_2$Se$_3$ on EuS & 5~nm\\
        BL1 & Bi$_2$Se$_3$ on EuS & 3~nm\\
        BL2 & Bi$_2$Se$_3$ on EuS & 3~nm\\
        BL3 & Bi$_2$Se$_3$ on EuS & 3~nm\\
        BL4 & Bi$_2$Se$_3$ on EuS & 3~nm\\
		\hline\hline
    \end{tabularx}
    \caption[Summary of Bi$_2$Se$_3$ thin films and Bi$_2$Se$_3$ / EuS bilayer samples in Chap.~\ref{ch:bilayer2014}]{\label{tab:bl2014_samples}Summary of Bi$_2$Se$_3$ thin films and Bi$_2$Se$_3$ / EuS bilayer samples in this chapter. All samples are on Al$_2$O$_3$ (0001) substrates. The Bi$_2$Se$_3$ layers on the TI and BL samples with the same numerical suffix were deposited in the same PLD batch.}
\end{table}
Transport properties of two representative thicknesses are shown in fig.~\ref{fig:bl2014_MR_thickness}. %
%
\begin{figure}[h]%
\centering%
\subfloat{\label{fig:TI0_MR}}%
\subfloat{\label{fig:TI3_MR}}%
\subfloat{\label{fig:BL0_MR}}%
\subfloat{\label{fig:BL3_MR}}%
\includegraphics[width=0.75\columnwidth]{figs_bilayer2014/MR_thickness.pdf}%
\caption[Magnetoresistance of PLD-grown Bi$_2$Se$_3$ thin films and Bi$_2$Se$_3$ / EuS bilayers]{Magnetoresistance (MR) and its temperature dependence of PLD-grown Bi$_2$Se$_3$ thin films (TI) and Bi$_2$Se$_3$-EuS bilayer (BL) devices. (a, b)~The Bi$_2$Se$_3$-only samples have positive MR from WAL regardless of thicknesses. (c, d)~The BL devices with TI-layer thicknesses $t\gtrsim4\mathrm{QL}$ behave similarly to the TI-only films, whereas with TI-layer thicknesses $t\lesssim4\mathrm{QL}$ a distinctive WL-like negative MR is observed at low-fields below the Curie temperature of EuS. The thickness limit coincides with occurrence of coupling between the top and the bottom surfaces of a TI thin film.}%
\label{fig:bl2014_MR_thickness}%
\end{figure}%
%
