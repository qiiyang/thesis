\footnote[2]{A part of this chapter is adapted from ref.~\cite{bilayer2014}~\fullcite{bilayer2014}, with permission of the publisher. Copyright (2013) by the American Physical Society.}%
%
At the conclusion of the experiments presented in this chapter and the publication of ref.~\cite{bilayer2014}, probably the most extensively studied 3D-TI hasd been bismuth-selenide (Bi$_2$Se$_3$)~\cite{TI_electronic_structure_zhang, Zhanybek3, TI_other1}, exhibiting crystal structure that consists of atomic quintuple layers (QLs), with three QLs forming a unit cell. As made, uncompensated samples typically have a Fermi level above the Dirac point and intersecting the bulk conduction band~\cite{TI_ARPES1, ARPES_thickness}. In particular, low temperature transport measurements on ungated and uncompensated TI films show positive magnetoresistance (MR) at low magnetic fields and in a wide range of film thicknesses~\cite{ TI_WAL_Hongkong, TI_WAL_thickness, zhangli2012}. This was explained in terms of weak antilocalization (WAL) that results from spin-momentum locking on the surface state Dirac cone~\cite{TI_WAL_Hongkong, WL_theory}. While the inability to account for the bulk bands (presumably because of their low mobility) has challenged this simple assignment, the discovery of weak localization (WL) effects at higher fields~\cite{zhangli2013} and the ability to accurately separate quantum oscillation effects~\cite{Ando_PRL} in high-quality films may provide a first step towards a more comprehensive understanding of transport in these systems.

In this chapter, procedure of pulsed laser deposition of Bi$_2$Se$_3$ on EuS~(100) and Al$_2$O$_3$~(0001) are described, and magneto-transport measurement results are presented, demonstrating a negative magnetoresistance that emerges below the Curie temperature of EuS, in contrast to the ubiquitously observed positive magnetoresistance in topological insulator thin films.

\section{Sample Fabrication and Characterization}\label{sec:bilayer2014_char}
All bilayer (BL) samples presented in this chapter were grown using a pulsed laser deposition (PLD) method. For the bottom layer, EuS thin films of thicknesses 20--200nm were grown on Al$_2$O$_3$ (0001) using the optimal growth procedure outlined in Chap.~\ref{ch:EuS}. Titanium Ohmic contacts were subsequently evaporated {\it ex situ} on each EuS thin film through a shadow mask. Electric insulation ($R_\Box > 20~\mathrm{M\Omega}$) was verified for $2~\mathrm{K}<T<300~\mathrm{K}$. Finally, the Bi$_2$Se$_3$ layer was deposited, overlapping with the titanium contacts at their inner corners, forming a van der Pauw configuration (fig.~\ref{fig:bl2014_sketch}).%
%
\begin{figure}[h]%
\centering%
\subfloat{\label{fig:bl2014_sketch}}%
\subfloat{\label{fig:bl2014_TEM}}%
\includegraphics[width=0.75\columnwidth]{figs_bilayer2014/schetch_tem.pdf}%
\caption[Schematic and cross-section TEM of Bi$_2$Se$_3$ / EuS thin film bilayers]{(a)~Schematic of a bilayer device: an EuS film was deposited on an Al$_2$O$_3$(0001) substrate, followed by 15nm Titanium contacts with gradual height on the edges, finally a Bi$_2$Se$_3$ layer were deposited. (b)~Transmission electron micrograph of a cross-section of a bilayer sample, showing the quintuple layers (QL) of Bi$_2$Se$_3$ and a smooth TI-IF interface.}%
\end{figure}%

Similarly to the EuS target in section~\ref{sec:EuSPLD}, PLD targets of Bi$_2$Se$_3$ were prepared by SPS~\cite{Jinfeng2, Subhash1},\footnote{Target preparation was carried out by Jinfeng Zhao and Subhash H. Risbud at the Department of Chemical Engineering and Materials Science, University of California, Davis.}\footnote{PLD recipe for Bi$_2$Se$_3$ was developed together with Li Zhang.} and the target surface was similarly polished with a 800~grit diamond sandpaper before transferring to the vacuum chamber. Unlike EuS, the plasma plumes generated by ablating Bi$_2$Se$_3$ in high vacuum are narrow in comparison to the substrate size ($\sim$5~mm). To diffuse the plasma plumes and to prevent potential oxidation from residual gases, 200~millitorr of argon gas mixed with 2\% hydrogen was introduced after pumping to $10^{-6}$ high vacuum. For final target surface treatment and to deposit the Bi$_2$Se$_3$ thin films, the target was ablated by a 25~ns 248~nm KrF excimer pulsed laser beam at 5~Hz repetition rate and 0.54~$\mathrm{J\cdot{}cm^{-2}}$ fluence, while spinning at 18 rpm. Bare Al$_2$O$_3$(0001) substrates or that with previously deposited EuS were heated to $\SI{150}{\degreeCelsius}$ $\SI{5}{cm}$ away from the target. The growth rate was estimated to be 0.15~\AA{} per pulse.

To compensate for selenium deficiencies that are typical for as-grown Bi$_2$Se$_3$ thin films \cite{Zhanybek3, zhangli2012, zhangli2013, TI_ARPES1, ARPES_thickness}, a selenium capping layer was deposited immediately following the Bi$_2$Se$_3$ layer by ablating a selenium sputtering target (Kurt J. Lesker, 99.999\% pure) for 300 pulses. The sample was annealed with the capping layer at $\SI{150}{\degreeCelsius}$ for roughly 15 minutes before the heater was switched off. When the sample was cooled to below $\SI{80}{\degreeCelsius}$, an additional selenium capping layer was applied before venting to prevent exposure to air.

The cross-section transmission electron micrographs (fig.~\ref{fig:bl2014_TEM}) indicate smooth interface and excellent layering of the Bi$_2$Se$_3$. While the brightness is heightened at the interface in fig.~\ref{fig:bl2014_TEM}, individual atoms are still discernible when the micrograph is presented in digital format. Such heightened brightness is likely artefact resulted from the preparation of the cross section, which involves mechanical polishing followed by argon ion milling. Very different mechanical properties of EuS (cubic lattice with ionic bonds) and Bi$_2$Se$_3$ (quintuple layers weakly bound by van der Waals force) likely result in different response to mechanical polishing, which might be further compounded by different etching rate during ion milling. Similar heightened brightness is observed at the interface between EuS and Al$_2$O$_3$ to a lesser extent (sec.~\ref{sec:EuSPLD}, fig.~\ref{fig:EuS_AFM}).

{\color{red} work later} The crucial component to open a gap at the Dirac point of a topological insulator's surface state is to break the protection of time-reversal symmetry on its surface. In context of  The polar Kerr effect measured by a Sagnac interferometer is a direct probe of time-reversal symmetry breaking~\cite{Xia2006}. Fig.~\ref{fig:bl2014_Fried} shows the Kerr angles at $T=\SI{10}{K}$ measured by a scanning Sagnac interferometer tracing a line across the boundary between a bare EuS region and one covered with a Bi$_2$Se$_3$ layer.\footnote{Sagnac interferometry measurements were carried out by Alexander Fried and Elisabeth Schemm.} %
%
%
\begin{figure}[ht]%
\centering%
\subfloat{\label{fig:bl2014_Fried}}%
\subfloat{\label{fig:bl2014_SchemmT}}%
\subfloat{\label{fig:bl2014_SchemmH}}%
\includegraphics[width=0.75\columnwidth]{figs_bilayer2014/fried.pdf}\\
\includegraphics[width=\columnwidth]{figs_bilayer2014/schemm.pdf}
\caption[Polar Kerr effect in Bi$_2$Se$_3$ / EuS bilayers]{Polar Kerr effect measured by Sagnac interferometers. Kerr angles were measured (a)~at $T=\SI{10}{K}$ as functions of applied magnetic fields and in-plane distance across the edge of the Bi$_2$Se$_3$ layer, where $x<0$ region is bare EuS and $x>0$ is Bi$_2$Se$_3$ on top of EuS; and at a fixed point on Bi$_2$Se$_3$ / EuS bilayer: (b) in zero magnetic field as a function of increasing temperature; (c) at $T=\SI{310}{mK}$ as functions of sweeping magnetic field, where the arrows denote the direction of field sweeps.}%
\end{figure}%
%
The Kerr angle is reduced by the presence of the Bi$_2$Se$_3$ while remains finite in size. At a fixed point on Bi$_2$Se$_3$ covered EuS area, the polar Kerr effect exhibits a transition at the Curie temperature of EuS $T_C\approx\SI{16}{K}$ (fig.~\ref{fig:bl2014_SchemmT}) and hysteresis in a sweeping magnetic field (fig.~\ref{fig:bl2014_SchemmH}), consistent with an origin in ferromagnetism.

\section{Emergent Negative Magnetoresistance}\label{sec:bl2014_negtive_MR}
Hitherto, the data presented in this dissertation demonstrated that the important features necessary for a reliable study of the TI-IF proximity effect, which are an insulating ferromagnet with a perpendicular anisotropy component, and a well defined interface to the TI material, are all met in our bilayer thin films. The transport properties of the bilayer samples were studied using the van der Pauw technique. For shaping the layers in fig.~\ref{fig:bl2014_sketch}, we used shadow masks to produce gradual height profiles and smooth overlapping near the edges of Titanium contacts. The effectiveness was confirmed with scanning electron microscopy (SEM). For comparison, Bi$_2$Se$_3$-only samples were made in each of the PLD sessions in which the Bi$_2$Se$_3$ of bilayers were deposited, with Ohmic contacts between Bi$_2$Se$_3$ and bare Al$_2$O$_3$ (0001) substrates, i.e. identical configuration to fig.~\ref{fig:bl2014_sketch} except for the absence of an EuS layer. To examine the relevance of surface conduction, samples with different thicknesses for the Bi$_2$Se$_3$ layer were fabricated and compared. Table~\ref{tab:bl2014_samples} lists all the samples to be discussed in this chapter. The Bi$_2$Se$_3$ layers on the bilayer (BL) and Bi$_2$Se$_3$-only (TI) samples with the same numerical suffix were deposited in the same batch.%
%
\begin{table}[ht]
    \centering
    \begin{tabularx}{0.6\columnwidth}[t]{l|l|X}
		\hline\hline
        Samples & Configurations & TI Thicknesses\\
        \hline
        TI0 & Bi$_2$Se$_3$ only & 5~nm\\
        TI3 & Bi$_2$Se$_3$ only & 3~nm\\
        BL0 & Bi$_2$Se$_3$ on EuS & 5~nm\\
        BL1 & Bi$_2$Se$_3$ on EuS & 3~nm\\
        BL2 & Bi$_2$Se$_3$ on EuS & 3~nm\\
        BL3 & Bi$_2$Se$_3$ on EuS & 3~nm\\
        BL4 & Bi$_2$Se$_3$ on EuS & 3~nm\\
		\hline\hline
    \end{tabularx}
    \caption[Summary of Bi$_2$Se$_3$ thin films and Bi$_2$Se$_3$ / EuS bilayer samples in Chap.~\ref{ch:bilayer2014}]{\label{tab:bl2014_samples}Summary of Bi$_2$Se$_3$ thin films and Bi$_2$Se$_3$ / EuS bilayer samples in this chapter. All samples are on Al$_2$O$_3$ (0001) substrates. The Bi$_2$Se$_3$ layers on the TI and BL samples with the same numerical suffix were deposited in the same PLD batch.}
\end{table}

Transport properties of two representative thicknesses are shown in fig.~\ref{fig:bl2014_MR_thickness}. %
%
\begin{figure}[h]%
\centering%
\subfloat{\label{fig:TI0_MR}}%
\subfloat{\label{fig:TI3_MR}}%
\subfloat{\label{fig:BL0_MR}}%
\subfloat{\label{fig:BL3_MR}}%
\includegraphics[width=0.75\columnwidth]{figs_bilayer2014/MR_thickness.pdf}%
\caption[Magnetoresistance of PLD-grown Bi$_2$Se$_3$ thin films and Bi$_2$Se$_3$ / EuS bilayers]{Magnetoresistance (MR) and its temperature dependence of PLD-grown Bi$_2$Se$_3$ thin films (TI) and Bi$_2$Se$_3$-EuS bilayer (BL) devices. (a, b)~The Bi$_2$Se$_3$-only samples have positive MR from WAL regardless of thicknesses. (c, d)~The BL devices with TI-layer thicknesses $t\gtrsim4\mathrm{QL}$ behave similarly to the TI-only films, whereas with TI-layer thicknesses $t\lesssim4\mathrm{QL}$ a distinctive WL-like negative MR is observed at low-fields below the Curie temperature of EuS. The thickness limit coincides with occurrence of coupling between the top and the bottom surfaces of a TI thin film.}%
\label{fig:bl2014_MR_thickness}%
\end{figure}%
%
As ubiquitously seen in TI thin films, the PLD-grown Bi$_2$Se$_3$-only samples show positive MR at low fields regardless of thicknesses, which broadens monotonically with increasing temperature (figs.~\ref{fig:TI0_MR}~\& \ref{fig:TI3_MR}), consistent with weak antilocalization (WAL). Fittings to the standard Hikami-Larkin-Nagaoka (HLN) formula describing the WAL magnetoconductance\cite{WL_HLN, WL_Khmel, WL_theory} yield dephasing lengths ($l_\phi$) comparable to MBE-grown samples with the same thicknesses~\cite{TI_WAL_thickness, zhangli2013}. For thicknesses of the Bi$_2$Se$_3$ layer greater than $\sim4\mathrm{QL}$, the TI-IF bilayers have similar low-field MR features to their TI-only counterparts (fig.~\ref{fig:BL0_MR}). By contrast, for thicknesses $t\lesssim4\mathrm{QL}$, the bilayers show distinctive negative low-field MR at low temperatures (fig.~\ref{fig:BL3_MR}), resembling weak localization (WL) effects. Such negative MR features are clearly distinguishable well below the Curie temperature ($T_C=15.7$K) of the IF (figs.~\ref{fig:bl2014_2K8K} \& \ref{fig:bl2014_8K16K}), %
%
%
\begin{figure}[h]%
\centering%
\subfloat{\label{fig:bl2014_16K30K}}%
\subfloat{\label{fig:bl2014_8K16K}}%
\subfloat{\label{fig:bl2014_2K8K}}%
\subfloat{\label{fig:bl2014_He3}}%
\includegraphics[width=0.75\columnwidth]{figs_bilayer2014/MR_temperature.pdf}%
\caption[Magnetoresistance of Bi$_2$Se$_3$ / EuS bilayer in different temperature r\'egimes]{\label{fig:bl2014_MR_temperature}Low-field MR of a TI-IF bilayer device (BL3, $t\approx3\mathrm{QL}$) in magnetic fields perpendicular to the film: (a)~Above $T_C$, a WAL-like positive MR sharpens with decreasing temperature; (b)~Just below $T_C$, MR broadens with decreasing temperature; (c)~Well below $T_C$, a WL-like negative MR emerges near zero field. The emergence of WL below $T_C$ of the EuS indicates a clear TI-IF proximity effect. (d)~MR of BL3 below $T=2$K, with solid lines as illustrative guides. Resistance at $T\leq0.8\mathrm{K}$ was measured with two-terminal configurations.}
\end{figure}%
%
whereas positive MR from WAL appears at higher fields. Below and close to $T_C$ (fig.~\ref{fig:bl2014_8K16K}), the negative MR can no longer be directly observed. However, its remnant contribution reverses the thermal broadening of the overall positive MR. This suggests that the WL effect is reduced rapidly close to the ferromagnetic transition. Above $T_C$ (fig.~\ref{fig:bl2014_16K30K}), the positive MR is eventually broadened when increasing the temperature, similar to common WAL features in TI-only thin films. The agreement between $T_C$ and the temperatures at which WL becomes dominant strongly indicates a proximity effect between the IF and the TI. In figs.~\ref{fig:bl2014_MR_thickness}~\&~\ref{fig:bl2014_MR_temperature} we presented the negative MR in the same sample, whereas four bilayer samples (labeled as BL1--4) with $t\lesssim4\mathrm{QL}$ from different growth batches all demonstrated such proximity effect in a consistent manner. The thickness criterion ($t\lesssim4\mathrm{QL}$) coincides with the thickness when the two surfaces of a Bi$_2$Se$_3$ film are observed to be coupled~\cite{ARPES_thickness}, suggesting a surface-originated WL mechanism.

Below 2K (fig.~\ref{fig:bl2014_He3}), we observed a continuous sharpening of the low-field WL feature when lowering the temperature, as expected from diminishing thermal dephasing. Unexpectedly, the magnitude of negative MR was reduced when lowering the temperature below 1K. This can be explained by the inhomogeneity observed in the Bi$_2$Se$_3$ layer grown on top of EuS. While the Bi$_2$Se$_3$ thin films grown on bare substrates were verified by AFM and XRD to be adequately uniform in thickness, the TEM images taken at different locations on the cross-section of the BL3 sample show large variations ($\pm2\mathrm{QL}$) in thickness of the Bi$_2$Se$_3$ layer, with an estimated mean value consistent with the thickness of the Bi$_2$Se$_3$-only film grown in the same PLD session. For the thicknesses of interest ($t\lesssim4\mathrm{QL}$), both the resistance of Bi$_2$Se$_3$ films and its temperature dependence are known to change sharply with thickness at low temperatures~\cite{TI_WAL_thickness}. Thus special difficulty is introduced when measuring the sheet resistance with a van der Pauw method, where the sample thickness is assumed to be uniform.\cite{VdP1958, VdP_contact_size} With such inhomogeneous geometry, electric conduction is limited by the thinner and therefore more resistive parts of the sample. Indeed, the sheet resistance of the BL samples are one order of magnitude higher than that of the Bi$_2$Se$_3$-only samples with similar thicknesses (fig.~\ref{fig:bl2014_RvT}). %
%
\begin{figure}[ht]%
\centering%
\subfloat{\label{fig:bl2014_angular}}%
\subfloat{\label{fig:bl2014_RvT}}%
\includegraphics[width=0.75\columnwidth]{figs_bilayer2014/4.pdf}%
\caption[Angular dependence and temperature dependence of the transport properties of Bi$_2$Se$_3$ / EuS bilayers]{(a)~The WL and WAL features in BL4 broaden with increasing angles between the magnetic field and the normal of the sample plane, confirming their orbital origins. The insert is zoomed-in version of low-field features. (b)~Sheet resistances of the four TI-IF bilayer samples with WL features (BL1--4) and a PLD-grown TI-only film (TI3) measured with the van der Pauw method. The non-vanishing MR near parallel fields and the high resistances of the BL samples are consistent with the observed inhomogeneity in sample thickness.}%
\end{figure}%
%
When lowering the temperature, the change in the zero-field resistance, $R(0)$, is dominated by the sharp temperature dependence of the thinner parts of the sample. The low-field features in magnetoresistance, however, are mainly contributed by the thicker parts, where the dephasing length is known to be longer~\cite{TI_WAL_thickness}. Therefore, after dividing by $R(0)$ as the normalization factor, $\Delta{}R(H)/R(0)$ appears with a reduced magnitude. On the other hand, measuring the Hall effect, which is insensitive to film inhomogeneity~\cite{Landauer_Porous_Media}, the sheet carrier density is calculated showing similar values (\(n_{2D}\approx2\times10^{13}\mathrm{cm^{-2}}\)) for both TI-only films and TI-IF bilayers and are very close to reported values for MBE films~\cite{TI_WAL_thickness}.

So far in our discussions, we assumed that the low-field negative magnetoresistance has an orbital origin such as WL effects suggested by recent theories~\cite{WL_WAL_competition, WL_Glazman, WL_bulk_Lu}. The presence of a magnetic material brings possibilities of MR originated from scatterings at localized spins~\cite{KondoMR}. However, the magnitude of any spin-introduced MR should be greater above $T_C$ and should diminish rapidly below $T_C$ as the spins being aligned during the ferromagnetic transition~\cite{SpinMagnetic}, which is contrary to what we observed. Moreover, we show in fig.~\ref{fig:bl2014_angular} the broadening of he negative MR  as the sample is rotated  from perpendicular to near parallel applied magnetic field, clearly indicating its orbital origin. The fact that some negative  MR features remain finite in near-parallel field is likely due to the uneven thickness in the Bi$_2$Se$_3$ layer as a consequence of a locally slanted top surface, on which local electron transport may have a finite angle to the magnetic field that is parallel to the sample plane. In fact, when the magnetic field is scaled with an effective angle $\theta^*$ between magnetic flux and the normal of local electron transport, both the WL at low fields and the WAL at higher fields are found to coincide well with perpendicular-field MR. The difference between the effective angles ($\theta^*$) and the nominal angles read from the instrument ($\theta$) is small for most angles ($|\theta-\theta^*|\leq{}3^\circ$ for $\theta\leq{}60^\circ$) and increasing towards parallel field ($\theta^*\approx{}75^\circ$ for $\theta=89^\circ$). This is consistent with expectations from a film with a smooth bottom surface and an uneven top surface.

Theory predicts gap-opening at the Dirac point as a result of proximity between TI and IF~\cite{QAH_TI_Yu}. Such gap-opening is expected to result in WL only when the Fermi level is sufficiently near the gap~\cite{WL_WAL_competition, WL_Glazman, WL_bulk_Lu}. However, as common in ungated and uncompensated Bi$_2$Se$_3$, our carrier densities suggest a Fermi level intersecting the conduction band and therefore far away from the Dirac point. Recent calculations suggest that surface charges on the IF layer may result in band-bending at the interface~\cite{MnSe}, hence provide a possible mechanism to move the gap towards the Fermi level. However, whether it is the case depends on the details of atomic arrangement at the interface~\cite{xiaoliang}, which cannot be determined with available data. On the other hand, low levels of sulfur doping are known to modify the band structure of Bi$_2$Se$_3$ while preserving its structural phase~\cite{Bi2Se3S, BiSeS}. Such effect may as well be present at the Bi$_2$Se$_3$-EuS interface. Attempts to perform ARPES measurements to determine the band gap, location of the Dirac point and the Fermi level produced inconclusive results, primarily due to the inhomogeneity in films thickness.  We note that a recent study of EuS / Bi$_2$Se$_3$ bilayers show weak tendency towards anomalous Hall effect which was argued to indicate  emergence of a ferromagnetic phase in TI~\cite{Moodera2013}. Samples in that study consisted of a bottom Bi$_2$Se$_3$ film and a top EuS film, making it impossible to determine whether the EuS is truly insulating. We contrast it with the present approach in which we first obtained high-quality, well characterized EuS films, on which the TI film was deposited. Since a true 3D TI will have one and only one surface state, irrespective of inhomogeneities and local defects, we prefer this approach and further argue that the unusual negative magnetoresistance at low fields below the Curie temperature of the ferromagnet are strong indication of a proximity effect between a topological insulator and an insulating ferromagnet.