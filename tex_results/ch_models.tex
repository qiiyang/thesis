Given the spin-momentum locking in the surface states, the electric conduction in topological insulator thin films is typically discussed in literature in terms of the weak antilocalization effect. While the weak antilocalization explains well the positive magnetoresistance that is ubiquitously observed in topological insulator thin films, two discrepancies have been observed at low temperatures. Firstly, the weak antilocalization effect should increase the conductivity as the temperature is lowered~\cite{bergmann1984}; however, a reduction of conductivity is observed often in thin films with thicknesses $t \lesssim 10~\mathrm{nm}$, with the conductivity varies logarithmically with the temperature~\cite{Chen2011, Liu2011, Roy2013}, similarly to that described by the weak localization (the opposite) effect (equation~\ref{eq:wl_T}). This has been attributed to the effect of electron-electron interaction in the diffusive r\'egime in two dimensions~\cite{WL_ee}. Secondly, the weak antilocalization effect is applicable only when the sheet resistance is much below the Mott-Ioffe-Regel limit in two-dimensions ($R_\Box \ll h/e^2$, \textit{q.v.} section~\ref{sec:wl}); however, the resistance of topological insulator thin films sometimes approaches or exceeds such limit~\cite{TI_WAL_thickness, ZhangJS2011}.

In chapter~\ref{ch:bilayer2018}, we remarked that such high resistance ($R_\Box \gtrsim h/e^2$) was othe commonality amongst TI-ferromagnet thin film bilayers that show negative magnetoresistance (figure~\ref{fig:bilayer2018_wl_trend}). After the publication of \cite{bilayer2018}, we were brought to attention to ref.~\cite{liao2015}, in which \citeauthor{liao2015} analyzed the temperature dependence of resistance of \SI{2}{nm} (Bi$_x$Sb$_{1-x}$)$_2$Te$_3$ thin films, demonstrating that by reducing the thicknesses of thin films, electric conduction undergoes a crossover from the diffusive r\'egime with electron-electron interaction to the variable-range hopping r\'egime. These suggest that the temperature dependence of resistance should be similarly analyzed for the thin film samples in this thesis that approach or exceed the Mott-Ioffe-Regel limit.

Both topological insulator materials presented in this thesis should be considered doped semiconductors. In the case of Bi$_2$Se$_3$, perfect crystalline samples should have a Fermi level in the bulk band gap~\cite{TI_electronic_structure_zhang}; whereas as-made thin films typically exhibit elevated Fermi levels that intersect with the bulk conduction band, primarily due to interstitials and vacancies~\cite{TI_ARPES1, ARPES_thickness, zhangli2013, Zhanybek3, Fisher2010}. In the case of (Bi$_x$Sb$_{1-x}$)$_2$Te$_3$, on the other hand, the difference between the ionization energy of Bi$^{3+}$ and that of Sb$^{3+}$ is utilized to move the Fermi level into the bulk band gap, typically with Bi$^{3+}$ serving as substitutional dopants~\cite{ZhangJS2011, TI_electronic_structure_zhang}.

The electric conduction modes in doped semiconductors discussed in ref~\cite{schklovskii_efros} and section~\ref{sec:vrh} are, therefore, relevant to the thin film samples in this thesis. Here we will test four conduction models against the temperature dependence of resistance of our samples: the diffusive conduction with electron-electron interaction (EE) that was found relevant in refs.~\cite{Chen2011, Liu2011, Roy2013, liao2015}, both Mott type and Efros-Shklovskii type variable-range hopping (VRH) models, as well a thermal activation model that might describe the crossover between the variable-range hopping and fully diffusive r\'egimes (equation~\ref{eq:activation}, and ref.~\cite{schklovskii_efros}: chapter 4). The temperature dependence of sheet conductivity resulted from these models in two-dimensions are discussed in section~\ref{sec:vrh}, and summarized below in equations~\ref{eq:modes}:%
\begin{equation}\label{eq:modes}
    \sigma_\Box(T) \propto %
    \begin{cases}
        \log(T / T_0)  & \quad \text{(diffusive with EE, \cite{WL_ee})}\\
        e^{-\frac{\varepsilon_1}{k_B T}}   & \quad \text{(thermal activation, equation~\ref{eq:activation})}\\
        e^{-\left(\frac{T}{T_0}\right)^{-\frac{1}{3}}}   & \quad \text{(Mott type VRH, equation~\ref{eq:vrh_mott})}\\
        e^{-\left(\frac{T}{T_0}\right)^{-\frac{1}{2}}} & \quad \text{(Efros-Shklovskii type VRH, equation~\ref{eq:vrh_efros})}~.
    \end{cases}
\end{equation}%

To test how well each of these models describes the temperature dependence of the thin film samples presented in earlier chapters, the equations~\ref{eq:modes} are linearized into the form $y=ax+b$ (table~\ref{tab:models}), and the relevance of each model should be reflected in the linearity in its respective data plots.%
\begin{table}[ht]
    \centering
    \begin{tabularx}{0.95\columnwidth}[t]{l|l|l|X}
    \caption[Linearized formulae for testing conduction models]{\label{tab:models}Linearized formulae for conduction models. The shorthands will be used to annotate figures.}\\
		\hline\hline
        Conduction models & $x$ & $y$ & Shorthands\\
        \hline%
        diffusive with electron-electron interaction & $\log(T)$ & $\sigma_\Box$ & EE\\
        thermal activation & $T^{-1}$ & $\log\left(R_\Box\right)$ & TA\\
        Mott type variable-range hopping & $T^{-1/3}$ & $\log\left(R_\Box\right)$ & M--VRH\\
        Efros-Shklovskii type variable-range hopping & $T^{-1/2}$ & $\log\left(R_\Box\right)$ & ES--VRH\\
		\hline\hline
    \end{tabularx}
\end{table} %

\FloatBarrier%
\section{Bi$_2$Se$_3$--EuS Bilayers and Bi$_2$Se$_3$ Thin Films}
In this section, the models listed in table~\ref{tab:models} will be tested against the temperature dependence of the resistance of the Bi$_2$Se$_3$--EuS bilayers, and Bi$_2$Se$_3$ thin films without EuS, presented in chapter~\ref{ch:bilayer2014}. Descriptions of these samples are reiterated in table~\ref{tab:models_bl2014}, with remarks on whether a negative magnetoresistance ($-$ve MR) was observed in each sample.%
\begin{table}[ht]
    \centering
    \begin{tabularx}{0.75\columnwidth}[t]{l|l|l|X}
    \caption[Summary of Bi$_2$Se$_3$ thin films and Bi$_2$Se$_3$--EuS bilayers]{\label{tab:models_bl2014}Summary of Bi$_2$Se$_3$ thin films and Bi$_2$Se$_3$--EuS bilayers in chapter~\ref{ch:bilayer2014}.}\\
		\hline\hline
        Samples & Configurations & TI Thicknesses & $-$ve MR\\
        \hline
        TI0 & Bi$_2$Se$_3$ only & 5~nm & none\\
        TI3 & Bi$_2$Se$_3$ only & 3~nm & none\\
        BL0 & Bi$_2$Se$_3$ on EuS & 5~nm & none\\
        BL1 & Bi$_2$Se$_3$ on EuS & 3~nm & observed\\
        BL2 & Bi$_2$Se$_3$ on EuS & 3~nm & observed\\
        BL3 & Bi$_2$Se$_3$ on EuS & 3~nm & observed\\
        BL4 & Bi$_2$Se$_3$ on EuS & 3~nm & observed\\
		\hline\hline
    \end{tabularx}
\end{table}%

The models are first compared for the four Bi$_2$Se$_3$--EuS bilayer samples that showed negative magnetoresistance below the Curie temperature of EuS (BL1--3, figure~\ref{fig:models_bls}).%
\begin{figure}[ht]%
    \centering%
    \includegraphics[width=0.95\columnwidth]{figs_models/bls}%
    \caption[Conduction model comparison: Bi$_2$Se$_3$--EuS bilayers: BL1--BL4]{\label{fig:models_bls}(Color online) Comparison between conduction models for Bi$_2$Se$_3$--EuS bilayer samples in chapter~\ref{ch:bilayer2014} that showed negative magnetoresistance below the Curie temperature of EuS (BL1--4): (a)~diffusive with electron-electron interaction, (b)~thermal activation, (c)~Mott type variable-range hopping, (e)~Efros-Shklovskii type variable-range hopping, which seems to describe the behavior of sheet resistance for the largest temperature range.}%
\end{figure} %
%
These samples all have the Bi$_2$Se$_3$ layers around \SI{3}{nm} thick, and showed sheet resistance comparable to $h/e^2$ at room temperature: $R_\Box(T=300~\mathrm{K}) \sim h/e^2$. Their sheet resistance increases at low temperatures, becoming up to two orders of magnitude higher resistance at $T = 2~\mathrm{K}$. Among the four models, the diffusive model with electron-electron interaction (figure~\ref{fig:models_bls}a), and the thermal activation model (figure~\ref{fig:models_bls}b) do not display a clear linearity in any significant temperature range. Between the variable-range hopping models (figures~\ref{fig:models_bls}c~\&~d), the Efros-Shklovskii type seems to fit in a wide temperature range $3~\mathrm{K} \lesssim T \lesssim 100~\mathrm{K}$. Below $T \approx 3~\mathrm{K}$, the samples show resistance values that deviate from each other, and from the Efros-Shklovskii model. This is possibly an artefact introduced by inhomogeneity in sample thickness, as discussed in chapter~\ref{ch:bilayer2014}.

The Bi$_2$Se$_3$--EuS bilayer sample, BL0, has a \SI{5}{nm} Bi$_2$Se$_3$ layer, and only showed positive magnetoresistance (figure~\ref{fig:TI0_MR}). Its sheet resistance is near the $h/e^2$ for the entire $2~\mathrm{k} < T < 300~\mathrm{K}$. In this sample, the comparison is less clear (figure~\ref{fig:models_bl0}).%
\begin{figure}[ht]%
    \centering%
    \includegraphics[width=0.95\columnwidth]{figs_models/bl0}%
    \caption[Conduction model comparison: Bi$_2$Se$_3$--EuS bilayer: BL0]{\label{fig:models_bl0}(Color online) Comparison between conduction models for Bi$_2$Se$_3$--EuS bilayer sample in chapter~\ref{ch:bilayer2014} that only showed positive magnetoresistance. Both the diffusive model with electron-electron interaction (b) and Efros-Shklovskii type variable-range hopping (d) seem to describe the low-temperature parts of the data reasonably well.}%
\end{figure} %
%
Both the thermal activation model (figure~\ref{fig:models_bl0}b) and Efros-Shklovskii type variable-range hopping (figure~\ref{fig:models_bl0}d) seem to describe the low-temperature parts of the data ($T \lesssim 30~\mathrm{K}$ and $T \lesssim 10~\mathrm{K}$, respectively) reasonably well. This might be an indication that, as the thickness increases, this sample resides within the cross-over between the variable-range hopping and the diffusive conduction r\'egimes.

The Bi$_2$Se$_3$ thin films on Al$_2$O$_3$~(0001) without EuS, TI0 and TI3, have considerably lower sheet resistance than that of their bilayer counterparts BL0 and BL3 (figure~\ref{fig:models_tis_rt}).%
\begin{figure}[ht]%
    \centering%
    \includegraphics[width=0.95\columnwidth]{figs_models/tis_rt}%
    \caption[Temperature dependence of sheet resistance of Bi$_2$Se$_3$ thin films: TI0~\&~TI3]{\label{fig:models_tis_rt}(Color online) Temperature dependence of sheet resistance of Bi$_2$Se$_3$ thin films: (a)~TI0: \SI{5}{nm}-thick, counterpart of BL0; (b)~TI3: \SI{3}{nm}-thick, counterpart of BL3.}%
\end{figure} %
%
The resistance of the \SI{5}{nm} Bi$_2$Se$_3$ thin film, TI0, is an order of magnitude below $h/e^2$, that decreases as the temperature lowers (figure~\ref{fig:models_tis_rt}a). Therefore none of the models in table~\ref{tab:models} applies for TI0. The resistance of the \SI{3}{nm} film, TI3, on the other hand, is near $h/e^2$, and increases at lower temperatures. The respective data plots of the models for TI3 are shown in figure~\ref{fig:models_ti3}.%
\begin{figure}[ht]%
    \centering%
    \includegraphics[width=0.95\columnwidth]{figs_models/ti3}%
    \caption[Conduction model comparison: Bi$_2$Se$_3$ thin film: TI3]{\label{fig:models_ti3}(Color online) Comparison between conduction models for the \SI{3}{nm} Bi$_2$Se$_3$ thin film, TI3, namely the counterpart of BL3. While both types of variable-range hopping models seem to display reasonable linearity, the Efros-Shklovskii type (d) seems to account for the temperature dependence at low temperatures ($T \lesssim 50~\mathrm{K}$) in the most satisfactory manner.}%
\end{figure} %
%
In this case, while both types of variable-range hopping models seem to display reasonable linearity, the Efros-Shklovskii type (d) seems to account for the temperature dependence at low temperatures ($T \lesssim 50~\mathrm{K}$) in the most satisfactory manner.

The results of conduction model comparisons for the Bi$_2$Se$_3$--EuS bilayers and Bi$_2$Se$_3$ thin films are summarized in table~\ref{tab:models_bl2014_results}. The models which displayed reasonable linearity in their respective data plots for each sample are marked with check marks (\checkmark).

\begin{minipage}{0.95\columnwidth}
    \begin{tabularx}{1\columnwidth}[t]{l|l|l|l|l|l|l|X}
    \caption[Conduction model comparisons for Bi$_2$Se$_3$ thin films and Bi$_2$Se$_3$--EuS bilayers]{\label{tab:models_bl2014_results}Conduction model comparisons for Bi$_2$Se$_3$ thin films and Bi$_2$Se$_3$--EuS bilayers in chapter~\ref{ch:bilayer2014}.}\\
        \hline\hline
        Samples & Configurations & TI & EE & TA & M--VRH & ES--VRH & $-$ve MR\\
        \hline
        TI0 & Bi$_2$Se$_3$ only & 5~nm & ~ & ~ & ~ & ~ & none\\
        TI3 & Bi$_2$Se$_3$ only & 3~nm & ~ & ~ & \checkmark & \checkmark & none\\
        BL0 & Bi$_2$Se$_3$ on EuS & 5~nm & ~ & \checkmark & ~ & \checkmark & none\\
        BL1 & Bi$_2$Se$_3$ on EuS & 3~nm & ~ & ~ & ~ & \checkmark & observed\\
        BL2 & Bi$_2$Se$_3$ on EuS & 3~nm & ~ & ~ & ~ & \checkmark & observed\\
        BL3 & Bi$_2$Se$_3$ on EuS & 3~nm & ~ & ~ & ~ & \checkmark & observed\\
        BL4 & Bi$_2$Se$_3$ on EuS & 3~nm & ~ & ~ & ~ & \checkmark & observed\\
        \hline\hline
    \end{tabularx}
\end{minipage}

\FloatBarrier%
\section{(Bi$_x$Sb$_{1-x}$)$_2$Te$_3$--EuS Bilayers}
Descriptions of the (Bi$_x$Sb$_{1-x}$)$_2$Te$_3$--EuS bilayers presented in chapter~\ref{ch:bilayer2018} are reiterated in table~\ref{tab:models_bl2018}, with remarks on whether a negative magnetoresistance ($-$ve MR) was observed in each sample.%
%
\begin{table}[ht]
    \centering
    \begin{tabularx}{0.75\columnwidth}[t]{l|l|l|l|X}
    \caption[Summary of (Bi$_x$Sb$_{1-x}$)$_2$Te$_3$--EuS bilayers]{\label{tab:models_bl2018}Summary of (Bi$_x$Sb$_{1-x}$)$_2$Te$_3$--EuS bilayers in chapter~\ref{ch:bilayer2018}.}\\
		\hline\hline
        Samples & Ferromagnet & \multicolumn{2}{X|}{TI layers} & $-$ve MR\\
        \hline%
        S1 & EuS (100) & (Bi$_{0.05}$Sb$_{0.95}$)$_2$Te$_3$ & 4~nm & fully observed\\
        S2 & EuS (100) & (Bi$_{0.05}$Sb$_{0.95}$)$_2$Te$_3$ & 6.5~nm & trace observed\\
        S3 & EuS (100) & Sb$_2$Te$_3$ & 13~nm & none\\
        S4 & EuS (100) & Sb$_2$Te$_3$ & 65~nm & none\\
		\hline\hline
    \end{tabularx}
\end{table} %
Since some of the (Bi$_x$Sb$_{1-x}$)$_2$Te$_3$--EuS bilayer samples presented in chapter~\ref{ch:bilayer2018} undergo transitions at $T\approx 30~\mathrm{K}$ and at $T \approx 60~\mathrm{K}$, the analyses on their sheet resistance will be focused on the temperature range $T < 30~\mathrm{K}$. The thickest, undoped sample, S4, has sheet resistance three orders of magnitudes below $h/e^2$, and exhibits a negative dependence on temperature, therefore none of the models in table~\ref{tab:models} applies. The conduction models are therefore tested on the other three samples, S1--S3, that showed insulating behavior at low temperatures.

Both of the thin, optimally doped samples, S1 and S2, show sheet resistance slightly below $h/e^2$ at $T = 30~\mathrm{K}$, which increases at the temperature lowers, and exceeds $h/e^2$ at $T = 2~\mathrm{K}$. In both cases (figures~\ref{fig:models_s1} \& \ref{fig:models_s2}),%
%
\begin{figure}[ht]%
    \centering%
    \includegraphics[width=0.95\columnwidth]{figs_models/s1}%
    \caption[Conduction model comparison: (Bi$_x$Sb$_{1-x}$)$_2$Te$_3$--EuS bilayer: S1]{\label{fig:models_s1}(Color online) Comparison between conduction models for the (Bi$_x$Sb$_{1-x}$)$_2$Te$_3$--EuS bilayer sample, S1.}%
\end{figure}%
%
\begin{figure}[ht]%
    \centering%
    \includegraphics[width=0.95\columnwidth]{figs_models/s2}%
    \caption[Conduction model comparison: (Bi$_x$Sb$_{1-x}$)$_2$Te$_3$--EuS bilayer: S2]{\label{fig:models_s2}(Color online) Comparison between conduction models for the (Bi$_x$Sb$_{1-x}$)$_2$Te$_3$--EuS bilayer sample, S2.}%
\end{figure} %
%
the Efros-Shklovskii type variable-range hopping model displays linearity for almost the entire temperature range, $T < 30~\mathrm{K}$. The difference between the Mott type and the Efros-Shklovskii type is, however, marginal in the case of S1. The other two models (EE \& TA) both showed distinguishable curvatures in their respective plots.

The bilayer with \SI{13}{nm} undoped Sb$_2$Te$_3$ layer, S3, has sheet resistance almost two orders of magnitudes below $h/e^2$. Its sheet resistance shows a general trend of negative dependence on temperature. However, roughly below the Curie temperature of the EuS layer, $T_C = 14.5~\mathrm{K}$, the resistance starts to increase with decreasing temperature (figure~\ref{fig:models_s3}).%
%
\begin{figure}[ht]%
    \centering%
    \includegraphics[width=0.95\columnwidth]{figs_models/s3}%
    \caption[Conduction model comparison: (Bi$_x$Sb$_{1-x}$)$_2$Te$_3$--EuS bilayer: S3]{\label{fig:models_s3}(Color online) Comparison between conduction models for the (Bi$_x$Sb$_{1-x}$)$_2$Te$_3$--EuS bilayer sample, S3. The vertical broken line mark the Curie temperature of the EuS layer, $T_C = 14.5~\mathrm{K}$. For the insulating behavior below $T_C$, the thermal activation model (b), and both types of variable-range hopping (b \& c) all seem to describe the temperature dependence reasonably well.}%
\end{figure} %
For such insulating behavior below $T_C$, the thermal activation model (figure~\ref{fig:models_s3}b), and both types of variable-range hopping (figure~\ref{fig:models_s3}b \& c) all seem to describe the temperature dependence reasonably well, suggesting a crossover between the variable-range hopping and the diffusive conduction r\'egimes.

The results of conduction model comparisons for the (Bi$_x$Sb$_{1-x}$)$_2$Te$_3$--EuS bilayers are summarized in table~\ref{tab:models_bl2018_results}. The models which displayed reasonable linearity in their respective data plots for each sample are marked with check marks (\checkmark).%
%
\begin{table}[ht]
    \centering
    \begin{tabularx}{1\columnwidth}[t]{l|l|l|l|l|l|l|l|X}
    \caption[Conduction model comparisons for (Bi$_x$Sb$_{1-x}$)$_2$Te$_3$--EuS bilayers]{\label{tab:models_bl2018_results}Conduction model comparisons for (Bi$_x$Sb$_{1-x}$)$_2$Te$_3$--EuS bilayers in chapter~\ref{ch:bilayer2018}.}\\
		\hline\hline
        Samples & Ferromagnet & \multicolumn{2}{X|}{TI layers} & EE & TA & M--VRH & ES--VRH & $-$ve MR\\
        \hline%
        S1 & EuS (100) & doped & 4~nm & ~ & ~ & \checkmark & \checkmark & fully observed\\
        S2 & EuS (100) & doped & 6.5~nm & ~ & ~ & ~ & \checkmark & trace observed\\
        S3 & EuS (100) & undoped & 13~nm & ~ & \checkmark & \checkmark & \checkmark & none\\
        S4 & EuS (100) & undoped & 65~nm & ~ & ~ & ~ & ~ & none\\
		\hline\hline
    \end{tabularx}
\end{table} %

\FloatBarrier%
\section{Discussion}
In chapter~\ref{ch:bilayer2018}, we suggested that the negative magnetoresistance observed in topological insulator-insulating ferromagnet bilayers seem to arise from a conduction r\'egime in which localization effects play dominant roles. Indeed, the analyses in this chapter support such conclusion, and suggest that the electric conduction in samples that display negative magnetoresistance is best described by the variable-range hopping model. The variable-range hopping mode of conduction is, however, not only observed in bilayer samples where negative magnetoresistance occurs, but also in bilayers where the negative magnetoresistance is absence (figures~\ref{fig:models_bl0}~\&~\ref{fig:models_s3}) as well as in the \SI{3}{nm} Bi$_2$Se$_3$ thin film without EuS (figure~\ref{fig:models_ti3}). This suggests that variable-range hopping is a necessary but not sufficient condition for the negative magnetoresistance to emerge. It should be noted, however, that the scalings of temperature, $T^{-1}$, $T^{-1/3}$, and $T^{-1/2}$, give more weight to data points at lower temperatures in their respective plots. Thus the qualitative observations regarding the linearity in the data plots were made only considering a relatively small temperature range: from \SI{2}{K} to a few tens of Kelvins. To ascertain the accuracy of the such observations, further measurements on similar samples at lower temperatures would be helpful.

Between the two types of variable-range hopping, overall, the Efros-Shklovskii type seems to fit the data more closely than the Mott type, although the difference is marginal in some cases. This suggests that the electron-electron interaction observed in the diffusive conduction r\'egime~\cite{Chen2011, Liu2011, Roy2013} persists in the variable-range hopping r\'egime. In both cases of Bi$_2$Se$_3$--EuS bilayers and (Bi$_x$Sb$_{1-x}$)$_2$Te$_3$--EuS bilayers, as the thicknesses of the TI layer increase, the differences between how well the thermal activation model, and the variable-range hopping models fit the data seem to decrease. This might be a manifestation of the crossover from the variable-range hopping conduction mode to the diffusive mode suggested in ref.~\cite{liao2015}.

However, there are a few key differences between the observations in ref.~\cite{liao2015} and that in this chapter. Firstly, in ref.~\cite{liao2015} the variable-range hopping r\'egime was observed only when the TI thin films are as thin as \SI{2}{nm}, and their sheet resistance well exceeds the Mott-Ioffe-Regel limit, $R_\Box \gtrsim 4h/e^2$; whereas in this chapter, the Efros-Shklovskii type variable-range hopping was observed even when the films are 4--\SI{6}{nm} thick, and the sample resistance is near or below $h/e^2$ (figures~\ref{fig:models_ti3}--\ref{fig:models_s3}). Secondly, the temperature dependence that is fully described by the electron-electron interaction in the diffusive model, $\sigma(T) \propto \log(T)$, was observed in ref.~\cite{liao2015}, but not in any of our samples. Such differences cannot be explained by the difference in the temperature range presented in ref.~\cite{liao2015} (0.3--4~K), and those in this chapter (2--300~K or 2--30~K), for variable-range hopping is expected to occur at lower temperatures compared to the diffusive conduction~\cite[ch.~4]{schklovskii_efros}. In the case of Bi$_2$Se$_3$, such differences might be caused by different material properties. While the PLD--grown Bi$_2$Se$_3$ in this thesis in general showed similar transport properties to MBE--grown samples, which also showed insulating behavior for thicknesses $t \lesssim 10~\mathrm{nm}$, with the sheet resistance approaches or exceeds the $h/e^2$~\cite{TI_WAL_thickness}; to our knowledge, their properties have not been tested systematically against the variable-range hopping model. In the case of (Bi$_x$Sb$_{1-x}$)$_2$Te$_3$--EuS bilayers, the correlation between the Curie temperature of the EuS layer and sample S3 becoming insulating seems to suggest an effect of the ferromagnetic order on the (Bi$_x$Sb$_{1-x}$)$_2$Te$_3$ layer.

A negative magnetoresistance was observed in ref.~\cite{liao2015} and was suggested to be a universal property of the strongly localized r\'egime. However, such negative magnetoresistance only occurred at very high resistance ($R_\Box \gtrsim 1~\mathrm{M}\Omega$), and its temperature dependence and angular dependence were not reported. In contrast, the negative magnetoresistance reported in this thesis occurred when the sheet resistance was around $R_\Box \approx h/e^2 \approx 26~\mathrm{k}\Omega$, and its appearance correlates with the ferromagnetic transition in the EuS (figures~\ref{fig:bl2014_MR_temperature} \& \ref{fig:bl2014_quadratic}). While the relation between the two remains elusive given the limited available data, considering the correlation between the negative magnetoresistance and the resistance exceeding $h/e^2$ (figure~\ref{fig:bilayer2018_wl_trend}), and the evidence supporting a variable-range hopping model, it might be possible that the ferromagnetic order enabled such negative magnetoresistance that would otherwise only occur at much higher resistance.