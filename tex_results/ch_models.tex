Given the spin-momentum locking in the surface states, the electric conduction in topological insulator thin films is typically discussed in literature in terms of the weak anti-localization effect. While the weak anti-localization explains well the positive magnetoresistance that is ubiquitously observed in topological insulator thin films, two discrepancies have been observed at low temperatures. Firstly, while the weak anti-localization effect should increase the conductivity as the temperature is lowered~\cite{bergmann1984}, a reduction of conductivity is observed often in thin films with thicknesses $\lesssim 10~\mathrm{nm}$, with the conductivity varies logarithmically with the temperature~\cite{Chen2011, Liu2011, Roy2013} similarly to that described by the weak localization (the opposite) effect (equation~\ref{eq:wl_T}). This has been attributed to the effect of electron-electron interaction in the diffusive r\'egime in two dimensions~\cite{WL_ee}. Secondly, while the weak anti-localization effect is applicable only for low sheet resistance ($R_\Box \ll h/e^2$, see section~\ref{sec:wl}), the resistance of topological insulator thin films sometimes approaches or exceeds the criterion~\cite{TI_WAL_thickness, ZhangJS2011}.

In chapter~\ref{ch:bilayer2018}, such high resistance ($R_\Box \gtrsim h/e^2$) was observed as the commonality amongst TI-ferromagnet thin film bilayers that show negative magnetoresistance (figure~\ref{fig:bilayer2018_wl_trend}). After the publication of \cite{bilayer2018}, we were brought to attention to ref.~\cite{liao2015}, in which \citeauthor{liao2015} analyzed the temperature dependence of resistance of \SI{2}{nm} (Bi$_x$Sb$_{1-x}$)$_2$Te$_3$ thin films, demonstrating that by reducing the thicknesses of thin films, electric conduction undergoes a crossover from the diffusive r\'egime with electron-electron interaction to the variable-range hopping r\'egime. These suggest that the temperature dependence of resistance should be analyzed for the thin film samples presented so far in this thesis.

Both topological insulator materials presented in this thesis should be considered doped semiconductors. In the case of Bi$_2$Se$_3$, perfect crystalline samples were predicted to have a Fermi level in the bulk band gap~\cite{TI_electronic_structure_zhang}; whereas as-made thin film samples in practice typically exhibit elevated Fermi levels that intersect with the bulk conduction band, primarily due to interstitials and vacancies~\cite{TI_ARPES1, ARPES_thickness, zhangli2013, Zhanybek3, Fisher2010}. In the case of (Bi$_x$Sb$_{1-x}$)$_2$Te$_3$, on the other hand, the difference between the ionization energy of Bi$^{3+}$ and that of Sb$^{3+}$ is utilized to move the Fermi level into the bulk band gap, typically with Bi$^{3+}$ serving as substitutional dopants~\cite{ZhangJS2011, TI_electronic_structure_zhang}.

The electric conduction modes in doped semiconductors discussed in ref~\cite{schklovskii_efros} and section~\ref{sec:vrh} are, therefore, relevant to the thin film samples in this thesis. Here we will test four conduction models against the temperature dependence of resistance of our samples: the diffusive conduction with electron-electron interaction (EE) as demonstrated in refs.~\cite{Chen2011, Liu2011, Roy2013, liao2015}, both Mott type and Efros-Shklovskii type variable-range hopping (VRH), as well a thermal activation model that might describe the crossover between the variable-range hopping and fully diffusive r\'egimes (equation~\ref{eq:activation}, and ref.~\cite{schklovskii_efros}: chapter 4). The temperature dependence of sheet conductivity resulted from these models in two-dimensions are discussed in section~\ref{sec:vrh}, and summarized below in equations~\ref{eq:modes}:%
\begin{equation}\label{eq:modes}
    \sigma_\Box(T) \propto %
    \begin{cases}
        \log(T / T_0)  & \quad \text{(diffusive with EE, \cite{WL_ee})}\\
        e^{-\frac{\varepsilon_1}{k_B T}}   & \quad \text{(thermal activation, equation~\ref{eq:activation})}\\
        e^{-\left(\frac{T}{T_0}\right)^{-\frac{1}{3}}}   & \quad \text{(Mott type VRH, equation~\ref{eq:vrh_mott})}\\
        e^{-\left(\frac{T}{T_0}\right)^{-\frac{1}{2}}} & \quad \text{(Efros-Shklovskii type VRH, equation~\ref{eq:vrh_efros})}~.
    \end{cases}
\end{equation}%

To test how well each of these models describes the temperature dependence of the thin film samples presented in earlier chapters, the equations~\ref{eq:modes} are linearized into the form $y=ax+b$ (table~\ref{tab:models}), %
\begin{table}[ht]
    \centering
    \begin{tabularx}{0.95\columnwidth}[t]{l|l|l|X}
    \caption[Linearized formulae for conduction models]{\label{tab:models}Linearized formulae for conduction models. The shorthands will be used to annotate figures.}\\
		\hline\hline
        Conduction models & $x$ & $y$ & Shorthands\\
        \hline%
        diffusive with electron-electron interaction & $\log(T)$ & $\sigma_\Box$ & EE\\
        thermal activation & $T^{-1}$ & $\log\left(R_\Box\right)$ & TA\\
        Mott type variable-range hopping & $T^{-1/3}$ & $\log\left(R_\Box\right)$ & M-VRH\\
        Efros-Shklovskii type variable-range hopping & $T^{-1/2}$ & $\log\left(R_\Box\right)$ & ES-VRH\\
		\hline\hline
    \end{tabularx}
\end{table}%
%
and the linearity of the data plots according to each of the models will be compared.

\section{Bi$_2$Se$_3$--EuS Bilayers and Bi$_2$Se$_3$ Thin Films}
In this section, the models listed in table~\ref{tab:models} will be tested against the temperature dependence of the resistance of the Bi$_2$Se$_3$--EuS bilayer samples and Bi$_2$Se$_3$ thin films without EuS presented in chapter~\ref{ch:bilayer2014} (see table~\ref{tab:bl2014_samples} for a summary of the samples).

The models are first compared for the four Bi$_2$Se$_3$--EuS bilayer samples that showed negative magnetoresistance below the Curie temperature of EuS (figure~\ref{fig:models_bls}). %
\begin{figure}[ht]%
    \centering%
    \includegraphics[width=0.8\columnwidth]{figs_models/bls}%
    \caption[Conduction model comparison: Bi$_2$Se$_3$--EuS bilayers: BL1--BL3]{\label{fig:models_bls}(Color online) Comparison between conduction models for Bi$_2$Se$_3$--EuS bilayer samples in chapter~\ref{ch:bilayer2014} that showed negative magnetoresistance below the Curie temperature of EuS: (a)~diffusive with electron-electron interaction, (b)~thermal activation, (c)~Mott type variable-range hopping, (e)~Efros-Shklovskii type variable-range hopping.}%
\end{figure}%
%
These samples all have the Bi$_2$Se$_3$ layers around \SI{3}{nm} thick, and showed sheet resistance comparable to the von Klitzing constant at room temperature: $R_\Box(T=300~\mathrm{K}) \sim h/e^2$, and up to two orders of magnitude higher resistance at $T = 2~\mathrm{K}$. Among the four models tested, the diffusive model with electron-electron interaction (figure~\ref{fig:models_bls}a), and the thermal activation model (figure~\ref{fig:models_bls}b) do not display a clear linearity in any significant temperature range. Among the variable-range hopping models (figures~\ref{fig:models_bls}c~\&~d), the Efros-Shklovskii type seems to fit in a wide temperature range $3~\mathrm{K} \lesssim T \lesssim 100~\mathrm{K}$. Below $T \approx 3~\mathrm{K}$, the samples show resistance values that deviate from each other, and from the Efros-Shklovskii model. This is possibly an artefact introduced by inhomogeneity in sample thickness that was discussed in chapter~\ref{ch:bilayer2014}.

In the Bi$_2$Se$_3$--EuS bilayer sample in figure~\ref{fig:TI0_MR}, that has \SI{5}{nm}-thick Bi$_2$Se$_3$ layer, and showed only positive magnetoresistance similarly to a \SI{5}{nm}-thick Bi$_2$Se$_3$ film without EuS, the comparison is less clear (figure~\ref{fig:models_bl0}). %
\begin{figure}[ht]%
    \centering%
    \includegraphics[width=0.8\columnwidth]{figs_models/bl0}%
    \caption[Conduction model comparison: Bi$_2$Se$_3$--EuS bilayer: BL0]{\label{fig:models_bl0}(Color online) Comparison between conduction models for Bi$_2$Se$_3$--EuS bilayer sample in chapter~\ref{ch:bilayer2014} that only showed positive magnetoresistance.}%
\end{figure}%
%
Both the thermal activation model (figure~\ref{fig:models_bl0}b) and the Efros-Shklovskii type variable-range hopping (figure~\ref{fig:models_bl0}d) both seem to describe well the low-temperature part of the data ($T \lesssim 30~\mathrm{K}$ and $T \lesssim 10~\mathrm{K}$, respectively). This might be an indication that this sample resides within the cross-over between the variable-range hopping and the diffusive conduction r\'egimes.

The Bi$_2$Se$_3$ thin films on Al$_2$O$_3$~(0001) without EuS, TI0 and TI3, show considerably lower sheet resistance than that of their bilayer counterparts BL0 and BL3 (figure~\ref{fig:models_tis_rt}). %
\begin{figure}[h!]%
    \centering%
    \includegraphics[width=0.75\columnwidth]{figs_models/tis_rt}%
    \caption[Temperature dependence of sheet resistance of Bi$_2$Se$_3$ thin films: TI0~\&~TI3]{\label{fig:models_tis_rt}(Color online) Temperature dependence of sheet resistance of Bi$_2$Se$_3$ thin films: (a)~TI0: \SI{5}{nm}-thick, counterpart of BL0; (b)~TI3: \SI{3}{nm}-thick, counterpart of BL3.}%
\end{figure}%
%
The resistance of TI0 decreases as the temperature decreases (figure~\ref{fig:models_tis_rt}a), therefore none of the models in table~\ref{tab:models} applies. The resistance of TI3, on the other hand, increases as the temperature increases. The comparison between models for TI3 are shown in figure~\ref{fig:models_ti3}. %
\begin{figure}[h!]%
    \centering%
    \includegraphics[width=0.8\columnwidth]{figs_models/ti3}%
    \caption[Conduction model comparison: Bi$_2$Se$_3$ thin films: TI3]{\label{fig:models_ti3}(Color online) Comparison between conduction models for the \SI{3}{nm} Bi$_2$Se$_3$ thin film, TI3, in chapter~\ref{ch:bilayer2014}.}%
\end{figure}%
%
Again, the Efros-Shklovskii type variable-range hopping seems to account for the temperature dependence at low temperatures ($T \lesssim 50~\mathrm{K}$) in the most satisfactory manner.

\section{(Bi$_x$Sb$_{1-x}$)$_2$Te$_3$--EuS Bilayers}
Since some of the (Bi$_x$Sb$_{1-x}$)$_2$Te$_3$--EuS bilayer samples presented in chapter~\ref{ch:bilayer2018} undergo transitions at $T\approx 30~\mathrm{K}$ and at $T \approx 60~\mathrm{K}$, the analyses in this section will be focused on the temperature range $T < 30~\mathrm{K}$. The models in table~\ref{tab:models} are tested on the three samples S1--S3, that showed insulating behavior at low temperatures (figures~\ref{fig:models_s1}--\ref{fig:models_s3}).%
%
\begin{figure}[h!]%
    \centering%
    \includegraphics[width=0.8\columnwidth]{figs_models/s1}%
    \caption[Conduction model comparison: (Bi$_x$Sb$_{1-x}$)$_2$Te$_3$--EuS bilayer: S1]{\label{fig:models_s1}(Color online) Comparison between conduction models for the (Bi$_x$Sb$_{1-x}$)$_2$Te$_3$--EuS bilayer sample S1 in chapter~\ref{ch:bilayer2018}.}%
\end{figure}%
%
\begin{figure}[h!]%
    \centering%
    \includegraphics[width=0.8\columnwidth]{figs_models/s2}%
    \caption[Conduction model comparison: (Bi$_x$Sb$_{1-x}$)$_2$Te$_3$--EuS bilayer: S2]{\label{fig:models_s2}(Color online) Comparison between conduction models for the (Bi$_x$Sb$_{1-x}$)$_2$Te$_3$--EuS bilayer sample S2 in chapter~\ref{ch:bilayer2018}.}%
\end{figure}%
%
\begin{figure}[h!]%
    \centering%
    \includegraphics[width=0.8\columnwidth]{figs_models/s3}%
    \caption[Conduction model comparison: (Bi$_x$Sb$_{1-x}$)$_2$Te$_3$--EuS bilayer: S3]{\label{fig:models_s3}(Color online) Comparison between conduction models for the (Bi$_x$Sb$_{1-x}$)$_2$Te$_3$--EuS bilayer sample S3 in chapter~\ref{ch:bilayer2018}.}%
\end{figure}%

In all cases, the Efros-Shklovskii type variable-range hopping model provides the best fit to the measured data, while the differences between the Mott type and the Efros-Shklovskii type are marginal.

\section{Discussion}
In chapter~\ref{ch:bilayer2018}, we suggested that the negative magnetoresistance observed in topological insulator-insulating ferromagnet bilayers seem to arise from a conduction r\'egime in which localization effects play dominant roles. Indeed, the analyses in this chapter support such conclusion, and suggest that the electric conduction is best described by the variable-range hopping model. The variable-range hopping mode of conduction is, however, not only observed in bilayer samples where negative magnetoresistance occurs, but also in bilayers where the negative magnetoresistance is absence (figures~\ref{fig:models_bl0}~\&~\ref{fig:models_s3}) as well as in the \SI{3}{nm} Bi$_2$Se$_3$ thin film without EuS (figure~\ref{fig:models_ti3}). This suggests that variable-range hopping is a necessary but not sufficient condition for the negative magnetoresistance to emerge.

Overall, the Efros-Shklovskii type variable-range hopping model seems to fit the data more closely than the Mott type, although the difference is marginal in some cases. This suggests that the electron-electron interaction observed in the diffusive conduction r\'egime~\cite{Chen2011, Liu2011, Roy2013} persists in the variable-range hopping r\'egime.

In both cases of Bi$_2$Se$_3$--EuS bilayers and (Bi$_x$Sb$_{1-x}$)$_2$Te$_3$--EuS bilayers, as the thicknesses of the TI layer increase, the differences between how well the thermal activation model (and the diffusive model, to a lesser extent), and the variable-range hopping models fit the data seem to decrease. This might be a manifestation of the crossover from the variable-range hopping conduction mode to the diffusive mode as discussed in re.~\cite{liao2015}.

It should be noted that the scalings of temperature, $T^{-1}$, $T^{-1/3}$, and $T^{-1/2}$, give more weight to data points at lower temperatures in their respective plots. Thus the observations above regarding the linearity in the data plots were made only considering a relatively small temperature range: from \SI{2}{K} to a few tens of Kelvins. To ascertain the accuracy of the such observations, further measurements on similar samples at lower temperatures would be helpful.