\footnote[2]{A part of this chapter is adapted from ref.~\cite{bilayer2014}~\fullcite{bilayer2014}, with permission of the publisher. Copyright (2013) by the American Physical Society.}%
%
At the conclusion of the experiments presented in this chapter and the publication of ref.~\cite{bilayer2014}, probably the most extensively studied three-dimensional topological insulator (3D TI) had been bismuth(III) selenide (Bi$_2$Se$_3$)~\cite{TI_electronic_structure_zhang, Zhanybek3, TI_other1}, exhibiting crystal structure that consists of atomic quintuple layers (QLs), with three QLs forming a unit cell. As made, uncompensated samples typically have a Fermi level above the Dirac point and intersecting the bulk conduction band~\cite{TI_ARPES1, ARPES_thickness}. In particular, low temperature transport measurements on ungated and uncompensated TI films show positive magnetoresistance (MR) at low magnetic fields and in a wide range of film thicknesses~\cite{ TI_WAL_Hongkong, TI_WAL_thickness, zhangli2012}. This was explained in terms of weak antilocalization (WAL) resulted from spin-momentum locking on the surface state Dirac cone (\textit{q.v.} sections \ref{sec:ti} \& \ref{sec:wl})~\cite{TI_WAL_Hongkong, bergmann1984}. While the inability to account for the bulk bands (presumably because of their low mobility) has challenged this simple assignment, the discovery of weak localization (WL) effects at higher fields~\cite{zhangli2013} and the ability to accurately separate quantum oscillation effects~\cite{Ando_PRL} in high-quality films may provide a first step towards a more comprehensive understanding of transport in these systems.

As initial steps to investigate the effects due to proximity between a 3D TI and an insulating ferromagnet (IF), Bi$_2$Se$_3$ thin films were deposited on top of the high-quality EuS thin films presented in chapter~\ref{ch:EuS}, forming thin film bilayers (BLs). The pulsed laser deposition (PLD) of Bi$_2$Se$_3$ thin films on EuS~(100) and Al$_2$O$_3$~(0001) are described in section~\ref{sec:bilayer2014_char}. The magnetic properties of the resultant thin films are documented in section~\ref{sec:bl2014_mag}. In section~\ref{sec:bl2014_negtive_MR}, magneto-transport measurements are presented, demonstrating a negative magnetoresistance that emerges below the Curie temperature of EuS, in contrast to the ubiquitously observed positive magnetoresistance in topological insulator thin films.

\section{Sample Fabrication}\label{sec:bilayer2014_char}
The fabrication of all the bilayer samples in this chapter follows a scheme illustrated by figure~\ref{fig:bl2014_sketch}.%
\begin{figure}[h]%
    \centering%
    \subfloat{\label{fig:bl2014_sketch}}%
    \subfloat{\label{fig:bl2014_TEM}}%
    \includegraphics[width=0.95\columnwidth]{figs_bilayer2014/schetch_tem}%
    \caption[Schematic and cross-section micrograph of Bi$_2$Se$_3$--EuS thin film bilayers]{(Color online) (a)~Schematic of a bilayer device: an EuS film was deposited on an Al$_2$O$_3$(0001) substrate, followed by 15nm Titanium contacts with gradual height on the edges, finally a Bi$_2$Se$_3$ layer were deposited. (b)~Transmission electron micrograph of a cross-section of a bilayer sample, showing the quintuple layers (QL) of Bi$_2$Se$_3$ and a smooth TI-IF interface.}%
\end{figure} %
%
For the bottom layer, EuS thin films of thicknesses 20--200nm were grown on Al$_2$O$_3$ (0001) using the optimal procedure outlined in chapter~\ref{ch:EuS}. Titanium Ohmic contacts were subsequently deposited \textit{ex situ} on each EuS thin film through a shadow mask with electron-beam evaporation. Electric insulation ($R_\Box > 20~\mathrm{M\Omega}$) was verified for $2~\mathrm{K}<T<300~\mathrm{K}$. Finally, the Bi$_2$Se$_3$ layer was deposited, overlapping with the titanium contacts at their inner corners, forming a van der Pauw configuration (\textit{q.v.} section~\ref{sec:vdp}).

Similarly to the EuS targets in section~\ref{sec:EuSPLD}, PLD targets of Bi$_2$Se$_3$ were prepared by SPS~\cite{Jinfeng2, Subhash1},\footnote{Target preparation was carried out by Jinfeng Zhao and Subhash H. Risbud at the Department of Chemical Engineering and Materials Science, University of California, Davis.}\footnote{PLD recipe for Bi$_2$Se$_3$ was developed together with Li Zhang.} and the target surface was polished with a 800~grit diamond sandpaper before transferring to the vacuum chamber. Unlike EuS, the plasma plumes generated by ablating Bi$_2$Se$_3$ in high vacuum are narrow in comparison to the substrate size ($\sim$5~mm). To diffuse the plasma plumes and to prevent potential oxidation from residual gases, 200~millitorr of argon gas mixed with 2\% hydrogen was introduced after pumping to $10^{-6}$ high vacuum. For final target surface treatment and to deposit the Bi$_2$Se$_3$ thin films, the target was ablated by a 25~ns 248~nm KrF excimer pulsed laser beam at 5~Hz repetition rate and 0.54~$\mathrm{J\cdot{}cm^{-2}}$ fluence, while spinning at 18 rpm. EuS thin films on Al$_2$O$_3$(0001) substrates were heated to $\SI{150}{\degreeCelsius}$, $\SI{5}{cm}$ away from the target. The growth rate was estimated to be 0.15~\AA{} per pulse.

To compensate for selenium deficiencies that are typical for as-grown Bi$_2$Se$_3$ thin films \cite{Zhanybek3, zhangli2012, zhangli2013, TI_ARPES1, ARPES_thickness}, a selenium capping layer was deposited \textit{in situ} immediately following the Bi$_2$Se$_3$ layer by ablating a selenium sputtering target (Kurt J. Lesker, 99.999\% pure) for 300 pulses. The sample was annealed with the capping layer at $\SI{150}{\degreeCelsius}$ for roughly 15 minutes before the heater was switched off. When the sample was cooled to below $\SI{80}{\degreeCelsius}$, an additional selenium capping layer was applied before venting, to prevent exposure to air.

The cross-section transmission electron micrograph (figure~\ref{fig:bl2014_TEM}) indicates smooth interface and excellent layering of the Bi$_2$Se$_3$. While the brightness is heightened at the interface in figure~\ref{fig:bl2014_TEM}, individual atoms are still discernible when the micrograph is presented in digital format. Such heightened brightness is likely artefact resulted from the preparation of the cross section, which involves mechanical polishing followed by argon ion milling. Very different mechanical properties of EuS (cubic lattice with ionic bonds) and Bi$_2$Se$_3$ (quintuple layers weakly bound by van der Waals force) likely result in different response to mechanical polishing, which might be further compounded by different etching rate during ion milling. Similar heightened brightness is observed at the interface between EuS and Al$_2$O$_3$ to a lesser extent (\textit{cf.} figure~\ref{fig:EuS_TEM}).

To control for potential differences between Bi$_2$Se$_3$ thin films by PLD and previously reported thin films, which were fabricated predominantly by molecular beam epitaxy (MBE), a bare Al$_2$O$_3$(0001) substrate with Ohmic contacts was placed next to the EuS thin film in each PLD session. Therefore, for each bilayer sample, a Bi$_2$Se$_3$--only sample was fabricated with the same deposition condition and the same geometry. To examine the relevance of surface conduction, samples with different thicknesses for the Bi$_2$Se$_3$ layer were fabricated and compared. The samples of interest are summarized in table~\ref{tab:bl2014_samples}.%
\begin{table}[ht]
    \centering
    \begin{tabularx}{0.6\columnwidth}[t]{l|l|X}
    \caption[Summary of Bi$_2$Se$_3$ thin films and Bi$_2$Se$_3$--EuS bilayer samples]{\label{tab:bl2014_samples}Summary of Bi$_2$Se$_3$ thin films and Bi$_2$Se$_3$--EuS bilayer samples discussed in this chapter. All samples are on Al$_2$O$_3$ (0001) substrates. The Bi$_2$Se$_3$ layers on the TI and BL samples with the same numerical suffix were deposited in the same PLD session.}\\
		\hline\hline
        Samples & Configurations & TI Thicknesses\\
        \hline
        TI0 & Bi$_2$Se$_3$ only & 5~nm\\
        TI3 & Bi$_2$Se$_3$ only & 3~nm\\
        BL0 & Bi$_2$Se$_3$ on EuS & 5~nm\\
        BL1 & Bi$_2$Se$_3$ on EuS & 3~nm\\
        BL2 & Bi$_2$Se$_3$ on EuS & 3~nm\\
        BL3 & Bi$_2$Se$_3$ on EuS & 3~nm\\
        BL4 & Bi$_2$Se$_3$ on EuS & 3~nm\\
		\hline\hline
    \end{tabularx}
\end{table} %
The Bi$_2$Se$_3$ layers on the bilayer (BL) and Bi$_2$Se$_3$--only (TI) samples with the same numerical suffix were deposited in the same batch.

\FloatBarrier%
\section{Magnetic Properties}\label{sec:bl2014_mag}
A crucial condition for the ferromagnet to affect the topological insulator is to break the time-reversal symmetry (TRS) in the relevant dimensions, \textit{e.g.,} by a magnetization perpendicular to the interface~\cite{TI_Col, QAH_TI_Yu, MnSe}. The perpendicular magnetization in EuS thin films was confirmed by SQUID magnetometry, and discussed in chapter~\ref{ch:EuS}. Figure~\ref{fig:bl2014_squid} shows SQUID measurements on an example of EuS thin films used to fabricate bilayers.\footnote{Error bars in this chapter represent 90\% confidence intervals.}%
%
\begin{figure}[ht]%
    \centering%
    \subfloat{\label{fig:bl2014_MvH}%
        \includegraphics[width=0.45\columnwidth]{figs_bilayer2014/MvH_EuS04}%
    }%
    \subfloat{\label{fig:bl2014_MvT}
        \includegraphics[width=0.45\columnwidth]{figs_bilayer2014/MvT_EuS04}%
    }%
    \caption[SQUID magnetometry of an EuS thin film used for bilayers]{\label{fig:bl2014_squid}(Color online) SQUID magnetometry of an EuS thin film used for bilayers. (a)~The perpendicular component of the magnetization as functions of the perpendicular magnetic fields. (b)~The ferromagnetic transition observed in a small perpendicular field. A fitting to the Curie-Weiss law in the paramagnetic r\'egime yields a Curie temperature of $T_C \approx \SI{15.7}{K}$.}%
\end{figure} %
%
Ferromagnetic transition was observed around a Curie temperature $T_C \approx \SI{15.7}{K}$ (\textit{q.v.} appendix~\ref{ap:curie}).

Since such TRS breaking can be directly probed by measuring the polar Kerr effect~\cite{Xia2006}, the bilayer sample BL3 was measured with Sagnac interferometry in two configurations. Measured at a fixed position where the EuS is covered by the Bi$_2$Se$_3$ layer,\footnote{Polar Kerr effect at fixed position was measured and analyzed by Elisabeth Schemm.} the Kerr angles exhibit hysteresis at $T=\SI{310}{mK}$ in a sweeping magnetic field perpendicular to the film (figure~\ref{fig:bl2014_SchemmH}),%
%
\begin{figure}[ht]%
\centering%
\subfloat{\label{fig:bl2014_SchemmT}}%
\subfloat{\label{fig:bl2014_SchemmH}}%
\includegraphics[width=\columnwidth]{figs_bilayer2014/schemm}
\caption[Polar Kerr effect in a Bi$_2$Se$_3$--EuS bilayer]{\label{fig:bl2014_schemm}(Color online) Polar Kerr effect measured by a Sagnac interferometer. Kerr angles were measured (a) at $T=\SI{310}{mK}$ as functions of the sweeping magnetic field perpendicular to the film, where the arrows denote the directions of field change; (b) in zero magnetic field as a function of increasing temperature, after exposure to a 5~kOe field at \SI{310}{mK}.}%
\end{figure} %
%
and disappear when the temperature exceeds the $T_C$ of EuS (figure~\ref{fig:bl2014_SchemmH}). This confirms that the TRS breaking observed is originated from the ferromagnetic state in the EuS. When measured at $T=\SI{10}{K}$, and scanning on a straight line across the boundary between a bare EuS region and one covered by the Bi$_2$Se$_3$ layer,\footnote{Scanning Sagnac interferometry was carried out and analyzed by Alexander Fried.} the Kerr angles are reduced by the presence of the Bi$_2$Se$_3$, while remaining finite in size (figure~\ref{fig:bl2014_Fried}).%
%
\begin{figure}[ht]%
\centering%
\includegraphics[width=0.95\columnwidth]{figs_bilayer2014/fried}
\caption[Spatial distribution of the Kerr angles in a Bi$_2$Se$_3$--EuS bilayer]{\label{fig:bl2014_Fried}(Color online) Polar Kerr effect measured by a scanning Sagnac interferometers at $T=\SI{10}{K}$. Kerr angles were measured as functions of applied magnetic fields and in-plane distance across the edge of the Bi$_2$Se$_3$ layer, where $x<0$ region is bare EuS and $x>0$ is Bi$_2$Se$_3$ on top of EuS.}%
\end{figure} %
The Kerr angles in both regions indicate a uniform spatial distribution of magnetization, without obvious domain boundaries.

\FloatBarrier%
\section{Emergent Negative Magnetoresistance}\label{sec:bl2014_negtive_MR}
Hitherto in this thesis, we have ascertained the necessary conditions stated in section~\ref{sec:ti} for studying the proximity effects between a 3D TI and a ferromagnet. The perpendicular component of magnetization was confirmed by SQUID magnetometry (figure~\ref{fig:EuS_MvH_z}), and the consequent TRS breaking by the Sagnac interferometry (figure~\ref{fig:bl2014_schemm}). The uniformity of the magnetization was verified by scanning Sagnac interferometry (figure~\ref{fig:bl2014_Fried}). The interface between the TI and the ferromagnet was observed in the cross-section electron micrograph (figure~\ref{fig:bl2014_TEM}). The absence of electric conduction in the ferromagnet was confirmed by direct measurements (\textit{q.v.} chapter~\ref{ch:EuS}).

The manifestation of such proximity effects was observed in the magneto-transport properties of the bilayer samples. The magnetoresistance of four samples of two representative thicknesses are shown in figure~\ref{fig:bl2014_MR_thickness}.%
%
\begin{figure}[ht]%
\centering%
\subfloat{\label{fig:TI0_MR}}%
\subfloat{\label{fig:TI3_MR}}%
\subfloat{\label{fig:BL0_MR}}%
\subfloat{\label{fig:BL3_MR}}%
\includegraphics[width=0.95\columnwidth]{figs_bilayer2014/MR_thickness}%
\caption[Magnetoresistance of Bi$_2$Se$_3$ thin films and Bi$_2$Se$_3$--EuS bilayers]{\label{fig:bl2014_MR_thickness}(Color online) Magnetoresistance and its temperature dependence of PLD--grown Bi$_2$Se$_3$ thin films (TI) and Bi$_2$Se$_3$--EuS bilayer (BL) samples. (a, b)~The Bi$_2$Se$_3$--only samples have positive magnetoresistance regardless of thicknesses. (c)~The BL samples with TI-layer thicknesses $t\gtrsim4\mathrm{QL}$ behave similarly to the TI--only films. (d)~With TI--layer thicknesses $t\lesssim4\mathrm{QL}$, a distinctive negative magnetoresistance is observed at low-fields in BLs below the Curie temperature of EuS. The thickness limit coincides with occurrence of coupling between the top and the bottom surfaces of a TI thin film.}%
\end{figure} %
%
As ubiquitously seen in TI thin films, the PLD--grown Bi$_2$Se$_3$--only samples show positive magnetoresistance at low fields regardless of thicknesses, which broadens monotonically with increasing temperature (figures~\ref{fig:TI0_MR}~\& \ref{fig:TI3_MR}), consistent with the weak antilocalization effect (WAL). Fittings to the standard Hikami-Larkin-Nagaoka (HLN) formula describing the WAL magnetoconductance (\textit{q.v.} section \ref{sec:wl}) yield dephasing lengths ($l_\phi$) comparable to MBE--grown samples with the same thicknesses~\cite{TI_WAL_thickness, zhangli2013}. With thicknesses of the Bi$_2$Se$_3$ layer greater than $\sim4\mathrm{QL}$, the TI--IF bilayers have similar low-field magnetoresistance features to their TI--only counterparts (figure~\ref{fig:BL0_MR}). By contrast, with TI--layer thicknesses $t\lesssim4\mathrm{QL}$, the bilayers show distinctive negative low-field magnetoresistance at low temperatures (figure~\ref{fig:BL3_MR}). Such negative magnetoresistance features are clearly distinguishable well below the Curie temperature ($T_C=15.7$K) of EuS (figures~\ref{fig:bl2014_2K8K} \& \ref{fig:bl2014_8K16K}), %
%
%
\begin{figure}[ht]%
\centering%
\subfloat{\label{fig:bl2014_16K30K}}%
\subfloat{\label{fig:bl2014_8K16K}}%
\subfloat{\label{fig:bl2014_2K8K}}%
\subfloat{\label{fig:bl2014_He3}}%
\includegraphics[width=0.95\columnwidth]{figs_bilayer2014/MR_temperature}%
\caption[Temperature dependence of the magnetoresistance of a Bi$_2$Se$_3$--EuS bilayer]{\label{fig:bl2014_MR_temperature}(Color online) Low-field magnetoresistance of a TI--IF bilayer device (BL3, $t\approx3\mathrm{QL}$) in magnetic fields perpendicular to the film: (a)~Above $T_C$, a WAL--like positive magnetoresistance sharpens with decreasing temperature; (b)~Just below $T_C$, magnetoresistance broadens with decreasing temperature; (c)~Well below $T_C$, a sharp negative magnetoresistance emerges near zero field. The emergence of negative magnetoresistance below $T_C$ of the EuS indicates a TI--IF proximity effect. (d)~Magnetoresistance of BL3 below $T=2$K, with solid lines as illustrative guides. Resistance at $T\leq0.8\mathrm{K}$ was measured with two-terminal configurations.}
\end{figure}%
%
whereas positive magnetoresistance resembling WAL appears at higher fields. Below and close to $T_C$ (figure~\ref{fig:bl2014_8K16K}), the negative magnetoresistance can no longer be directly observed. However, its remnant contribution reverses the thermal broadening of the overall positive magnetoresistance. This suggests that the mechanism producing such negative magnetoresistance is reduced rapidly close to the ferromagnetic transition. Above $T_C$ (figure~\ref{fig:bl2014_16K30K}), the positive magnetoresistance is eventually broadened when increasing the temperature, similar to common WAL features in TI--only thin films. In figures~\ref{fig:bl2014_MR_thickness}~\&~\ref{fig:bl2014_MR_temperature}, we presented the negative magnetoresistance in the same sample, whereas four bilayer samples (labeled as BL1--4, table~\ref{tab:bl2014_samples}) with $t\lesssim4\mathrm{QL}$ from different growth batches all demonstrated such proximity effect in a consistent manner. The thickness criterion ($t\lesssim4\mathrm{QL}$) coincides with the thickness below which the two surfaces of a Bi$_2$Se$_3$ film are observed to be coupled~\cite{ARPES_thickness}, suggesting that the mechanism for such negative magnetoresistance is originated from surface effects.

To quantify the changes in the low-field magnetoresistance with respect to temperatures, a quadratic model was fitted to the magnetoresistance data near zero field:\footnote{At suggestion by Xiaoliang Qi.}%
\begin{equation}\label{eq:bl2014_quadratic}%
    \frac{\Delta R(H)}{R(0)} = a(T)\cdot H^2~.%
\end{equation}%
Since at different temperatures the magnetoresistance has different low-field and high-field features, the ranges of the data points to fit the model was chosen dynamically. For each temperature, the largest field interval, $H \in [-H_0, +H_0]$, up to $H_0 = 0.5 \mathrm{T}/\mu_0$, was chosen such that the mean-square of the residuals ($\overline{r^2}$) is still comparable to that of the statistical errors of the data points ($\overline{\epsilon^2}$):
\begin{equation}
    \overline{r^2} < 2\overline{\epsilon^2}~.%
\end{equation}%
The resultant fitting intervals are narrower for the sharp negative magnetoresistance at low temperatures (figure~\ref{fig:bl2014_quadratic}a),%
%
\begin{figure}[ht]%
    \centering%
    \includegraphics[width=0.95\columnwidth]{figs_bilayer2014/quadratic}%
    \caption[Quadratic fittings to the magnetoresistance of a Bi$_2$Se$_3$--EuS bilayer]{\label{fig:bl2014_quadratic}(Color online) The magnetoresistance of Bi$_2$Se$_3$--EuS bilayers is fitted at low fields to a quadratic model $\Delta R(H) / R(0) = a(T)\cdot H^2$. The fitting ranges were dynamically chosen such that the mean-square of the residuals are comparable to that of the measurement errors. The data points (crosses) and fittings (solid curve) are shown for (a)~$T=2~\mathrm{K}$, and (b)~$T=30~\mathrm{K}$. The fitting parameter, $a$, is plotted as a function of the temperature for (c)~$2~\mathrm{K} \leq T \leq 30~\mathrm{K}$, and (d)~$10~\mathrm{K} \leq T \leq 30~\mathrm{K}$.}%
\end{figure} %
%
and wider for the broad positive magnetoresistance at higher temperatures (figure~\ref{fig:bl2014_quadratic}b). The fitting parameters, $a$, are shown as a function of the temperatures (figure~\ref{fig:bl2014_quadratic}c). Consistent with the previous qualitative observation in figures~\ref{fig:bl2014_16K30K}~\&~\ref{fig:bl2014_8K16K}, a maximum in $a(T)$ is found near the Curie temperature of the EuS, $T_C = \SI{15.7}{K}$ (figure~\ref{fig:bl2014_quadratic}d). The agreement between $T_C$ and the temperatures below which the negative magnetoresistance starts to dominate strongly indicates a proximity effect between the IF and the TI layers.

Below 2K (figure~\ref{fig:bl2014_He3}), we observed a continuous sharpening of the low-field negative magnetoresistance when lowering the temperature, as expected from diminishing thermal broadenings. Unexpectedly, the magnitude of negative magnetoresistance was reduced when lowering the temperature below 1K. This can be explained by the inhomogeneity observed in the Bi$_2$Se$_3$ layer grown on top of EuS. While the Bi$_2$Se$_3$ thin films grown on bare substrates were verified by AFM and XRD to be adequately uniform in thickness, the TEM images taken at different locations on the cross-section of the BL3 sample show large variations ($\pm2\mathrm{QL}$) in thickness of the Bi$_2$Se$_3$ layer, with an estimated mean value consistent with the thickness of the Bi$_2$Se$_3$-only film grown in the same PLD session. Near the thicknesses of interest ($t\lesssim4\mathrm{QL}$), both the resistance of Bi$_2$Se$_3$ films and its temperature dependence are known to change sharply with thickness at low temperatures~\cite{TI_WAL_thickness}. Thus special difficulty is introduced when measuring the sheet resistance with a van der Pauw method, where the sample thickness is assumed to be uniform (\textit{q.v.} section~\ref{sec:vdp})~\cite{VdP1958, VdP_contact_size}. With such inhomogeneous geometry, electric conduction is limited by the thinner and therefore more resistive parts of the sample. Indeed, the sheet resistance of the BL samples are one order of magnitude higher than that of the Bi$_2$Se$_3$-only samples with similar thicknesses (figure~\ref{fig:bl2014_RvT}).%
%
\begin{figure}[ht]%
    \centering%
    \includegraphics[width=0.65\columnwidth]{figs_bilayer2014/rvt}%
    \caption[Temperature dependence of the sheet resistance of Bi$_2$Se$_3$--EuS bilayers]{\label{fig:bl2014_RvT}(Color online) Temperature dependence of the sheet resistance of the four TI--IF bilayer samples that showed negative magnetoresistance (BL1--4) and a PLD--grown TI--only film (TI3) measured with the van der Pauw method.}%
    \end{figure} %
%
When lowering the temperature, the change in the zero-field resistance, $R(0)$, is dominated by the sharp temperature dependence of the thinner parts of the sample. The low-field features in magnetoresistance, however, are mainly contributed by the thicker parts, where the dephasing length is longer~\cite{TI_WAL_thickness}. Therefore, after dividing by $R(0)$ as the normalization factor, $\Delta{}R(H)/R(0)$ appears with a reduced magnitude. On the other hand, measuring the Hall effect, which is insensitive to film inhomogeneity~\cite{Landauer_Porous_Media}, the sheet carrier density is calculated showing similar values (\(n_{2D}\approx2\times10^{13}\mathrm{cm^{-2}}\)) for both TI--only films and TI--IF bilayers and are very close to reported values for MBE films~\cite{TI_WAL_thickness}.

Recent theories have suggested that the weak localization effect (WL) may occur in a TI thin film instead of WAL as result of gap-opening at the Dirac point~\cite{WL_WAL_competition, WL_Glazman, WL_bulk_Lu}. Besides an orbital effect such as WL, the presence of a magnetic material brings possibilities of magnetoresistance originated from scatterings at localized spins~\cite{KondoMR}. The magnitude of any spin-introduced magnetoresistance, however, should be greater above $T_C$ and should diminish rapidly below $T_C$ as the spins being aligned during the ferromagnetic transition~\cite{SpinMagnetic}. This is contrary to what we observed. Moreover, we show in figure~\ref{fig:bl2014_angular}%
%
\begin{figure}[ht]%
    \centering%
    \includegraphics[width=1.0\columnwidth]{figs_bilayer2014/angles}%
    \caption[Angular dependence of the magnetoresistance of a Bi$_2$Se$_3$--EuS bilayer]{\label{fig:bl2014_angular}(Color online) Angular dependence of the magnetoresistance of sample BL4, measured at $T = \SI{4}{K}$. (a, b)~As the sample is rotated from being perpendicular ($\theta = 0^\circ$) to being near parallel ($\theta = 89^\circ$) to the applied magnetic fields, both the low-field and the high-field features are broadened. (c, d)~The magnetoresistance coincide when plotted as functions of the effective perpendicular fields, $H_{eff} = H\cos(\theta^*)$. The individual low-field data points in (a) and (c) are shown in (b) and (d), respectively.}%
\end{figure} %
%
the broadening of the negative magnetoresistance as the sample is rotated from being perpendicular to being near parallel to the applied magnetic fields, favoring an orbital origin over a spin origin. While magnetoresistance remains finite in near-parallel fields, it is likely due to the uneven thickness in the Bi$_2$Se$_3$ layer, and consequently a locally slanted top surface, on which local electron paths may have a finite angle to the magnetic field that is parallel to the sample plane. In fact, when the magnetic field is scaled with an effective angle $\theta^*$ between the magnetic field and the normal of local electron transport, both the negative magnetoresistance at low fields and the positive magnetoresistance at higher fields are found to coincide well with perpendicular-field magnetoresistance (figure~\ref{fig:bl2014_angular}c--d). The differences between the effective angles ($\theta^*$) and the nominal angles read from the instrument ($\theta$) are small for most angles ($|\theta-\theta^*|\leq{}5^\circ$ for $\theta\leq{}60^\circ$), and lager towards parallel field ($\theta^*\approx{}72^\circ$ for $\theta=89^\circ$). This is consistent with expectations from a film with a smooth bottom surface and an uneven top surface.

While the orbital origin indicated by the angular dependence of magnetoresistance is consistent with the theoretical prediction of WL as consequence of gap-opening~\cite{WL_WAL_competition, WL_Glazman, WL_bulk_Lu, QAH_TI_Yu}, such phenomenon is expected to occur only when the Fermi level is sufficiently near the gap~\cite{WL_WAL_competition, WL_Glazman, WL_bulk_Lu}. However, as common in ungated and uncompensated Bi$_2$Se$_3$, our carrier densities suggest a Fermi level intersecting the conduction band and therefore far away from the Dirac point. Recent calculations suggest that surface charges on the IF layer may result in band-bending at the interface~\cite{MnSe}, hence provide a possible mechanism to move the gap towards the Fermi level. However, whether it is the case depends on the details of atomic arrangement at the interface~\cite{xiaoliang}, which cannot be determined with available data. On the other hand, low levels of sulfur doping are known to modify the band structure of Bi$_2$Se$_3$ while preserving its structural phase~\cite{Bi2Se3S, BiSeS}. Such effect may as well be present at the Bi$_2$Se$_3$-EuS interface. Attempts to perform ARPES measurements to determine the band gap, location of the Dirac point and the Fermi level produced inconclusive results, primarily due to the inhomogeneity in films thickness.

In parallel to the work presented in this chapter, a study contemporary to ref.~\cite{bilayer2014} on EuS / Bi$_2$Se$_3$ bilayers showed weak tendency towards anomalous Hall effect, which was argued to indicate emergence of a ferromagnetic phase in TI~\cite{Moodera2013}. Samples in that study consisted of a bottom Bi$_2$Se$_3$ film and a top EuS film, making it impossible to determine whether the EuS is truly insulating, and consequently whether the observed effect was from Bi$_2$Se$_3$ or due to conduction in the EuS layer. To contrast, the approach in this chapter allows the EuS and Bi$_2$Se$_3$ thin films to be fabricated and characterized separately. Since high-quality, well characterized EuS films were obtained prior to the deposition of Bi$_2$Se$_3$, the electric insulation in EuS was ascertained. Hence the the unusual negative magnetoresistance at low fields below the Curie temperature of the ferromagnet can only come from the Bi$_2$Se$_3$ layer, there is a strong indication of a proximity effect between a topological insulator and an insulating ferromagnet.