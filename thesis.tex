\documentclass{report}

\usepackage[online]{suthesis-2e}

% For Bibliography
\usepackage[american]{babel}
\usepackage{csquotes}
\usepackage[style=phys, citestyle=phys,%
    articletitle=false, biblabel=brackets, chaptertitle=false, pageranges=false, %APS style
    backend=bibtex8, bibencoding=ascii, doi=false, url=false, eprint=false]{biblatex}
\addbibresource{bib_ours.bib}
\addbibresource{bib_general.bib}
\addbibresource{bib_EuS.bib}
\addbibresource{bib_bilayer.bib}


\usepackage{bm}        % for math
\usepackage{amssymb}   % for math

\usepackage{graphicx}  % needed for figures
\usepackage{siunitx}    % for degree Celcius etc

\usepackage[caption=false]{subfig}  % for sub-references
\usepackage{caption} %

% for hyperlinks
\usepackage{xcolor}
\definecolor{linkAccentColor}{HTML}{0696bb} % Bauhaus Itten cyan
\usepackage[colorlinks=false, breaklinks=true, bookmarks=true, linktocpage=true]{hyperref}
\hypersetup{%
  colorlinks=false,% hyperlinks will be coloured
  linkcolor=black,% hyperlink text will be {}
  linkbordercolor=linkAccentColor,% hyperlink border will be {}
  urlbordercolor=linkAccentColor,% hyperlink border will be {}
  citebordercolor=linkAccentColor,% hyperlink border will be {}
  pdfborderstyle={/S/U/W 1}% Overrides border style set with colorlinks=true
                          % Hyperlink border style will be underline of width 1pt
}


\usepackage[perpage, symbol*]{footmisc}	% for footnote symbols

\usepackage{tabularx}	% for tables
\setlength{\extrarowheight}{2pt}	% row padding

\usepackage{CJK}


\hyphenation{QAHE TI}

\dept{Physics}


\begin{document}
\title{Proximity Effects between Topological Insulators and Insulating Ferromagnets}
\author{Qi Yang}
\principaladviser{Aharon Kapitulnik}
\firstreader{Malcolm R. Beasley}
\secondreader{Ian R. Fisher}
\thirdreader{Xiaoliang Qi}

\beforepreface
\chapter*{}
This dissertation is dedicated to grandmas and grandpas, who showed me with their words and actions the responsibilities of a man and of a scholar; and to Lord Rees of Ludlow, Ms. Elisabeth Seaman and Prof. Jordan B. Peterson, who reminded me of such responsibilities.\\

\begin{CJK*}{UTF8}{gkai}
谨以此论文献给爷爷、奶奶、外公、外婆。他们言传身教告诉了我做人和学者理应承担的责任。同时献给拉德罗的里斯男爵、伊丽萨白$\cdot$希曼女士以及乔丹$\cdot\mathrm{B}\cdot$彼得森教授。他们的言行提醒了我这些责任。
\end{CJK*}

\prefacesection{Abstract}
    Abstract 


\prefacesection{Acknowledgments}
    I would like to thank...
\afterpreface

\captionsetup{width=0.95\columnwidth}%
\chapter{Introduction}
    ...

\chapter{Theoretical Background}
    \section{Topological Insulators and Ferromagnetism}
    \label{sec:TI-FM}
    \section{Weak (Anti-)Localization Effects}
    
\chapter{Experimental Methods}
    \section{Thin Film Growth: Pulsed Laser Deposition}
Pulsed laser deposition (PLD) is a thin film growth method that utilizes high energy laser photons. A pulsed laser beam is focused and transmitted 
\begin{figure}[ht]%
    \includegraphics[width=0.618\columnwidth]{figs_methods/PLD_schematic.png}%
    \caption[A simple schematic of PLD]{\label{fig:PLD_schematic}Schematic of a simple configuration of PLD, licensed from Wikimedia Commons under Creative Commons CC0 1.0 Universal Public Domain Dedication.}%
\end{figure}%
 
    
    \section{Superconducting Quantum Interference Device (SQUID) Magnetometry}
    \section{AC Magnetic Susceptibility by Mutual Inductance}
    \section{Electron Transport: Van der Pauw Method}
    \section{Measurement Automation: Python and Parallel Computing}

\chapter{Europium(II) Sulfide: An Insulating Ferromagnet}
\label{ch:EuS}\footnote[2]{A part of this chapter is adapted from ref.~\cite{EuS_PLD}~\fullcite{EuS_PLD}, with the permission of AIP Publishing.}%
As discussed in section~\ref{sec:ti}, two crucial ingredients to study time-reversal symmetry breaking in a topological insulator are a spontaneous magnetic order, and good electric insulation. In addition to the present purpose, a variety of applications, including spintronics, and $\pi$-Josephson junctions for quantum qubits~\cite{pi_qubit, pi_junction, Wolf2001, EuS_spin_filter, EuS_app1, EuS_spin_filter2}, also require fabrication of high-quality thin films with both robust magnetic properties and highly insulating behavior. Materials with both spontaneous magnetization and insulating behavior are often interchangeably referred to as ``magnetic insulators'' or ``magnetic semiconductors''. Among these materials, many are ferrimagnets. Such examples include magnetite (Fe$_3$O$_4$, Curie temperature $T_C\approx850~\mathrm{K}$)~\cite{Neel1948}, and yittrium-iron garnet (``YIG'', Y$_3$Fe$_2$(FeO$_4$)$_3$, $T_C\approx550~\mathrm{K}$)~\cite{YIG}. In such case, the net spontaneous magnetization is formed by opposite but unequal magnetic lattices, therefore the surface of these material might consist of alternating opposite magnetic domains, which would complicate the interface to a topological insulator. Insulating ferromagnets, on the other hand, only have a small number of candidate materials available.

Two of the first known insulating ferromagnets are the europium(II) chalcogenides EuO and EuS. Divalent compounds of europium with elements of the sixth group (O, S, Se, Te) exhibit a rock-salt (NaCl) type crystal structure with ordered magnetic states at low temperatures. As the lattice parameter increases from EuO to EuTe, a ferromagnetically ordered state of moments localized on Eu$^{2+}$ ions appear in EuO ($T_C\approx69~\mathrm{K}$) and in EuS ($T_C\approx16.5~\mathrm{K}$)~\cite{EuO_TC, EuS_Shafer, EuS_specific_heat}, while EuSe and EuTe show collinear antiferromagnetic ordering with N\'eel temperatures $T_N\approx4.2~\mathrm{K}$, $T_N\approx9.8~\mathrm{K}$ respectively~\cite{EuSe_AF, EuTe_AF}. In these chalcogenide compounds, the localized isotropic $L=0$ ground state ($^8S_{7/2}$) of Eu$^{2+}$ ions and their simple face centered cubic (FCC) magnetic lattice make them prime candidates to test the Heisenberg model of local-moment magnetism and theories of critical phenomena~\cite{divalent_Eu, EuX_indirect_exchange, EuS_neighbor_exchange, EuS_critical, EuS_neutron, EuS_spin_wave}.

Between the two ferromagnets, EuS is favored over EuO for our purposes. Firstly, the properties of EuS has been more thoroughly studied experimentally due to better sample qualities achieved in the past~\cite{EuS_band_th2}. Furthermore, the relatively low Curie temperature of EuS would allow switching on and off its ferromagnetism at low temperatures, when changes in quantum phenomena are more likely observable due to reduced thermal motions. Finally, considering the material systems to interface with in later chapters, namely Bi$_2$Se$_3$ and (Bi$_x$Sb$_{1-x}$)$_2$Te$_3$, and potential unintended anion substitution at the interface: Bi$_2$Te$_{1.6}$S$_{1.4}$ was shown to be a topological insulator itself~\cite{BiTeS}, therefore should introduce minimum complication; whereas Bi$_2$Se$_2$S is a fully gapped trivial insulator~\cite{BiSeS}, therefore at least should not introduce exotic transport properties on its own. Therefore, EuS will be the focus of the rest of this chapter.

\section{Reported Properties of Crystals and Thin Films}\label{sec:EuS_previous}
EuS has a crystal parameter $a_0=5.967~\mathrm\AA$~\cite{EuS_Shafer}, and exhibits an indirect gap of 1.65~eV between the valence band maximum at the $\Gamma$ point (Brillouin zone center) and the conduction band minimum at the X point ([100] Brillouin zone boundary)~\cite{EuX_absorption, EuS_band_th1, EuS_band_th2}. High quality stoichiometric single crystals of EuS exhibit Curie temperatures around $T_C\approx16.5~\mathrm{K}$, and good electric insulation with resistivity values $\rho\approx10~\Omega\cdot\mathrm{cm}$ at room temperature and $\rho\approx10^4~\Omega\cdot\mathrm{cm}$ at $T=135~\mathrm{K}$~\cite{EuS_Shafer}. However, when excessive electrons are present, either by intentional doping or due to difficulties in material fabrication, such as sulfur deficiency and crystal defects, the resistivity may be drastically reduced by n-type conduction to as low as $\rho\approx10^{-2}~\Omega\cdot\mathrm{cm}$, both at room temperature and at low temperatures~\cite{EuS_LaDoped, EuS_ntype}. Such reduction in resistivity was found to be universally accompanied by increased Curie temperatures (up to $T_C\approx30~\mathrm{K}$) due to interactions between charge carriers and the Eu$^{2+}$ ions~\cite{EuS_TC_doping, EuS_ntype, EuX_doped_transport, EuX_RKKY}. On the other hand, cases of p-type conduction in EuS has not been reported, presumably due to the readily available Eu$^{3+}$ ionic state, cancelling the effect of any acceptor doping~\cite{EuX_doped_transport}.

In one and two dimensions, N. D. Mermin and H. Wagner predicted that neither ferromagnetism nor antiferromagnetism can exist at any non-zero temperature within an isotropic Heisenberg model~\cite{Mermin1966}. From 3D crystals to thin films, namely crossing over from three to two dimensions, one may therefore expect the Curie temperature of an isotropic Heisenberg ferromagnet to decrease as the thickness decreases. Indeed, such effect of dimensionality reduction has been investigated theoretically and observed experimentally~\cite{thickness_Tc_theory, thickness_Tc_exp}. Particularly for EuS, lower-than-bulk Curie temperatures were observed in thin films at the thicknesses $t < 6~\mathrm{nm}$~\cite{EuS_MBE_Muller}. However, above such thickness threshold, thin films fabricated by a variety of methods, such as electron beam evaporation, MBE, and PLD, often exhibited $T_C$ higher than single crystal values, which indicate significant n-type carrier doping~\cite{EuS_MBE_Muller, EuS_thin_film_Keller, EuS_PLD1, EuS_PLD2}. Additionally, these reported growth methods often resulted in samples with multiple crystal orientations, which might give rise to fractured magnetic domains given the considerable magnetocrystalline anisotropy of EuS~\cite{EuS_anisotropy}, that might in turn complicate further experiments.

The PLD procedure introduced in the next section has yield EuS thin films with significantly improved qualities. In particular, characterization results will be presented, indicating excellent electric insulation, a significant and uniform out-of-plane component of the magnetization, a single lattice orientation, and a near-ideal surface topography.

\section{Improved Pulsed Laser Deposition of Thin Films}\label{sec:EuSPLD}
For PLD targets, solid disks (approximately $19~\mathrm{mm}$ in diameter and $3~\mathrm{mm}$ thick) were prepared from high-purity (99.95\%) EuS powder by a fast consolidation technique popularly referred to as spark plasma sintering (SPS).\footnote{Target preparation was carried out by Jinfeng Zhao and Subhash H. Risbud at the Department of Chemical Engineering and Materials Science, University of California, Davis.} This technique uses an electric discharge to activate the surface of the powder particles prior to rapid resistance heating, aiming at achieving complete densification. SPS has been effectively used to make solid disk-like targets of a wide range of materials including chalcogenides~\cite{Jinfeng2, Subhash1}, and its efficiency in forming clean grain boundaries in polycrystalline targets has been shown for nitrides and refractory high-temperature materials~\cite{Subhash2, Jinfeng1}. The target surface was polished with a 800~grit diamond sandpaper before transferring to high vacuum. For final conditioning of the target surface and to deposit EuS thin films, the target was spun at 18 rpm and ablated by a 25~ns 248~nm KrF excimer pulsed laser beam in high vacuum ($p=6\times{}10^{-7}~\mathrm{Torr}$) at 10~Hz repetition rate. The typical ablation spot size was $2.1\pm0.3~\mathrm{mm^2}$ and the measured fluence was $1.0\pm0.2~\mathrm{J\cdot{}cm^{-2}}$. The ablation results in broad plumes a few centimeters in size. Corundum Al$_2$O$_3$ (0001) and Si (100) substrates were cleaned \textit{ex situ} by solvent sonication prior to transfer to high vacuum. The substrates were heated to $650~^{\circ}\mathrm{C}$ and placed 5~cm away from the target at the center of the plasma plume. The growth rate was estimated to be 1.3~\AA{} per pulse. After each deposition, the substrates were cooled in vacuum to $60~^{\circ}\mathrm{C}$ with a rate no faster than $15~^{\circ}\mathrm{C}/\mathrm{min}$.

The resultant thin films with thicknesses $20~\mathrm{nm}<t<200~\mathrm{nm}$ have a translucent purple color on Al$_2$O$_3$ and are dark green on Si. Figure~\ref{fig:EuS_TEM} shows a transmission electron micrograph (TEM) of a thin film cross-section, where the FCC lattice of EuS can be clearly observed.\footnote{TEM was done by Ann F. Marshall at the Stanford Nanocharacterization Laboratory.} %
%
\begin{figure}[ht]%
    \subfloat{\label{fig:EuS_TEM}}%
    \subfloat{\label{fig:EuS_AFM}}%
    \centering%
    \includegraphics[width=0.95\columnwidth]{figs_EuS/EuS_TEM_AFM.jpg}%
    \caption[Cross-section and surface micrographs of an EuS thin film]{\label{fig:EuS_TEM_AFM}(Color online) (a)~Cross-sectional TEM image of an EuS thin film, showing its interface to the Al$_2$O$_3$ (0001) substrate. (b)~AFM image of a $1~\mathrm{\mu{}m}\times{}1~\mathrm{\mu{}m}$ area, showing the surface topography of a $20~\mathrm{nm}$ EuS film. The root-mean-square roughness of $\sigma=1.8~\mathrm\AA$ indicates smoothness to the atomic scale.}%
\end{figure}%
%
The lattice constant is estimated from direct length measurements to be $a=6.0\pm0.2~\mathrm\AA$, consistent with the established results~\cite{EuS_Shafer}. Surface topography was measured with an atomic force microscope (AFM). Figure~\ref{fig:EuS_AFM} shows a $1~\mathrm{\mu{}m}\times{}1\mathrm{\mu{}m}$ area on the surface of a 20~nm film on Al$_2$O$_3$. The difference between the minima and maxima in height is roughly twice the lattice constant. The root-mean-square roughness $\sigma=1.8~\mathrm\AA$ calculated from a randomly selected line profile indicates near-ideal smoothness. Similar smoothness were obtained on films with thicknesses up to 200~nm deposited on either Al$_2$O$_3$~(0001) or Si~(100).

Figure~\ref{fig:EuS_XRD} shows the X-ray diffraction spectra of a few PLD grown EuS thin films.%
\begin{figure}[ht]%
    \subfloat{\label{fig:EuS_XRD_sap}}%
    \subfloat{\label{fig:EuS_XRD_Si}}%
    \centering%
    \includegraphics[width=0.75\columnwidth]{figs_EuS/XRD}%
    \caption[X-ray diffraction spectra of EuS thin films]{\label{fig:EuS_XRD}(Color online) Semi-log X-ray diffraction patterns of 20~nm EuS thin films. (a)~On Al$_2$O$_3$ (0001) substrates, optimal growth conditions lead to a single (100) orientation, whereas multiple orientations were observed in non-optimal samples. Spikes near substrate peaks are due to the K-$\beta$ components of the X-ray source. (b)~On Si (100) substrates with native oxides, single (100) orientation was obtained at optimal growth conditions. To distinguish the EuS (400) peak, a monochromator was used in the Si (100) case to eliminate the K-$\beta$ components.}%
\end{figure} %
%
On both Al$_2$O$_3$ (0001) and Si (100) substrates, the optimal conditions described earlier produced samples with a single orientation where the [100] planes are parallel to the substrate surface. On Al$_2$O$_3$ (0001) substrates (figure~\ref{fig:EuS_XRD_sap}), the (200) and (400) reflections are easily identified whereas the (600) reflection is discernible from the background. On Si (100) substrates (figure~\ref{fig:EuS_XRD_Si}), all [100] reflections are clearly observable. For comparison, the diffraction pattern of a non-optimal sample deposited on Al$_2$O$_3$ (0001) at a lower temperature ($T=600~^{\circ}\mathrm{C}$) was plotted in figure~\ref{fig:EuS_XRD_sap}. Reflections from both the [100] and the [111] orientations were observed with comparable weights. Similar multiple orientations were observed in samples deposited at higher-than-optimal temperatures ($T>700~^{\circ}\mathrm{C}$) or higher ambient pressures ($p>2\times{}10^{-6}~\mathrm{Torr}$).

The resistances of the EuS thin films were measured with the van der Pauw technique (\textit{q.v.} section~\ref{sec:vdp})~\cite{VdP1958}. When deposited at the optimal conditions on either Al$_2$O$_3$ (0001) or Si (100) substrates, samples with thicknesses $20~\mathrm{nm}<t<200~\mathrm{nm}$ all show sheet resistances $R_\Box>20~\mathrm{M\Omega}$ at temperatures $T > 100~\mathrm{K}$, and immeasurably  high resistance at lower temperatures. This is equivalent to bulk resistivity exceeding  $\rho>400~\mathrm{\Omega\cdot{}cm}$, consistent to values obtained on high-purity single crystals~\cite{EuS_Shafer}. In contrast, films deposited at non-optimal conditions show sheet resistances as low as $R_\Box\sim~\mathrm{k\Omega}$ (figure~\ref{fig:EuS_RvT}), %
%
\begin{figure}[ht]%
    \subfloat{\label{fig:EuS_RvT}}%
    \subfloat{\label{fig:EuS_RvH}}%
    \centering%
    \includegraphics[width=0.95\columnwidth]{figs_EuS/trans}%
    \caption[Electrical properties of EuS thin films]{\label{fig:EuS_transport}(Color online) (a)~While samples grown at optimal conditions have sheet resistance $R_\Box>20~\mathrm{M\Omega}$ for $2~\mathrm{K}<T<300~\mathrm{K}$, samples grown under non-optimal conditions (N1--N3, with 200~nm thickness) show finite resistance, indicating high carrier densities. (b)~These non-optimal samples show negative giant magnetoresistance at $T=2~\mathrm{K}$, similar to that observed in n-type single crystals.}%
\end{figure}%
%
which corresponds to a bulk resistivity of $\rho\sim10^{-2}\mathrm{\Omega\cdot{}cm}$, consistent with the conductive r\'egime in doped single crystals~\cite{EuS_ntype}. Different from n-type single crystals results, where resistance anomalies were observed near $T_C$ and attributed to change in carrier concentrations~\cite{EuS_ntype, EuX_doped_transport}, monotonic increases in resistance were observed towards low temperatures in thin films. Such difference could be resulted from different natures of dopants or due to the effects of reduced dimensionality~\cite{2D_conduction}. Similar to n-type doped single crystals, negative giant magnetoresistance was observed in the conducting samples at low temperatures (figure~\ref{fig:EuS_RvH}).

Magnetizations of the thin films were measured in a superconducting quantum interference device (SQUID) magnetometer down to $T=2~\mathrm{K}$. A significant perpendicular component of the magnetization was observed (figure~\ref{fig:EuS_MvH_z}), whereas the easy axes are in the sample plane (figure~\ref{fig:EuS_MvH_x}).%
%
\begin{figure}[ht]%
    \subfloat{\label{fig:EuS_MvH_z}}%
    \subfloat{\label{fig:EuS_MvH_x}}%
    \subfloat{\label{fig:EuS_MvT_z}}%
    \subfloat{\label{fig:EuS_Sagnac}}%
    \centering%
    \includegraphics[width=0.90\columnwidth]{figs_EuS/mag}%
    \caption[Magnetic properties of EuS thin films]{\label{fig:EuS_magnetic}(Color online) Magnetization of a 20~nm EuS film on Al$_2$O$_3$ (0001), (a)~in perpendicular fields, (b)~in parallel fields, and (c)~its temperature dependence, plotted in the same arbitrary unit with a linear paramagnetic component of the substrate subtracted. (a) and (b) were measured at $T=2~\mathrm{K}$. The temperature dependence curve is of the parallel component, and was measured in a parallel external magnetic field $\mu_0H = +20~\mathrm{mT}$ while warming up, after exposure to $\mu_0H=+2~\mathrm{T}$ at $T=2~\mathrm{K}$. The fitting to the Curie-Weiss Law indicates a low Curie temperature $T_C=15.9~\mathrm{K}$. (d)~Kerr effect measured with a scanning Sagnac interferometer at $T=10~\mathrm{K}$, showing uniform magnetization. Errorbars represent 90\% confidence intervals.}%
\end{figure} %
%
While the coercive field of perpendicular magnetization may vary within the same order of magnitude for film thicknesses between 20~nm and 200~nm, the general hysteresis features are similar for all our samples on either Al$_2$O$_3$ (0001) or Si (100). By fitting to the Curie-Weiss law in the paramagnetic regime (figure~\ref{fig:EuS_MvT_z}), an upper limit of the Curie temperature of an optimal 20~nm thin film on Al$_2$O$_3$ (0001) was estimated to be $T_C=15.9~\mathrm{K}$. As discussed earlier in section~\ref{sec:EuS_previous}, a low $T_C$ is expected for thin film samples with diminishing carrier densities~\cite{EuS_TC_doping, EuS_ntype, EuX_doped_transport, thickness_Tc_theory, thickness_Tc_exp}. Compared to the lowest $T_C$ recorded of high-quality single crystals ($T_C\approx16.5~\mathrm{K}$), the lower $T_C$ is an indication of high thin film qualities.

To test for homogeneity of the magnetism in our films we used a scanning Sagnac interferometer.\footnote{Scanning Sagnac interferometry measurements were performed and analyzed by Alexander Fried.} This device is based on a zero-area loop Sagnac interferometer that was first demonstrated by \citeauthor{Xia2006}~\cite{Xia2006}, and can measure the Polar Kerr angle upon reflection from the film with shot-noise limited sensitivity at low power. Our scanning device is operated at a wavelength of 820~nm, and has a spatial resolution of 0.9~$\mu$m. Figure~\ref{fig:EuS_Sagnac} shows several line scans of length~100 $\mu$m, taken at a temperature of 10~K and at low magnetic fields, showing a very uniform Kerr response. This set of line scans also agrees with the coercive field found in the SQUID magnetometry measurements.

While the experimental setup for PLD is relatively simple, it is well known that complex and non-equilibrium mechanisms are involved in both laser ablation of the target and plume-substrate interactions (\textit{q.v.} section~\ref{sec:pld}). Here we present a tentative discussion on the growth process. At the optimal growth conditions, the resultant atomically smooth topography seems to suggest the Frank--van der Merwe mode of nucleation. The absence of microstructures, which indicates sufficient reduction of partial evaporation (``splashing'')~\cite{PLD_book}, may have been in part due to the effective densification with the SPS technique and appropriate target surface treatment. While EuS solid is stable up to $2300~^{\circ}\mathrm{C}$ in vacuum~\cite{EuS_phase_diagram}, we found that the film quality is sensitive to relatively small deviations ($\Delta{}T=\pm50~^{\circ}\mathrm{C}$) from the optimal substrate temperature $T=650~^{\circ}\mathrm{C}$. Lowering the substrate temperature or increasing the ambient pressure are known to increase the cooling rate of adatoms. Since the constituent elements of EuS have a large difference in vapor pressures: $p(\mathrm{S})/p(\mathrm{Eu})>10^4$ at $T=650~^{\circ}\mathrm{C}$~\cite{elements}, Such effects might worsen both stoichiometry and structure~\cite{Metev1988}. In addition, the detrimental effects of changing temperature at either directions may be related to the nearby eutectic point at $750~^{\circ}\mathrm{C}$ and the EuS$_2$ phase below $575~^{\circ}\mathrm{C}$~\cite{EuS_phase_diagram}, which may provide transient states that facilitate structural distortions or an Eu-rich stoichiometry. In either case, the distortion in stoichiometry would likely result in unintended doping.

To further illustrate the improvements over previously samples, particularly in the context of our present application, the Curie temperatures and room temperature resistivity of reported samples are compared in table~\ref{tab:EuS_comparison}.%
%
\begin{table}[ht]%
    \centering%
    \begin{tabularx}{1.0\columnwidth}[t]{l|l|l|l|X}
        \caption[Comparison of EuS sample qualities]{\label{tab:EuS_comparison}Comparison of EuS sample qualities. Inferred from their Curie temperatures, previously reported thin films are likely similar the n-type doped crystals and therefore should exhibit considerable electric conduction, whereas thin films obtained with our improved PLD recipe behave similarly to the high-quality stoichiometric single crystals with high resistivity.}\\
        \hline\hline
        Sample types & Thicknesses & $T_C~\mathrm{(K)}$ & RT $\rho~\mathrm{(\Omega\cdot{}cm)}$ & Sources\\
        \hline
        crystal (stoichiometric) & N/A & 16.7 & $10^1$ & \citeauthor{EuS_Shafer}~\citeyear{EuS_Shafer}~\cite{EuS_Shafer}\\
        \hline
        crystal (n-type) & N/A & 20 & $10^{-2}$ & \citeauthor{EuS_ntype}~\citeyear{EuS_ntype}~\cite{EuS_ntype}\\
        \hline
        MBE & 50~nm & 16.7 & unreported & \citeauthor{EuS_MBE_Dauth}~\citeyear{EuS_MBE_Dauth}~\cite{EuS_MBE_Dauth}\\
        \hline
        MBE & 6~nm & 18 & unreported & \citeauthor{EuS_MBE_Muller}~\citeyear{EuS_MBE_Muller}~\cite{EuS_MBE_Muller}\\
        \hline
        MBE & 1--10~nm & 20 & unreported & \citeauthor{Moodera2013}~\citeyear{Moodera2013}~\cite{Moodera2013}\\
        \hline
        e-beam evap. & 50~nm & 18 & $10^{-1}$ & \citeauthor{EuS_thin_film_Keller}~\citeyear{EuS_thin_film_Keller}~\cite{EuS_thin_film_Keller}\\
        \hline
        PLD & 60--700~nm & 18 & unreported & \citeauthor{EuS_PLD1}~\citeyear{EuS_PLD1}~\cite{EuS_PLD1}\\
        \hline
        \\
        \hline
        PLD & 20~nm & 15.9 & $>10^{2}$ & this chapter\\
        \hline\hline
    \end{tabularx}
\end{table} %
%
While previous research often focused on the magnetic properties of EuS thin films and left their resistivity unreported, the expected electric properties may be inferred from the Curie temperatures by considering the known effects of reduced dimensionality and excessive carriers. As discussed in section~\ref{sec:EuS_previous}, the former lowers the Curie temperature of EuS, whereas the latter increases the Curie temperature, therefore high quality insulating thin films of EuS should exhibit lower Curie temperatures than their crystalline counterparts. However, with exception of the \citeauthor{EuS_MBE_Dauth}~\citeyear{EuS_MBE_Dauth} case, all previous thin films samples have Curie temperatures higher than the stoichiometric single crystal value, therefore likely have considerable electric conduction similar to the n-type doped crystals. In contrast, the EuS thin films obtained by our improved PLD recipe exhibit Curie temperatures that are lower than bulk crystal values, and insulating behavior that is similar to high quality stoichiometric single crystals.

So far in this chapter, the focus has been on the properties of EuS thin films. The perpendicular component of spontaneous magnetization was demonstrated by SQUID magnetometry (figure~\ref{fig:EuS_MvH_z}), the homogeneity of the perpendicular magnetization by scanning Sagnac interferometry (figure~\ref{fig:EuS_Sagnac}), the surface smoothness by AFM (figure~\ref{fig:EuS_AFM}), and electric insulation by resistivity measurements. Having established these properties of EuS thin films, henceforth in the following chapters, experiments will be presented with two different topological insulators interfacing with EuS.

\chapter{Bismuth(III) Selenide on Europium(II) Sulfide}
\label{ch:bilayer2014}\footnote[2]{A part of this chapter is adapted from ref.~\cite{bilayer2014}~\fullcite{bilayer2014}, with permission of the publisher. Copyright (2013) by the American Physical Society.}%
%
At the conclusion of the experiments presented in this chapter and the publication of ref.~\cite{bilayer2014}, probably the most extensively studied three-dimensional topological insulator (3D TI) had been bismuth(III) selenide (Bi$_2$Se$_3$)~\cite{TI_electronic_structure_zhang, Zhanybek3, TI_other1}, exhibiting crystal structure that consists of atomic quintuple layers (QLs), with three QLs forming a unit cell. As made, uncompensated samples typically have a Fermi level above the Dirac point and intersecting the bulk conduction band~\cite{TI_ARPES1, ARPES_thickness}. In particular, low temperature transport measurements on ungated and uncompensated TI films show positive magnetoresistance (MR) at low magnetic fields and in a wide range of film thicknesses~\cite{ TI_WAL_Hongkong, TI_WAL_thickness, zhangli2012}. This was explained in terms of weak antilocalization (WAL) resulted from spin-momentum locking on the surface state Dirac cone (\textit{q.v.} sections \ref{sec:ti} \& \ref{sec:wl})~\cite{TI_WAL_Hongkong, bergmann1984}. While the inability to account for the bulk bands (presumably because of their low mobility) has challenged this simple assignment, the discovery of weak localization (WL) effects at higher fields~\cite{zhangli2013} and the ability to accurately separate quantum oscillation effects~\cite{Ando_PRL} in high-quality films may provide a first step towards a more comprehensive understanding of transport in these systems.

As initial steps to investigate the effects due to proximity between a 3D TI and an insulating ferromagnet (IF), Bi$_2$Se$_3$ thin films were deposited on top of the high-quality EuS thin films presented in chapter~\ref{ch:EuS}, forming thin film bilayers (BLs). The pulsed laser deposition (PLD) of Bi$_2$Se$_3$ thin films on EuS~(100) and Al$_2$O$_3$~(0001) are described in section~\ref{sec:bilayer2014_char}. The magnetic properties of the resultant thin films are documented in section~\ref{sec:bl2014_mag}. In section~\ref{sec:bl2014_negtive_MR}, magneto-transport measurements are presented, demonstrating a negative magnetoresistance that emerges below the Curie temperature of EuS, in contrast to the ubiquitously observed positive magnetoresistance in topological insulator thin films.

\section{Sample Fabrication}\label{sec:bilayer2014_char}
The fabrication of all the bilayer samples in this chapter follows a scheme illustrated by figure~\ref{fig:bl2014_sketch}.%
\begin{figure}[h]%
    \centering%
    \subfloat{\label{fig:bl2014_sketch}}%
    \subfloat{\label{fig:bl2014_TEM}}%
    \includegraphics[width=0.95\columnwidth]{figs_bilayer2014/schetch_tem}%
    \caption[Schematic and cross-section micrograph of Bi$_2$Se$_3$--EuS thin film bilayers]{(Color online) (a)~Schematic of a bilayer device: an EuS film was deposited on an Al$_2$O$_3$(0001) substrate, followed by 15nm Titanium contacts with gradual height on the edges, finally a Bi$_2$Se$_3$ layer were deposited. (b)~Transmission electron micrograph of a cross-section of a bilayer sample, showing the quintuple layers (QL) of Bi$_2$Se$_3$ and a smooth TI-IF interface.}%
\end{figure} %
%
For the bottom layer, EuS thin films of thicknesses 20--200nm were grown on Al$_2$O$_3$ (0001) using the optimal procedure outlined in chapter~\ref{ch:EuS}. Titanium Ohmic contacts were subsequently deposited \textit{ex situ} on each EuS thin film through a shadow mask with electron-beam evaporation. Electric insulation ($R_\Box > 20~\mathrm{M\Omega}$) was verified for $2~\mathrm{K}<T<300~\mathrm{K}$. Finally, the Bi$_2$Se$_3$ layer was deposited, overlapping with the titanium contacts at their inner corners, forming a van der Pauw configuration (\textit{q.v.} section~\ref{sec:vdp}).

Similarly to the EuS targets in section~\ref{sec:EuSPLD}, PLD targets of Bi$_2$Se$_3$ were prepared by SPS~\cite{Jinfeng2, Subhash1},\footnote{Target preparation was carried out by Jinfeng Zhao and Subhash H. Risbud at the Department of Chemical Engineering and Materials Science, University of California, Davis.}\footnote{PLD recipe for Bi$_2$Se$_3$ was developed together with Li Zhang.} and the target surface was polished with a 800~grit diamond sandpaper before transferring to the vacuum chamber. Unlike EuS, the plasma plumes generated by ablating Bi$_2$Se$_3$ in high vacuum are narrow in comparison to the substrate size ($\sim$5~mm). To diffuse the plasma plumes and to prevent potential oxidation from residual gases, 200~millitorr of argon gas mixed with 2\% hydrogen was introduced after pumping to $10^{-6}$ high vacuum. For final target surface treatment and to deposit the Bi$_2$Se$_3$ thin films, the target was ablated by a 25~ns 248~nm KrF excimer pulsed laser beam at 5~Hz repetition rate and 0.54~$\mathrm{J\cdot{}cm^{-2}}$ fluence, while spinning at 18 rpm. EuS thin films on Al$_2$O$_3$(0001) substrates were heated to $\SI{150}{\degreeCelsius}$, $\SI{5}{cm}$ away from the target. The growth rate was estimated to be 0.15~\AA{} per pulse.

To compensate for selenium deficiencies that are typical for as-grown Bi$_2$Se$_3$ thin films \cite{Zhanybek3, zhangli2012, zhangli2013, TI_ARPES1, ARPES_thickness}, a selenium capping layer was deposited \textit{in situ} immediately following the Bi$_2$Se$_3$ layer by ablating a selenium sputtering target (Kurt J. Lesker, 99.999\% pure) for 300 pulses. The sample was annealed with the capping layer at $\SI{150}{\degreeCelsius}$ for roughly 15 minutes before the heater was switched off. When the sample was cooled to below $\SI{80}{\degreeCelsius}$, an additional selenium capping layer was applied before venting, to prevent exposure to air.

The cross-section transmission electron micrograph (figure~\ref{fig:bl2014_TEM}) indicates smooth interface and excellent layering of the Bi$_2$Se$_3$. While the brightness is heightened at the interface in figure~\ref{fig:bl2014_TEM}, individual atoms are still discernible when the micrograph is presented in digital format. Such heightened brightness is likely artefact resulted from the preparation of the cross section, which involves mechanical polishing followed by argon ion milling. Very different mechanical properties of EuS (cubic lattice with ionic bonds) and Bi$_2$Se$_3$ (quintuple layers weakly bound by van der Waals force) likely result in different response to mechanical polishing, which might be further compounded by different etching rate during ion milling. Similar heightened brightness is observed at the interface between EuS and Al$_2$O$_3$ to a lesser extent (\textit{cf.} figure~\ref{fig:EuS_TEM}).

To control for potential differences between Bi$_2$Se$_3$ thin films by PLD and previously reported thin films, which were fabricated predominantly by molecular beam epitaxy (MBE), a bare Al$_2$O$_3$(0001) substrate with Ohmic contacts was placed next to the EuS thin film in each PLD session. Therefore, for each bilayer sample, a Bi$_2$Se$_3$--only sample was fabricated with the same deposition condition and the same geometry. To examine the relevance of surface conduction, samples with different thicknesses for the Bi$_2$Se$_3$ layer were fabricated and compared. The samples of interest are summarized in table~\ref{tab:bl2014_samples}.%
\begin{table}[ht]
    \centering
    \begin{tabularx}{0.6\columnwidth}[t]{l|l|X}
    \caption[Summary of Bi$_2$Se$_3$ thin films and Bi$_2$Se$_3$--EuS bilayer samples]{\label{tab:bl2014_samples}Summary of Bi$_2$Se$_3$ thin films and Bi$_2$Se$_3$--EuS bilayer samples discussed in this chapter. All samples are on Al$_2$O$_3$ (0001) substrates. The Bi$_2$Se$_3$ layers on the TI and BL samples with the same numerical suffix were deposited in the same PLD session.}\\
		\hline\hline
        Samples & Configurations & TI Thicknesses\\
        \hline
        TI0 & Bi$_2$Se$_3$ only & 5~nm\\
        TI3 & Bi$_2$Se$_3$ only & 3~nm\\
        BL0 & Bi$_2$Se$_3$ on EuS & 5~nm\\
        BL1 & Bi$_2$Se$_3$ on EuS & 3~nm\\
        BL2 & Bi$_2$Se$_3$ on EuS & 3~nm\\
        BL3 & Bi$_2$Se$_3$ on EuS & 3~nm\\
        BL4 & Bi$_2$Se$_3$ on EuS & 3~nm\\
		\hline\hline
    \end{tabularx}
\end{table} %
The Bi$_2$Se$_3$ layers on the bilayer (BL) and Bi$_2$Se$_3$--only (TI) samples with the same numerical suffix were deposited in the same batch.

\FloatBarrier%
\section{Magnetic Properties}\label{sec:bl2014_mag}
A crucial condition for the ferromagnet to affect the topological insulator is to break the time-reversal symmetry (TRS) in the relevant dimensions, \textit{e.g.,} by a magnetization perpendicular to the interface~\cite{TI_Col, QAH_TI_Yu, MnSe}. The perpendicular magnetization in EuS thin films was confirmed by SQUID magnetometry, and discussed in chapter~\ref{ch:EuS}. Figure~\ref{fig:bl2014_squid} shows SQUID measurements on an example of EuS thin films used to fabricate bilayers.\footnote{Error bars in this chapter represent 90\% confidence intervals.}%
%
\begin{figure}[ht]%
    \centering%
    \subfloat{\label{fig:bl2014_MvH}%
        \includegraphics[width=0.45\columnwidth]{figs_bilayer2014/MvH_EuS04}%
    }%
    \subfloat{\label{fig:bl2014_MvT}
        \includegraphics[width=0.45\columnwidth]{figs_bilayer2014/MvT_EuS04}%
    }%
    \caption[SQUID magnetometry of an EuS thin film used for bilayers]{\label{fig:bl2014_squid}(Color online) SQUID magnetometry of an EuS thin film used for bilayers. (a)~The perpendicular component of the magnetization as functions of the perpendicular magnetic fields. (b)~The ferromagnetic transition observed in a small perpendicular field. A fitting to the Curie-Weiss law in the paramagnetic r\'egime yields a Curie temperature of $T_C = \SI{15.7}{K}$.}%
\end{figure} %
%
Ferromagnetic transition was observed at a Curie temperature $T_C = \SI{15.7}{K}$.

Since such TRS breaking can be directly probed by measuring the polar Kerr effect~\cite{Xia2006}, the bilayer sample BL3 was measured with Sagnac interferometry in two configurations. Measured at a fixed position where the EuS is covered by the Bi$_2$Se$_3$ layer,\footnote{Polar Kerr effect at fixed position was measured and analyzed by Elisabeth Schemm.} the Kerr angles exhibit hysteresis at $T=\SI{310}{mK}$ in a sweeping magnetic field perpendicular to the film (figure~\ref{fig:bl2014_SchemmH}),%
%
\begin{figure}[ht]%
\centering%
\subfloat{\label{fig:bl2014_SchemmT}}%
\subfloat{\label{fig:bl2014_SchemmH}}%
\includegraphics[width=\columnwidth]{figs_bilayer2014/schemm}
\caption[Polar Kerr effect in a Bi$_2$Se$_3$--EuS bilayer]{\label{fig:bl2014_schemm}(Color online) Polar Kerr effect measured by a Sagnac interferometer. Kerr angles were measured (a) at $T=\SI{310}{mK}$ as functions of the sweeping magnetic field perpendicular to the film, where the arrows denote the directions of field change; (b) in zero magnetic field as a function of increasing temperature, after exposure to a 5~kOe field at \SI{310}{mK}.}%
\end{figure} %
%
and disappear when the temperature exceeds the $T_C$ of EuS (figure~\ref{fig:bl2014_SchemmH}). This confirms that the TRS breaking observed is originated from the ferromagnetic state in the EuS. When measured at $T=\SI{10}{K}$, and scanning on a straight line across the boundary between a bare EuS region and one covered by the Bi$_2$Se$_3$ layer,\footnote{Scanning Sagnac interferometry was carried out and analyzed by Alexander Fried.} the Kerr angles are reduced by the presence of the Bi$_2$Se$_3$, while remaining finite in size (figure~\ref{fig:bl2014_Fried}).%
%
\begin{figure}[ht]%
\centering%
\includegraphics[width=0.95\columnwidth]{figs_bilayer2014/fried}
\caption[Spatial distribution of the Kerr angles in a Bi$_2$Se$_3$--EuS bilayer]{\label{fig:bl2014_Fried}(Color online) Polar Kerr effect measured by a scanning Sagnac interferometers at $T=\SI{10}{K}$. Kerr angles were measured as functions of applied magnetic fields and in-plane distance across the edge of the Bi$_2$Se$_3$ layer, where $x<0$ region is bare EuS and $x>0$ is Bi$_2$Se$_3$ on top of EuS.}%
\end{figure} %
The Kerr angles in both regions indicate a uniform spatial distribution of magnetization, without obvious domain boundaries.

\FloatBarrier%
\section{Emergent Negative Magnetoresistance}\label{sec:bl2014_negtive_MR}
Hitherto in this thesis, we have ascertained the necessary conditions stated in section~\ref{sec:ti} for studying the proximity effects between a 3D TI and a ferromagnet. The perpendicular component of magnetization was confirmed by SQUID magnetometry (figure~\ref{fig:EuS_MvH_z}), and the consequent TRS breaking by the Sagnac interferometry (figure~\ref{fig:bl2014_schemm}). The uniformity of the magnetization was verified by scanning Sagnac interferometry (figure~\ref{fig:bl2014_Fried}). The interface between the TI and the ferromagnet was observed in the cross-section electron micrograph (figure~\ref{fig:bl2014_TEM}). The absence of electric conduction in the ferromagnet was confirmed by direct measurements (\textit{q.v.} chapter~\ref{ch:EuS}).

The manifestation of such proximity effects was observed in the magneto-transport properties of the bilayer samples. The magnetoresistance of four samples of two representative thicknesses are shown in figure~\ref{fig:bl2014_MR_thickness}.%
%
\begin{figure}[ht]%
\centering%
\subfloat{\label{fig:TI0_MR}}%
\subfloat{\label{fig:TI3_MR}}%
\subfloat{\label{fig:BL0_MR}}%
\subfloat{\label{fig:BL3_MR}}%
\includegraphics[width=0.95\columnwidth]{figs_bilayer2014/MR_thickness}%
\caption[Magnetoresistance of Bi$_2$Se$_3$ thin films and Bi$_2$Se$_3$--EuS bilayers]{\label{fig:bl2014_MR_thickness}(Color online) Magnetoresistance and its temperature dependence of PLD--grown Bi$_2$Se$_3$ thin films (TI) and Bi$_2$Se$_3$--EuS bilayer (BL) samples. (a, b)~The Bi$_2$Se$_3$--only samples have positive magnetoresistance regardless of thicknesses. (c)~The BL samples with TI-layer thicknesses $t\gtrsim4\mathrm{QL}$ behave similarly to the TI--only films. (d)~With TI--layer thicknesses $t\lesssim4\mathrm{QL}$, a distinctive negative magnetoresistance is observed at low-fields in BLs below the Curie temperature of EuS. The thickness limit coincides with occurrence of coupling between the top and the bottom surfaces of a TI thin film.}%
\end{figure} %
%
As ubiquitously seen in TI thin films, the PLD--grown Bi$_2$Se$_3$--only samples show positive magnetoresistance at low fields regardless of thicknesses, which broadens monotonically with increasing temperature (figures~\ref{fig:TI0_MR}~\& \ref{fig:TI3_MR}), consistent with the weak antilocalization effect (WAL). Fittings to the standard Hikami-Larkin-Nagaoka (HLN) formula describing the WAL magnetoconductance (\textit{q.v.} section \ref{sec:wl}) yield dephasing lengths ($l_\phi$) comparable to MBE--grown samples with the same thicknesses~\cite{TI_WAL_thickness, zhangli2013}. With thicknesses of the Bi$_2$Se$_3$ layer greater than $\sim4\mathrm{QL}$, the TI--IF bilayers have similar low-field magnetoresistance features to their TI--only counterparts (figure~\ref{fig:BL0_MR}). By contrast, with TI--layer thicknesses $t\lesssim4\mathrm{QL}$, the bilayers show distinctive negative low-field magnetoresistance at low temperatures (figure~\ref{fig:BL3_MR}). Such negative magnetoresistance features are clearly distinguishable well below the Curie temperature ($T_C=15.7$K) of EuS (figures~\ref{fig:bl2014_2K8K} \& \ref{fig:bl2014_8K16K}), %
%
%
\begin{figure}[ht]%
\centering%
\subfloat{\label{fig:bl2014_16K30K}}%
\subfloat{\label{fig:bl2014_8K16K}}%
\subfloat{\label{fig:bl2014_2K8K}}%
\subfloat{\label{fig:bl2014_He3}}%
\includegraphics[width=0.95\columnwidth]{figs_bilayer2014/MR_temperature}%
\caption[Temperature dependence of the magnetoresistance of a Bi$_2$Se$_3$--EuS bilayer]{\label{fig:bl2014_MR_temperature}(Color online) Low-field magnetoresistance of a TI--IF bilayer device (BL3, $t\approx3\mathrm{QL}$) in magnetic fields perpendicular to the film: (a)~Above $T_C$, a WAL--like positive magnetoresistance sharpens with decreasing temperature; (b)~Just below $T_C$, magnetoresistance broadens with decreasing temperature; (c)~Well below $T_C$, a sharp negative magnetoresistance emerges near zero field. The emergence of negative magnetoresistance below $T_C$ of the EuS indicates a TI--IF proximity effect. (d)~Magnetoresistance of BL3 below $T=2$K, with solid lines as illustrative guides. Resistance at $T\leq0.8\mathrm{K}$ was measured with two-terminal configurations.}
\end{figure}%
%
whereas positive magnetoresistance resembling WAL appears at higher fields. Below and close to $T_C$ (figure~\ref{fig:bl2014_8K16K}), the negative magnetoresistance can no longer be directly observed. However, its remnant contribution reverses the thermal broadening of the overall positive magnetoresistance. This suggests that the mechanism producing such negative magnetoresistance is reduced rapidly close to the ferromagnetic transition. Above $T_C$ (figure~\ref{fig:bl2014_16K30K}), the positive magnetoresistance is eventually broadened when increasing the temperature, similar to common WAL features in TI--only thin films. In figures~\ref{fig:bl2014_MR_thickness}~\&~\ref{fig:bl2014_MR_temperature}, we presented the negative magnetoresistance in the same sample, whereas four bilayer samples (labeled as BL1--4, table~\ref{tab:bl2014_samples}) with $t\lesssim4\mathrm{QL}$ from different growth batches all demonstrated such proximity effect in a consistent manner. The thickness criterion ($t\lesssim4\mathrm{QL}$) coincides with the thickness below which the two surfaces of a Bi$_2$Se$_3$ film are observed to be coupled~\cite{ARPES_thickness}, suggesting that the mechanism for such negative magnetoresistance is originated from surface effects.

To quantify the changes in the low-field magnetoresistance with respect to temperatures, a quadratic model was fitted to the magnetoresistance data near zero field:\footnote{At suggestion by Xiaoliang Qi.}%
\begin{equation}\label{eq:bl2014_quadratic}%
    \frac{\Delta R(H)}{R(0)} = a(T)\cdot H^2~.%
\end{equation}%
Since at different temperatures the magnetoresistance has different low-field and high-field features, the ranges of the data points to fit the model was chosen dynamically. For each temperature, the largest field interval, $H \in [-H_0, +H_0]$, up to $H_0 = 0.5 \mathrm{T}/\mu_0$, was chosen such that the mean-square of the residuals ($\overline{r^2}$) is still comparable to that of the statistical errors of the data points ($\overline{\epsilon^2}$):
\begin{equation}
    \overline{r^2} < 2\overline{\epsilon^2}~.%
\end{equation}%
The resultant fitting intervals are narrower for the sharp negative magnetoresistance at low temperatures (figure~\ref{fig:bl2014_quadratic}a),%
%
\begin{figure}[ht]%
    \centering%
    \includegraphics[width=0.95\columnwidth]{figs_bilayer2014/quadratic}%
    \caption[Quadratic fittings to the magnetoresistance of a Bi$_2$Se$_3$--EuS bilayer]{\label{fig:bl2014_quadratic}(Color online) The magnetoresistance of Bi$_2$Se$_3$--EuS bilayers is fitted at low fields to a quadratic model $\Delta R(H) / R(0) = a(T)\cdot H^2$. The fitting ranges were dynamically chosen such that the mean-square of the residuals are comparable to that of the measurement errors. The data points (crosses) and fittings (solid curve) are shown for (a)~$T=2~\mathrm{K}$, and (b)~$T=30~\mathrm{K}$. The fitting parameter, $a$, is plotted as a function of the temperature for (c)~$2~\mathrm{K} \leq T \leq 30~\mathrm{K}$, and (d)~$10~\mathrm{K} \leq T \leq 30~\mathrm{K}$.}%
\end{figure} %
%
and wider for the broad positive magnetoresistance at higher temperatures (figure~\ref{fig:bl2014_quadratic}b). The fitting parameters, $a$, are shown as a function of the temperatures (figure~\ref{fig:bl2014_quadratic}c). Consistent with the previous qualitative observation in figures~\ref{fig:bl2014_16K30K}~\&~\ref{fig:bl2014_8K16K}, a maximum in $a(T)$ is found near the Curie temperature of the EuS, $T_C = \SI{15.7}{K}$ (figure~\ref{fig:bl2014_quadratic}d). The agreement between $T_C$ and the temperatures below which the negative magnetoresistance starts to dominate strongly indicates a proximity effect between the IF and the TI layers.

Below 2K (figure~\ref{fig:bl2014_He3}), we observed a continuous sharpening of the low-field negative magnetoresistance when lowering the temperature, as expected from diminishing thermal broadenings. Unexpectedly, the magnitude of negative magnetoresistance was reduced when lowering the temperature below 1K. This can be explained by the inhomogeneity observed in the Bi$_2$Se$_3$ layer grown on top of EuS. While the Bi$_2$Se$_3$ thin films grown on bare substrates were verified by AFM and XRD to be adequately uniform in thickness, the TEM images taken at different locations on the cross-section of the BL3 sample show large variations ($\pm2\mathrm{QL}$) in thickness of the Bi$_2$Se$_3$ layer, with an estimated mean value consistent with the thickness of the Bi$_2$Se$_3$-only film grown in the same PLD session. Near the thicknesses of interest ($t\lesssim4\mathrm{QL}$), both the resistance of Bi$_2$Se$_3$ films and its temperature dependence are known to change sharply with thickness at low temperatures~\cite{TI_WAL_thickness}. Thus special difficulty is introduced when measuring the sheet resistance with a van der Pauw method, where the sample thickness is assumed to be uniform (\textit{q.v.} section~\ref{sec:vdp})~\cite{VdP1958, VdP_contact_size}. With such inhomogeneous geometry, electric conduction is limited by the thinner and therefore more resistive parts of the sample. Indeed, the sheet resistance of the BL samples are one order of magnitude higher than that of the Bi$_2$Se$_3$-only samples with similar thicknesses (figure~\ref{fig:bl2014_RvT}).%
%
\begin{figure}[ht]%
    \centering%
    \includegraphics[width=0.65\columnwidth]{figs_bilayer2014/rvt}%
    \caption[Temperature dependence of the sheet resistance of Bi$_2$Se$_3$--EuS bilayers]{\label{fig:bl2014_RvT}(Color online) Temperature dependence of the sheet resistance of the four TI--IF bilayer samples that showed negative magnetoresistance (BL1--4) and a PLD--grown TI--only film (TI3) measured with the van der Pauw method.}%
    \end{figure} %
%
When lowering the temperature, the change in the zero-field resistance, $R(0)$, is dominated by the sharp temperature dependence of the thinner parts of the sample. The low-field features in magnetoresistance, however, are mainly contributed by the thicker parts, where the dephasing length is longer~\cite{TI_WAL_thickness}. Therefore, after dividing by $R(0)$ as the normalization factor, $\Delta{}R(H)/R(0)$ appears with a reduced magnitude. On the other hand, measuring the Hall effect, which is insensitive to film inhomogeneity~\cite{Landauer_Porous_Media}, the sheet carrier density is calculated showing similar values (\(n_{2D}\approx2\times10^{13}\mathrm{cm^{-2}}\)) for both TI--only films and TI--IF bilayers and are very close to reported values for MBE films~\cite{TI_WAL_thickness}.

Recent theories have suggested that the weak localization effect (WL) may occur in a TI thin film instead of WAL as result of gap-opening at the Dirac point~\cite{WL_WAL_competition, WL_Glazman, WL_bulk_Lu}. Besides an orbital effect such as WL, the presence of a magnetic material brings possibilities of magnetoresistance originated from scatterings at localized spins~\cite{KondoMR}. The magnitude of any spin-introduced magnetoresistance, however, should be greater above $T_C$ and should diminish rapidly below $T_C$ as the spins being aligned during the ferromagnetic transition~\cite{SpinMagnetic}. This is contrary to what we observed. Moreover, we show in figure~\ref{fig:bl2014_angular}%
%
\begin{figure}[ht]%
    \centering%
    \includegraphics[width=1.0\columnwidth]{figs_bilayer2014/angles}%
    \caption[Angular dependence of the magnetoresistance of a Bi$_2$Se$_3$--EuS bilayer]{\label{fig:bl2014_angular}(Color online) Angular dependence of the magnetoresistance of sample BL4, measured at $T = \SI{4}{K}$. (a, b)~As the sample is rotated from being perpendicular ($\theta = 0^\circ$) to being near parallel ($\theta = 89^\circ$) to the applied magnetic fields, both the low-field and the high-field features are broadened. (c, d)~The magnetoresistance coincide when plotted as functions of the effective perpendicular fields, $H_{eff} = H\cos(\theta^*)$. The individual low-field data points in (a) and (c) are shown in (b) and (d), respectively.}%
\end{figure} %
%
the broadening of the negative magnetoresistance as the sample is rotated from being perpendicular to being near parallel to the applied magnetic fields, favoring an orbital origin over a spin origin. While magnetoresistance remains finite in near-parallel fields, it is likely due to the uneven thickness in the Bi$_2$Se$_3$ layer, and consequently a locally slanted top surface, on which local electron paths may have a finite angle to the magnetic field that is parallel to the sample plane. In fact, when the magnetic field is scaled with an effective angle $\theta^*$ between the magnetic field and the normal of local electron transport, both the negative magnetoresistance at low fields and the positive magnetoresistance at higher fields are found to coincide well with perpendicular-field magnetoresistance (figure~\ref{fig:bl2014_angular}c--d). The differences between the effective angles ($\theta^*$) and the nominal angles read from the instrument ($\theta$) are small for most angles ($|\theta-\theta^*|\leq{}5^\circ$ for $\theta\leq{}60^\circ$), and lager towards parallel field ($\theta^*\approx{}72^\circ$ for $\theta=89^\circ$). This is consistent with expectations from a film with a smooth bottom surface and an uneven top surface.

While the orbital origin indicated by the angular dependence of magnetoresistance is consistent with the theoretical prediction of WL as consequence of gap-opening~\cite{WL_WAL_competition, WL_Glazman, WL_bulk_Lu, QAH_TI_Yu}, such phenomenon is expected to occur only when the Fermi level is sufficiently near the gap~\cite{WL_WAL_competition, WL_Glazman, WL_bulk_Lu}. However, as common in ungated and uncompensated Bi$_2$Se$_3$, our carrier densities suggest a Fermi level intersecting the conduction band and therefore far away from the Dirac point. Recent calculations suggest that surface charges on the IF layer may result in band-bending at the interface~\cite{MnSe}, hence provide a possible mechanism to move the gap towards the Fermi level. However, whether it is the case depends on the details of atomic arrangement at the interface~\cite{xiaoliang}, which cannot be determined with available data. On the other hand, low levels of sulfur doping are known to modify the band structure of Bi$_2$Se$_3$ while preserving its structural phase~\cite{Bi2Se3S, BiSeS}. Such effect may as well be present at the Bi$_2$Se$_3$-EuS interface. Attempts to perform ARPES measurements to determine the band gap, location of the Dirac point and the Fermi level produced inconclusive results, primarily due to the inhomogeneity in films thickness.

In parallel to the work presented in this chapter, a study contemporary to ref.~\cite{bilayer2014} on EuS / Bi$_2$Se$_3$ bilayers showed weak tendency towards anomalous Hall effect, which was argued to indicate emergence of a ferromagnetic phase in TI~\cite{Moodera2013}. Samples in that study consisted of a bottom Bi$_2$Se$_3$ film and a top EuS film, making it impossible to determine whether the EuS is truly insulating, and consequently whether the observed effect was from Bi$_2$Se$_3$ or due to conduction in the EuS layer. To contrast, the approach in this chapter allows the EuS and Bi$_2$Se$_3$ thin films to be fabricated and characterized separately. Since high-quality, well characterized EuS films were obtained prior to the deposition of Bi$_2$Se$_3$, the electric insulation in EuS was ascertained. Hence the the unusual negative magnetoresistance at low fields below the Curie temperature of the ferromagnet can only come from the Bi$_2$Se$_3$ layer, there is a strong indication of a proximity effect between a topological insulator and an insulating ferromagnet.


\chapter{Bismuth Doped Antimony(III) Telluride on Europium(II) Sulfide}
\label{ch:bilayer2018}\footnote[2]{A part of this chapter is adapted from ~\cite{bilayer2018}~\fullcite{bilayer2018} and its supplemental material, with permission of the publisher. Copyright (2018) by the American Physical Society.}%
%
The first generation of TI--ferromagnet bilayers used bismuth selenide (Bi$_2$Se$_3$) as the TI platform, and EuS~\cite{bilayer2014,Moodera2013} or GdN~\cite{Samarth2013} for the insulating ferromagnet. Relevant to the present study, we previously reported magneto-transport measurements on bilayer samples with europium sulfide (EuS) as the insulating ferromagnet, where a crossover between positive and negative magnetoresistance suggested a proximity effect occurring at the Curie temperature ($T_C$) of EuS~\cite{bilayer2014}. Investigating a similar material system, Wei {\it et al.} further detected a low temperature weak hysteresis as a signature for a developing ferromagnetic phase~\cite{Moodera2013}. Further investigations by this group, using spin-polarized neutron reflectivity experiments, revealed interfacial magnetism that extended $\sim$2 nm into a $\sim$20 nm Bi$_2$Se$_3$ system, which persisted to temperatures much higher than the $T_C$ of EuS itself~\cite{Moodera2016}. While a small increase in $T_C$ of EuS has been reported before, and was attributed to the presence of free bulk carriers~\cite{EuS_ntype, EuS_thin_film_Keller}, the much larger increase in $T_C$ was attributed solely to an interface effect. However, progress in this bilayer material system has been slow, primarily because of Bi$_2$Se$_3$ quality problems such as interstitials and vacancies, which lift the Fermi level to the bulk conduction band, resulting in n-type bulk conductivity~\cite{TI_ARPES1, zhangli2013, Zhanybek3, Fisher2010}, thereby complicate the interpretation of experimental results.

A variety of other 3D-TI materials have been studied in search for an optimal TI platform. In particular, like Bi$_2$Se$_3$, both Bi$_2$Te$_3$ and Sb$_2$Te$_3$ share the same quintuple-layer (QL) crystalline structures with similar lattice constants~\cite{SbStructure, BiStructure}. However, unlike Bi$_2$Se$_3$, the Dirac point of either compound is not well exposed in the bulk band gap~\cite{TI_electronic_structure_zhang}. This was resolved by using the alloy (Bi$_x$Sb$_{1-x}$)$_2$Te$_3$ (BST), introducing a further advantage that electric conduction can be tuned between n-type and p-type by changing the Bi to Sb ratios~\cite{ZhangJS2011}. The realization of QAHE by Cr-doping of BST~\cite{Chang2013, Kou2014}, exhibiting high sample quality and robust magnetism at low temperatures, which persists even when the film thickness is beyond the 2D hybridization limit, suggests that it should also be tried with a bilayer configuration.

In this chapter we will discuss further results on magnetic behavior in the BST--EuS bilayer thin film system. In addition to reproducing similar results as in the Bi$_2$Se$_3$--EuS bilayer system in Chap.~\ref{ch:bilayer2014}, namely a positive to negative magnetoresistance crossover at the Curie temperature of EuS $T_C\approx15~\mathrm{K}$, novel magnetic order was observed at the interface between BST and EuS, which persists to $\sim 60~\mathrm{K}$, much higher than the bulk $T_C$ of EuS. Anomalies in the resistivity and AC Susceptibility suggest a two-stage magnetic proximity induced gap-opening mechanism. In the rest of this chapter, the magnetic and transport properties of four representative samples are reported and compared.

\section{Sample Fabrication and Characterization}
Samples of (Bi$_{x}$Sb$_{1-x}$)$_2$Te$_3$--EuS bilayer thin films were fabricated by pulsed laser deposition (PLD). For the ferromagnet layer, 40~nm EuS thin films were grown on intrinsic Si (100) substrates with native oxide. With the previously reported recipe~\cite{EuS_PLD}, high quality EuS thin films were consistently obtained, characterized by single (100) orientations, atomically smooth surfaces, immeasurably high sheet resistance, and magnetic anisotropy exhibiting an out-of plane component of the magnetizations. To optimize the quality of the interfaces, the TI layer was subsequently grown by PLD \textit{in situ}. Based on existing reports on both Sb$_2$Te$_3$ and Bi$_2$Te$_3$~\cite{telluride_PLD1, telluride_PLD2}, we established a procedure to deposit (Bi$_{x}$Sb$_{1-x}$)$_2$Te$_3$ by alternating the targets of the two compounds~\cite{PLD_alt_target2, PLD_alt_target}. Following each EuS deposition carried out in high vacuum of $\sim10^{-7}~\mathrm{torr}$,  the sample was allowed to cool to $\SI{300}{\degreeCelsius}$ before 200~millitorr of argon gas mixed with 2\% hydrogen was introduced to diffuse the plasma plumes and to prevent potential oxidation from residual gases. Sputtering targets (Kurt J. Lesker, 99.999\%) were ablated $\SI{5}{\cm}$ away from the sample with 25~ns 248~nm KrF excimer laser pulses at 0.55 $\mathrm{J\cdot{}cm^{-2}}$ fluence and 5~Hz repetition rate. The thickness of a $\sim\SI{40}{nm}$ film was measured by atomic force profiliometry and the average deposition rate was $\SI{0.22}{\angstrom}$/pulse. To achieve the optimal composition (5$\%$ bismuth doping), where the Fermi level is inside the bulk band gap and closest to the Dirac point~\cite{ZhangJS2011}, the Bi$_2$Te$_3$ target was ablated by one pulse once per 19 pulses on Sb$_2$Te$_3$. 


Here we focus on two thin and optimally doped samples (S1 \& S2), where the surface state should dominate the electric conduction~\cite{ZhangJS2011}; and, to contrast, two thicker and undoped samples (S3 \& S4), where the Fermi levels intersect the bulk valance band, hence a large contribution of p-type bulk conduction is expected (table.~\ref{tab:bl2018_samples}). 
\begin{table}[ht]
    \centering
    \begin{tabularx}{0.7\columnwidth}[t]{l|l|l|X}
		\hline\hline
        Samples & Ferromagnet & TI Compositions & TI Thicknesses\\
        \hline%
        S1 & EuS (100) & (Bi$_{0.05}$Sb$_{0.95}$)$_2$Te$_3$ & 4~nm\\
        S2 & EuS (100) & (Bi$_{0.05}$Sb$_{0.95}$)$_2$Te$_3$ & 6.5~nm\\
        S3 & EuS (100) & Sb$_2$Te$_3$ & 13~nm\\
        S4 & EuS (100) & Sb$_2$Te$_3$ & 65~nm\\
		\hline\hline
    \end{tabularx}
    \caption[Summary of (Bi$_x$Sb$_{1-x}$)$_2$Te$_3$--EuS bilayer samples presented in Chap.~\ref{ch:bilayer2018}]{\label{tab:bl2018_samples}Summary of samples. S1 \& S2 are thin and optimally doped and therefore should have dominant surface conduction; whereas S3 \& S4 are undoped and thicker therefore should have large contribution from the bulk. Composition and thickness are calculated from numbers of laser pulses.}
\end{table}
The X-ray diffraction (XRD) spectra of these samples (fig.~\ref{fig:bilayer2018_xrd}) indicate clear (001) orientations of the (Bi$_{x}$Sb$_{1-x}$)$_2$Te$_3$ layers. %
%
%
\begin{figure}[ht]%
    \centering%
    \includegraphics[width=0.7\columnwidth]{figs_bilayer2018/material.eps}%
    \subfloat{\label{fig:bilayer2018_xrd}}%
    \subfloat{\label{fig:bilayer2018_mvt}}%
    \subfloat{\label{fig:bilayer2018_mvh}}%
    \caption[X-Ray diffraction spectra and SQUID magnetometry of (Bi$_{x}$Sb$_{1-x}$)$_2$Te$_3$--EuS bilayer samples.]{\label{fig:bilayer2018_material}(a)~Semi-log X-ray diffraction spectra of the four (Bi$_{x}$Sb$_{1-x}$)$_2$Te$_3$--EuS bilayer samples, with thickness of the TI layer increasing from the top to the bottom, compared to a EuS-only thin film of similar thickness. K-$\beta$ spectral contamination exists in the spectrum of sample S4 due to unavailable monochromator. Dashed lines mark the expected positions of the (Bi,Sb)$_2$Te$_3$ [001] peaks~\protect\cite{SbStructure}. Magnetization of sample S2 as functions of (b)~temperatures and (c)~perpendicular magnetic fields (arrows indication field sweep directions). A fitting to the Curie-Weiss law is shown as the black solid curve.}%
\end{figure}%
%
%
%
Their bulk magnetic properties were studied with a superconducting quantum interference device (SQUID) magnetometer. Examples are presented in figs.~\ref{fig:bilayer2018_mvt}~\&~\ref{fig:bilayer2018_mvh} for sample S2. The sample was cooled in zero magnetic field to $T=2~\mathrm{K}$, at which the centering procedure of SQUID was carried out. Subsequently, the field-dependence of the magnetization was measured in an external magnetic field perpendicular to the thin film sweeping between $\mu_0H = +2~\mathrm{T}$ and $\mu_0H = -2~\mathrm{T}$ (fig.~\ref{fig:bilayer2018_mvh}). Finally the external field was reduced to zero from $\mu_0H = +2~\mathrm{T}$, and the temperature dependence of the magnetization was measured during warming up in zero field (fig.~\ref{fig:bilayer2018_mvt}). The component of magnetization perpendicular to the films behaves similarly to EuS thin films without the TI layer~\cite{EuS_PLD}. By fitting to the Curie-Weiss law in the paramagnetic r\'egime, the Curie temperature was determined to be $T_C = 14.5\pm0.3\mathrm{K}$.

\section{AC Magnetic Susceptibility and Electron Transport}
While useful as a bulk measurement, DC magnetometry is less suited to detect weak interface phenomena. In particular, measurements above $T_C$ are especially difficult when background interference dominates the SQUID coil centering process~\cite{squid_center_error}. AC magnetic susceptibility, on the other hand, has been proven to be very sensitive to thermodynamic transitions as well as surface and local phenomena, as demonstrated in studies of 2D ferromagnetism, spin-glass, superparamagnetism, heavy fermions and superconductivity~\cite{ac_nitroxide, ac_spin_glass, ac_superpara, Ando1994, Gegenwart2005, Schemm2014}. To better study the magnetic properties of the interface in a wider temperature range, AC susceptibility of the thin optimally doped sample S1 was measured with a home-made two-coil mutual inductance device~\cite{Jeanneret1989,Yazdani1993} at a drive frequency $f=71~\mathrm{kHz}$. The pick-up coil was wound in a gradiometer configuration and mounted inside the drive coil, both casted into a small epoxy cylinder. One end of the cylinder was then polished to allow the sample to be in close proximity to the top of the two concentric coils (see e.g. ref.~\cite{YazdaniThesis}). The same sample was measured in a van der Pauw configuration for DC and Hall resistance measurements. Indeed, where an unusual behavior of the bilayer system is observed, anomalies appear in both susceptibility and resistance measurements.

A striking example for the correspondence between the zero field AC susceptibility and DC resistance is shown in fig.~\ref{fig:bilayer2018_mirvt}. %
 \begin{figure}[ht]%
	\centering%
    \subfloat{\label{fig:bilayer2018_s1rvt}}%
    \subfloat{\label{fig:bilayer2018_s1mi}}%
    \includegraphics[width=0.7\columnwidth]{figs_bilayer2018/rvtmi.eps}%
    \caption[Coincidental anomalies in the temperature dependence of resistance and of AC magnetic susceptibility of the (Bi$_{x}$Sb$_{1-x}$)$_2$Te$_3$--EuS bilayer sample S1]{\label{fig:bilayer2018_mirvt}Temperature dependence of (a)~sheet resistance, and (b)~AC magnetic susceptibility of sample S1 in zero magnetic field. When an unusual behavior of the bilayer system is observed, anomalies appear in both susceptibility and resistance.}%
\end{figure}%
This 4~nm sample is expected to be very close to the 2D-TI r\'egime where magnetism from the proximitized EuS is expected to have maximum effect. There is a clear effect at the Curie temperature of EuS ($\sim$15 K), where the sheet resistance starts its low-temperature increase, while the the AC susceptibility saturates in magnitude. However, these expected effects are just the last of the magnetic response as we lower the temperature. A dramatic increase in resistance, associated with a cusp in the imaginary part of AC susceptibility, is first observed at 60 K. Lowering the temperature, the sheet resistance seems to almost saturate at $\sim$30 K, at which point the real (inductive) part of the susceptibility shows a dip and the imaginary (dissipative) part almost saturates. Such anomaly seems to be readily suppressed by a small perpendicular magnetic field (fig.~\ref{fig:bilayer2018_mi0.02t0}), which is consistent with a change in the magnetic configuration at the interface, such as that proposed in ref.~\cite{Moodera2016}. %
\begin{figure}[ht]%
    \centering%
    \subfloat{\label{fig:bilayer2018_mi0t0}}%
    \subfloat{\label{fig:bilayer2018_mi0t40}}%
    \subfloat{\label{fig:bilayer2018_mi0.02t0}}%
    \subfloat{\label{fig:bilayer2018_mi0.02t40}}%
    \subfloat{\label{fig:bilayer2018_mi2t0}}%
    \subfloat{\label{fig:bilayer2018_mi2t40}}%
    \includegraphics[width=0.7\columnwidth]{figs_bilayer2018/mi.eps}%
    \caption[AC magnetic susceptibility of sample S1 in external DC fields.]{\label{fig:bilayer2018_mi}~Real ($\chi'$, crosses) and imaginary ($\chi''$, circles) parts of the AC susceptibility of sample S1 as functions of temperatures close to the Curie temperature (dashed lines). (a, b)~In zero field, (c, d)~in 20~mT and (e, f)~in 2~T DC fields perpendicular to the film. The left column shows as-measured data whereas the right column includes $40^{\circ}$ phase rotations. Error bars in all figures in this paper represent the estimated 95~\protect\% confidence intervals.}%
\end{figure}%
In a strong perpendicular DC magnetic field, where the magnetization in the ferromagnetic phase is forced to align with the applied field similarly to ferromagnets measured on their easy axes, the real and imaginary parts of the AC susceptibility should exhibit peaks just above and below $T_C$ respectively~\cite{ac_nitroxide, Venus2004}. However the as-measured data slightly deviate from such expected behavior (fig.~\ref{fig:bilayer2018_mi2t0}). This is likely due to the phase rotation and complex offset introduced by the finite resistance, capacitance and inductance in the wiring of the cryostat and instruments. Indeed the expected behavior is recovered by applying a $40^{\circ}$ phase rotation (fig.~\ref{fig:bilayer2018_mi2t40}). For comparison, the AC susceptibility in zero and 20~mT DC fields are also presented with $40^{\circ}$ phase rotations in the right column in fig.~\ref{fig:bilayer2018_mi} next to their as-measured counterparts. In particular, in 20~mT DC field, where the magnetization is mostly in-plane and the anomaly above $T_C$ is suppressed, the AC susceptibility after phase rotation (fig.~\ref{fig:bilayer2018_mi0.02t40}) also roughly conforms with the expected behavior of a thin film ferromagnet measured on its hard axis~\cite{Jensen2003}.


Transport data are shown in fig.~\ref{fig:bilayer2018_rvt}, summarizing the zero-field sheet resistance (figs.~\ref{fig:bilayer2018_rvt_s1}--\ref{fig:bilayer2018_rvt_s4}) and the Hall resistance (figs.~\ref{fig:bilayer2018_hallp} \& \ref{fig:bilayer2018_halln}) of the four bilayer samples. %
\begin{figure}[ht]%
    \centering%
    \includegraphics[width=0.7\columnwidth]{figs_bilayer2018/rvt.eps}%
    \subfloat{\label{fig:bilayer2018_rvt_s1}}%
    \subfloat{\label{fig:bilayer2018_rvt_s2}}%
    \subfloat{\label{fig:bilayer2018_rvt_s3}}%
    \subfloat{\label{fig:bilayer2018_rvt_s4}}%
    \subfloat{\label{fig:bilayer2018_hallp}}%
    \subfloat{\label{fig:bilayer2018_halln}}%
    \caption[Temperature dependence of resistance and Hall effects in (Bi$_{x}$Sb$_{1-x}$)$_2$Te$_3$--EuS bilayer samples]{\label{fig:bilayer2018_rvt}~Resistive anomalies observed in samples (\protect\subref*{fig:bilayer2018_rvt_s1})~S1 and (\protect\subref*{fig:bilayer2018_rvt_s2})~S2 at the same temperatures where magnetic anomalies occur, but not in those with intrinsic thicker TI layers (\protect\subref*{fig:bilayer2018_rvt_s3})~S3 and (\protect\subref*{fig:bilayer2018_rvt_s4})~S4. (\protect\subref*{fig:bilayer2018_hallp},~\protect\subref*{fig:bilayer2018_halln}) The Hall effect indicates decreasing carrier densities per unit area from S4 to S1 and a shift from p-type to n-type.}%
\end{figure}%
The Hall effect indicates that S2--S4 have holes as the majority carrier, whereas S1 exhibits electron character. A possible reason for a change in majority carrier types between S2 and S1, from holes to electrons, could be the reduction in thickness, hence a stronger influence from the chemical potential of the EuS layer, which has a natural tendency to have electron donors~\cite{EuS_ntype}. While a small elevation of chemical potential may not produce measurable electric conduction in EuS due to its large band gap~\cite{EuS_band_th1, EuS_band_th2}, in the BST layer, however, if the Fermi level is below and very close to the Dirac point~\cite{ZhangJS2011}. where excitations exhibit electron-hole symmetry, even a small elevation may change the majority carrier type. Similarly to S1, a resistive transition was observed in the slightly thicker optimally doped sample S2 (fig.~\ref{fig:bilayer2018_rvt_s2}) near $T\approx30~\mathrm{K}$. Such resistive transitions were neither observed in samples S3 \& S4 (figs.~\ref{fig:bilayer2018_rvt_s3} \& \ref{fig:bilayer2018_rvt_s4}) nor in the Bi$_2$Se$_3$--EuS bilayers in chap.~\ref{ch:bilayer2014} and ref.~\cite{bilayer2014}, where in both cases the ferromagnetism is present but the bulk conduction is more dominant; nor in BST samples near the optimal doping level reported in ref.~\cite{ZhangJS2011}, where the surface conduction dominates but in absence of magnetism. These strongly suggest that the resistive transition observed is a result of proximity between the magnetic order and the surface state. Indeed, the interface magnetization is expected to open a gap at the TI's surface state, hence reduce its contribution to the overall conduction, which would only be observed when the EuS layer is highly insulating and the surface state dominates the conduction in the TI layer.

\section{Negative Magnetoresistance Reexamined}
The magnetoresistance (MR) of the bilayer samples was measured at representative temperatures and presented in fig.~\ref{fig:bilayer2018_mr}. %
%
\begin{figure}[ht]%
    \centering%
    \includegraphics[width=0.7\columnwidth]{figs_bilayer2018/mr.eps}%
    \subfloat{\label{fig:bilayer2018_mr_s1}}%
    \subfloat{\label{fig:bilayer2018_mr_s2}}%
    \subfloat{\label{fig:bilayer2018_mr_s3}}%
    \subfloat{\label{fig:bilayer2018_mr_s4}}%
    \caption[Magnetoresistance of (Bi$_{x}$Sb$_{1-x}$)$_2$Te$_3$--EuS bilayer samples]{\label{fig:bilayer2018_mr}~Magnetoresistance at representative temperatures of (\protect\subref*{fig:bilayer2018_mr_s1})~S1, (\protect\subref*{fig:bilayer2018_mr_s2})~S2, (\protect\subref*{fig:bilayer2018_mr_s3})~S3 and (\protect\subref*{fig:bilayer2018_mr_s4})~S4. (\protect\subref*{fig:bilayer2018_mr_s2}: insert) Low-field behavior of S2 at $T = 12\mathrm{K}$ (triangles) and at $T = 30\mathrm{K}$ (circles), showing reverse temperature dependence. (\protect\subref*{fig:bilayer2018_mr_s4}: insert) Low-field features of S4.}%
\end{figure}%
%
A positive to negative MR crossover at $T_C$ was observed in S1 (fig.~\ref{fig:bilayer2018_mr_s1}), similarly to previously reported behavior of thin Bi$_2$Se$_3$--EuS bilayers~\cite{bilayer2014}. Above $T_C$, a sharp positive MR feature exists near zero field as ubiquitously observed in TI thin films; whereas below $T_C$ a negative MR emerges. In S2 the MR remains positive at all measured temperatures (fig.~\ref{fig:bilayer2018_mr_s2}), however the low-field feature is broader at 12~K than at 30~K (fig.~\ref{fig:bilayer2018_mr_s2}: insert), suggesting a developing negative component, similar to Bi$_2$Se$_3$--EuS bilayers close to $T_C$~\cite{bilayer2014}. In thicker undoped samples S3~\&~S4 (figs.~\ref{fig:bilayer2018_mr_s3}~\&~\ref{fig:bilayer2018_mr_s4}), only positive MR was observed, which sharpens monotonously with decreasing temperature, in addition to parabolic backgrounds that are typically observed in thicker TI films~\cite{TI_WAL_thickness}. While in our previous studies of Bi$_2$Se$_3$-EuS bilayers the Fermi levels were likely well inside the bulk conduction band, and therefore the mechanism of the emergent negative MR remained inconclusive; in the present study, specifically for samples S1 and S2, the doping levels and the Hall effects suggest that the Fermi levels are very close to the Dirac point and well inside the bulk band gap. This case was studied theoretically, suggesting that either gap-opening at the Dirac point~\cite{WL_Glazman, WL_WAL_competition} or coexistence of ferromagnetism and spin-orbit coupling~\cite{WL_ferromagnetism} as the origin for the negative MR. Finally, we note that the crossovers from positive to negative MR have also been observed in bilayer structures with different TIs and ferromagnets~\cite{Samarth2017, Tian2016}, interestingly only when the sheet resistance exceeds the Mott-Ioffe-Regel limit~\cite{Mott_book, Fradkin1986b} in two-dimensions $h/e^2$ (fig.~\ref{fig:bilayer2018_wl_trend}). %
%
\begin{figure}[ht]%
    \centering
    \includegraphics[width=0.7\columnwidth]{figs_bilayer2018/meta.eps}
    \caption[Sheet resistance and signatures of negative magnetoresistance in TI-ferromagnet bilayer samples from literature]{\label{fig:bilayer2018_wl_trend} The maximum sheet resistance of bilayer samples at zero magnetic field from a variety of sources (A: this chapter, B: ref.~\protect\cite{bilayer2014} and unpublished data, C--I: refs.~\protect\cite{Samarth2017, Shi2014, Petta2014, Wang2014, Tian2016, Qiu2017}) The Mott-Ioffe-Regel limit ($R_\Box = h/e^2$) seems to separate samples showing signatures of negative MR below $T_C$ (violet circles), and those only display positive MR (gray crosses).}
%.
\end{figure}%
While most available theories on magneto-transport properties of TI thin films have been formulated in terms of weak localization, we note that, being an orbital quantum interference effect, the concept of weak localization is not easily applicable in such r\'egime.

To summarize, (Bi$_x$Sb$_{1-x}$)$_2$Te$_3$--EuS bilayers were fabricated by pulsed laser deposition. AC magnetic susceptibility displayed anomalies well above the bulk $T_C$ of EuS. Resistive transitions were observed concurrently with magnetic anomalies in thin optimally doped samples, where the Fermi levels are close to the Dirac point, suggesting a gap opened at the Dirac point at the interface. Similarly to previous results, negative magnetoresistance was observed below $T_C$ near zero fields whereas positive magnetoresistance was recovered above $T_C$. Together these suggest a two-stage gap-opening mechanism at the TI surface state Dirac point as result of proximity to an insulating ferromagnet.

\chapter{Conclusions}
\label{ch:conclusions}
     ...
\appendix
\chapter{Summary of Acronyms, Symbols and Chemical Formulae}
    \begin{tabularx}{1\columnwidth}[l]{p{96pt}|X}
\caption{Summary of Acronyms}\\
    \hline\hline
    Acronyms & Full phrases\\
    \hline
    AFM & atomic force microscopy\\
    BCB & bulk conduction band\\
	BST & bismuth doped antimony(III) telluride\\
    BVB & bulk valence band\\
    EE & electron-electron interaction\\
    EMF & electromotive force\\
    FCC & face-centered cubic\\
	HLN & Hikami-Larkin-Nagaoka\\
    IF & insulating ferromagnet\\
    MBE & molecular beam epitaxy\\
    PLD & pulsed laser deposition\\
    QAHE & quantum anomalous Hall effect\\
    QL & atomic quintuple layer\\
	rpm & revolution per minute\\
    RT & room temperature\\
	SEM & scanning electron microscopy\\
    SPS & spark plasma sintering\\
    SQUID & superconducting quantum interference device\\
    TEM & transmission electron microscopy\\
    TA & thermal activation\\
    TI & topological insulator\\
    TRS & time-reversal symmetry\\
    VRH & variable-range hopping\\
    WAL & weak antilocalization effect\\
    WL & weak localization effect\\
    XRD & x-ray diffraction\\
    \hline\hline
\end{tabularx}

\begin{tabularx}{1\columnwidth}[l]{p{96pt}|X}
\caption{Summary of Symbols}\\
    \hline\hline
    Symbols & Physical quantities\\
    \hline
    $e$ & natural base, or the elementary charge\\
    $\vec{E}$ & electric field strength\\
    $g(\vec{k}), g(E)$ & density of states\\
    $h, \hbar$ & Planck constant\\
    $H$ & applied magnetic field strength\\
    $\vec{J}$ & electric current density\\
    $k_B$ & Boltzmann constant\\
    $M$ & magnetization\\
    $n$ & charge carrier density\\
    $p$ & pressure, or momentum\\
    $R_K$ & von Klitzing constant, $R_K = h/e^2 \approx 26~\text{k}\Omega$\\
    $R_{xy}$ & Hall resistance\\
    $R_\Box$ & sheet resistance\\
    $T_C$ & Curie temperature\\
    $T_N$ & N\'eel temperature\\
    $\mu_0$ & vacuum permeability\\
    $\rho$ & electric resistivity\\
    $\sigma$ & electric conductivity\\
    $\sigma_0$ & $e^2/2\pi^2\hbar$\\
    $\sigma_\Box$, $\sigma_{2D}$ & two-dimensional conductivity\\
    $\Phi$ & magnetic flux\\
    $\chi_{AC}$, $\chi'$, $\chi''$ & AC magnetic susceptibility, its real and imaginary parts, respectively\\
    $\omega$ & angular frequency\\
    \hline\hline
\end{tabularx}

\begin{tabularx}{1\columnwidth}[l]{p{96pt}|X}
\caption{Summary of Chemical Formulae}\\
    \hline\hline
    Chemical formulae & Chemical / mineral / common names\\
    \hline
    Al$_2$O$_3$ & aluminum(III) oxide, corundum, sapphire\\
    Bi$_2$Se$_3$ & bismuth(III) selenide\\
    (Bi$_x$Sb$_{1-x}$)$_2$Te$_3$ & bismuth-antimony(III) telluride\\
    EuO & europium(II) oxide\\
    EuS & europium(II) sulfide\\
    EuSe & europium(II) selenide\\
    EuTe & europium(II) telluride\\
    Fe$_3$O$_4$ & iron(II,III) oxide, magnetite\\
    NaCl & sodium chloride, rock salt\\
    Sb$_2$Te$_3$ & antimony(III) telluride\\
    Si & silicon\\
    Y$_3$Fe$_2$(FeO$_4$)$_3$ & yittrium-iron garnet\\
    \hline\hline
\end{tabularx}

\printbibliography
\end{document}