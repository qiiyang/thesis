The weak localization and the weak anti-localization effects are quantum interference effects that lead to electric conductivity deviating from classical values predicted by the Drude model. In the context of this thesis, we will concentrate on the two-dimensional case. In a two-dimensional medium where finite disorder is present, consider electron scattering paths around a loop back to the same point that eventually invert the direction of travel: for every possible scattering path there exists a path that is counter-propagating, but otherwise identical (fig.~\ref{fig:bg_scattering}a).%
\begin{figure}[ht]%
    \centering%
    \includegraphics[width=0.75\columnwidth]{figs_misc/wl_scattering}%
    \caption[Illustrations of electron scattering paths in two-dimensions.]{\label{fig:bg_scattering}Illustration of electron scattering paths in a two-dimensional medium where finite disorder is present. (a)~For every scattering path there is a counter propagating, but otherwise identical, counterpart. (b) Each path in the real space involves a 180\degree{} rotation of the momentum, with the two paths rotate towards the opposite directions.}%
\end{figure}%

In absence of spin-orbit interaction, the two paths introduce exactly the same phase change to the incidental electron wave function and therefore interfere constructively, hence the probability for the electron to travel backwards is increased. The electric resistance increases as consequence of the constructive interference of all such pairs of paths. This is referred to as the weak localization effect. A magnetic field perpendicular to such two-dimensional medium introduces a phase difference $\Delta\psi = 2\cdot\frac{eBA}{\hbar}$ between a pair of counter-propagating loop enclosing an area $A$. Since the loops formed by different pairs of paths have different sizes, a magnetic field through the loops introduces different phase differences in different pairs.