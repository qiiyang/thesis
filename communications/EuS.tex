\documentclass[11pt]{article}
\linespread{1.2}
\usepackage{enumitem}


\usepackage{bm}        % for math
\usepackage{amssymb}   % for math
\usepackage{amsmath}   % for math
\usepackage{physics}

\newcommand*\diff{\mathop{}\!\mathrm{d}}    % d for derivative
\renewcommand{\vec}[1]{\vectorbold*{#1}} % vectors
\renewcommand{\op}[1]{\vectorbold*{\hat#1}} % operators

\usepackage[left=2cm, right=2cm, top=2cm, bottom=2cm]{geometry}
% For Bibliography
\usepackage[american]{babel}
\usepackage{csquotes}
\usepackage[style=phys, citestyle=phys,%
    articletitle=false, biblabel=brackets, chaptertitle=false, pageranges=false, %APS style
    backend=biber, bibencoding=ascii, doi=false, url=false, eprint=false]{biblatex}
\addbibresource{../bibs/bib_EuS.bib}

\usepackage{color}
\usepackage[colorlinks=true, breaklinks=true, bookmarks=false, linkcolor=blue, urlcolor=blue, citecolor=blue,]{hyperref}

\usepackage{graphicx}  % needed for figures
\usepackage{caption} %
    
\begin{document}
\title{Re: EuS Magnetic Properties}
\author{Qi Yang}
\maketitle
\tableofcontents

\section{Curie Temperatures}
Indeed the external field in figure~4.4c in the thesis was erroneously stated, the actual field was $H = 200~\mathrm{Oe}$. The caption has been corrected.

While a Curie-Weiss law fitting is not the most accurate way to determine the true Curie temperature, it was used as a matter of consistency. The estimated $T_C$ serves as a proxy to compare with the expected conductivities of other reported thin film samples in thesis table~4.1; and most of the references estimated the $T_C$ in the paramagnetic r\'egime.

The inverse magnetization plots for thesis figures~4.4c and 5.2b are shown in figure~\ref{fig:squid}.%
%
\begin{figure}[ht]%
    \centering%
    \includegraphics[width=0.50\columnwidth]{squid_11}%
    \includegraphics[width=0.50\columnwidth]{squid_04}%
    \caption[]{\label{fig:squid}Inverse of magnetization plots for thesis figures~4.4c (left) and 5.2b (right), including a few data points above the temperature when the centering and fitting mechanism of the MPMS start to become a problem.}%
\end{figure} %
The centering of the SQUID coil for each measurement was carried out in the ferromagnetic r\'egime. Since the EuS thin film stops being the dominant magnetic dipole compared to the surrounding material at $T \gtrsim T_C$, the center shifts (confirmed by watching the SQUID scan curves during the measurement), and there was always an artefact in the form of a jump in the measured magnetic moment near and above $T_C$. Above such jump, the SQUID seemed to be measuring a signal off the original center. Therefore indeed the Curie-Weiss fitting was carried out in a relatively narrow temperature range (a few Kelvins) above the transition and below such jumps in signals.

While such fittings are not ideal, the $T_C$'s estimated were consistent with the polar Kerr effect measurements (thesis fig.~5.3b), and AC susceptibility measurements (thesis fig.~6.4f) up to $\pm1$~K.

\section{Shape Anisotropy}
One possible reason why the EuS thin films show finite perpendicular components of magnetization is due to its relatively large coercive field. Here an order-of-magnitude argument is outlined.

Consider a disk of radius $r_0$ and thickness $t=2r_0$: roughly speaking, such disk is similar to a sphere, therefore does not have significant shape anisotropy. Now extend the disk to a whole thin film, and take the mean field approximation, such that the thin film has an uniform magnetization of $\vec{M}$. For a magnetic dipole moment $\vec{m}$ at the center of the disk, due to dipole-dipole interaction, the difference in energy between an in-plane ferromagnetism and an out-of-plane one is imposed by the part of the thin film that is outside the disk:
\begin{align}
    \Delta U &= U_\parallel - U_\perp\\
        &\approx -\frac{3\mu_0}{4\pi}\int_V\frac{1}{|\vec{r}|^3}(\vec{m}\cdot\op{r})(\vec{M}\cdot\op{r})\diff \vec{r}^3\\
        &= -\frac{3\mu_0}{4\pi}\int_{r_0}^\infty\int_0^{2\pi}\frac{2r_0mM}{r^3}\cos^2\theta~r\diff r\diff\theta\\
        &= -\frac{3\mu_0mMr_0}{2\pi}\int_{r_0}^\infty r^{-2}\diff r\int_0^{2\pi} \frac{1+\cos2\theta}{2}\diff\theta\\
        &= -\frac{3\mu_0mM}{2}\\
        &\sim -\mu_0mM\label{eq:ani}~.
\end{align}

The energy cost to rotate such magnetic moment from being perpendicular to the film to being parallel is roughly:
\begin{equation}
    W \approx \mu_0mH_c~,\label{eq:work}
\end{equation}
where $H_c$ is the coercive field.

If I take the magnetization of single-crystalline EuS at $T=1.4\mathrm{K}$ in ref.~\cite{EuS_Shafer}, $M = 210~\mathrm{Oe}$, and read the coercive field at the perpendicular direction from figure~4.4a in the thesis, $H_c \approx 500~\mathrm{Oe}$, then the energy scales in equations \ref{eq:ani} and \ref{eq:work} are on the same order of magnitude. Hence it is a possibility that the hardness in the perpendicular magnetization prevents the local moments from tilting to parallel by the shape anisotropy.

\printbibliography
\end{document}