A Josephson's junction is two pieces of superconductors separated by a thin barrier that is either an insulator or a normal conductor (figure~\ref{fig:josephson}). %
\begin{figure}[ht]%
	\centering%
    \includegraphics[width=0.55\columnwidth]{figs_misc/josephson}%
    \caption[Josephson's junction]{\label{fig:josephson}Schematic illustration of a Josephson's junction.}%
\end{figure}%
%
When the barrier is sufficiently thin, the supercurrent tunnels through the barrier, thereby the junction exhibits a constant current without applied voltage. This is referred to as the Josephson effect. The voltage across the junction, $V$, the current tunneling through the barrier, $I$, are related to the the difference between the phases of the Ginzburg-Landau order parameters on the two sides of the barrier $\Delta\theta$~\cite{Josephson1962, Josephson1974}:%
\begin{align}
    V(t) &= \frac{\hbar}{2e}\frac{\diff \Delta\theta(t)}{\diff t}~,\\
    I(t) &= I_c\sin\left(\Delta\theta(0) + \frac{2eV}{\hbar}t\right)~.
\end{align}

A Superconducting QUantum Interference Device (SQUID) is two Josephson's junctions connected by a superconductor in parallel (figure~\ref{fig:squid}). %
\begin{figure}[ht]%
	\centering%
    \includegraphics[width=0.65\columnwidth]{figs_misc/squid}%
    \caption[Superconducting quantum interference device]{\label{fig:squid}Schematic illustration of a superconducting quantum interference device (SQUID).}%
\end{figure}%