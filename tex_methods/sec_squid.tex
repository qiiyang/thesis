A Josephson's junction is two pieces of superconductors separated by a thin barrier that is either an insulator or a normal conductor (figure~\ref{fig:josephson}). %
\begin{figure}[ht]%
	\centering%
    \includegraphics[width=0.55\columnwidth]{figs_misc/josephson}%
    \caption[Josephson's junction]{\label{fig:josephson}(Color online) Schematic illustration of a Josephson's junction. Two pieces of superconductors are separated by a thin barrier that is either an insulator or a normal conductor. When the barrier is sufficiently thin, the supercurrent tunnels through the barrier.}%
\end{figure}%
%
When the barrier is sufficiently thin, the supercurrent tunnels through the barrier, thereby the junction can sustain a constant current without applied voltage. This is referred to as the Josephson effect. The voltage across the junction, $V$, the supercurrent tunneling through the barrier, $i$, are related to the the difference between the phases of the Ginzburg-Landau order parameters on the two sides of the barrier $\Delta\theta$~\cite{Josephson1962, Josephson1974}:%
\begin{align}
    V(t) &= \frac{\hbar}{2e}\frac{\diff \Delta\theta(t)}{\diff t}~,\\
    i(t) &= i_c\sin\left(\Delta\theta(t)\right)~.\label{eq:Josephson_I}
\end{align}%
The current constant $i_c$ is referred to as the critical current, namely the maximal current can pass through the junction without an applied voltage.

A Superconducting QUantum Interference Device (SQUID) is two Josephson's junctions connected by a superconductor in parallel (figure~\ref{fig:squid}). %
\begin{figure}[ht]%
	\centering%
    \includegraphics[width=0.65\columnwidth]{figs_misc/squid}%
    \caption[Superconducting quantum interference device]{\label{fig:squid}(Color online) Schematic illustration of a superconducting quantum interference device (SQUID). An ideal SQUID functions as a Fraunhoffer interferometer measuring the external flux through the loop.}%
\end{figure}%
%
In an ideal SQUID, the two Josephson's junctions are identical, the geometry is symmetric, and the area enclosed by the loop is small hence the electromagnetically induced screening current is negligible. An external magnetic flux $\Phi$ through the SQUID loop introduces a difference between the phase differences of the two junctions~\cite{Annett}:%
\begin{align}
    \Delta\theta_1 - \Delta\theta_2 &= \frac{2e}{\hbar}\Phi\nonumber\\
        &= 2\pi\frac{\Phi}{\Phi_0}~,\label{eq:SQUID}
\end{align}%
where $\Phi_0 = \frac{h}{2e}$ is the magnetic flux quantum. Equations~\ref{eq:Josephson_I}~\&~\ref{eq:SQUID} together illustrate that the combined supercurrent $I$ as a function of the external magnetic flux $\Phi$ is a Fraunhoffer interference pattern formed by the superposition of the supercurrents through either junctions. Let $\Delta\bar{\theta}(t) = \left[\Delta\theta_1(t) + \Delta\theta_2(t)\right] / 2$. The total supercurrent through the SQUID is therefore%
\begin{equation}
    I(t) = i_1(t) + i_2(t) = 2i_c\cos\left(\frac{\pi\Phi}{\Phi_0}\right)\sin\left(\Delta\bar{\theta}(t)\right)~.
\end{equation}%
The critical current, namely the maximal supercurrent through the SQUID, is therefore an oscillating function of $\Phi$ with a period of $\Phi_0$:%
\begin{equation}
    I_{c}(t) = 2i_c\left|\cos\left(\frac{\pi\Phi}{\Phi_0}\right)\right|~.
\end{equation}

In practice, a Josephson's junction may have finite shunt resistance $R$ and capacitance $C$, depending on the dimensionless parameter $\beta_c = 2\pi{}R^2i_cC$, it consequently may or may not exhibits hysteresis in its current-voltage characteristics~\cite{Hansma1971}. Typically, a SQUID used for magnetometry measurements is constructed with non-hysteric junctions, with $\beta_c \lesssim 1$. A constant bias current $I > I_{c}$ is applied, 