Pulsed laser deposition (PLD) is a thin film growth method that utilizes high energy laser photons. A pulsed laser beam is focused and transmitted through a window into a high vacuum chamber. The laser beam ablates a solid piece of source material (the target), generating a plume of energized species that is typically referred to as the ``plasma plume''. The plasma plume solidifies on a heated substrate a distance away, forming the desired thin film sample (figure~\ref{fig:PLD_schematic}). %
\begin{figure}[ht]%
	\centering%
    \includegraphics[width=0.75\columnwidth]{figs_methods/PLD_schematic.png}%
    \caption[Schematic Illustration of a PLD apparatus]{\label{fig:PLD_schematic}(Color online) A schematic illustration of a pulsed laser deposition apparatus.\footnotemark}%
\end{figure}%
\footnotetext{Figure~\ref{fig:PLD_schematic} is licensed from Wikimedia Commons under Creative Commons CC0 1.0 Universal Public Domain Dedication.}%
%
\citeauthor{PLD_review} reviewed in ref.~\cite{PLD_review} major advantages and drawbacks of PLD compared to other deposition methods. The advantages most relevant to this thesis include: (i)~under optimal growth conditions, the stoichiometry of the target is preserved in the resultant thin films, even for complex compounds; (ii)~the ablation of target is pulsed and local, therefore preserves the pristine state of the rest of the target; (iii)~the high kinetic energy of the ablated species facilitate surface diffusion during film growth; (iv)~the laser is outside the vacuum chamber, which allows more flexibility in material choice and apparatus geometry. All thin film samples presented in this thesis were grown with PLD, particularly for its effectiveness to grow thin films from pre-made bulk compounds.

While a PLD apparatus is relatively simple, the physical processes involved in PLD thin film growth are typically complex. Table~\ref{tab:pld_params} shows the typical set of parameters characterizing a PLD procedure. %
%
\begin{table}[ht]%
    \centering%
    \begin{tabularx}{0.90\columnwidth}[t]{l|l|X}
    \caption[Pulsed laser deposition parameters.]{\label{tab:pld_params}Summary of pulsed laser deposition parameters, and their typical values in experiments presented in this thesis. The first two parameters are determined by the laser hardware, whereas the rest are subject to optimization.}\\
		\hline\hline
        Parameters & Values & Comments\\
        \hline%
        photon wavelength & 248~nm & determined by the laser hardware\\
		pulse width & 25~ns & determined by the laser hardware\\
		repetition rate & 5--10~Hz &\\
		fluence & 0.5--1.0~J~/~cm$^2$ & energy density per pulse\\
		background gas & vacuum, or Ar with H$_2$ &\\
		pressure & 10$^{-7}$~Torr~--~0.2~Torr &\\
        target-sample distance & 5~cm &\\
        target properties & & material and preparation technique\\
		\hline\hline
    \end{tabularx}%
\end{table}%
Because of the photon energy of the laser, and the time scales involved in the ablation, the plume's propagation, and the nucleation, the physical and chemical processes in PLD are often non-equilibrium and involve coexistence of multiple phases. While the physics and novel techniques involved in PLD is still a subject of ongoing research and a comprehensive discussion is outside the scope of this thesis, key characters and complications of PLD's various components and processes are outlined in this section, whereas detailed treatise and review can be found in refs.~\cite{PLD_book, PLD_review}.

The typical wavelength used for PLD, achieved by either Nd$^{3+}$:YAG or excimer lasers, is ultraviolet and ranges between \SI{150}{nm} and \SI{400}{nm}. The work documented in this thesis was done with a KrF excimer laser, which has the highest gain among the available choices. Its \SI{248}{nm} wavelength corresponds to a photon energy of \SI{5}{eV}, similar to the first ionization energy of alkaline metals (\SI{5.1}{eV} for Na $\to$ Na$^+$), equivalent to a temperature $T=6\times10^5~\mathrm{K}$. The laser pulses have \SI{25}{ns} width and are typically generated at repetition rates on the order of \SI{10}{Hz}. Possible ablation mechanisms resulted from the interaction between the laser photons and the target surface include well understood phenomena, such as collisional sputtering, evaporation, electronic sputtering, exfoliation, and hydrodynamic formation of droplets, as well as less understood secondary phenomena documented in sec.~3.3 of ref.~\cite{PLD_book}. The ablation of the target by a laser pulse generates a plasma plume, and modifies the target surface for all the subsequent pulses that irradiate on the same area.

The plasma plume is typically constituted of both charge-neutral vapor and charged plasma, often exhibiting a visible color. The plasma temperatures measured by emission spectroscopy during the initial expansion are typically on the order of $\sim$\SI{10000}{K}, considerably higher than the boiling temperature of the typical source materials. Such phenomena is currently attributed to the inverse-Bremsstrahlung absorption of the laser photons. Additionally to vapor and plasma, common but less understood components of the plasma plume are liquid droplets or solid pieces. They are either ejected from the target due to partial evaporation, or condensed from vapor due to supersaturation. The former origin typically results in droplets or pieces with sizes in the micron or sub-micron ranges, whereas the latter yields sizes in the nanometer range. Such droplets and pieces form particulates in the resultant thin films, generally detrimental to sample quality.

A background gas environment may have multiple effects on the propagation of the plasma plume and the resultant thin film. Firstly, the presence of a background gas diffuses the plume, and widen its angular distribution. Secondly, the gas molecules thermalize and cool the plume through collisions. Lastly, chemical reactions may occur and be utilized to either compensate stoichiometric distortion due to ablation, or to reduce undesired chemical reactions (e.g., using H$_2$ to reduce oxidations).

Thin film nucleation and growth modes are often categorized into three categories~\cite{Greene1993}: (i)~accumulation of three-dimensional islands (Volmer--Weber), (ii)~two dimensional full-monolayer growth (Frank--van der Merwe), and (iii)~monolayers followed by islands (Stranski--Krastinov). In general, an atom or ion landed on the surface of the sample has finite probabilities to either re-evaporate or become adsorbed to the surface as a temporary mobile atom (known as an adatom). In particular for PLD, the plasma plume is often saturated or supersaturated, partially ionized, and constituted with species with high kinetic energy. Such high kinetic energy increases surface mobility of adatoms, hence increases the chance that they may reach their thermodynamically stable sites. \citeauthor{Metev1989} suggested that the energetic nature of the plasma plume favors single-atom nucleation centers and the two-dimensional full-monolayer (Frank--van der Merwe) growth mode or an continuous growth mode~\cite{Metev1989}. See ref.~\cite{PLD_book}, sec.~8.2, for a detailed discussion of the influence of deposition parameters on the various growth modes.

In practice, the optimization of a PLD recipe is typically highly heuristic. The characteristics of the processes outlined above, together with phase diagrams of the material to grow, may sometimes provide intuitions to form the initial guess on the parameter. However, due to the complexity of these processes, and the theoretical nature of the research from which the understanding of these processes is drawn, such understanding is not sufficient to accurately guide the optimization. To fabricate the thin film samples presented later in this thesis, existing recipes from literature were used as baselines, whereas the parameters were varied to optimize the sample qualities.