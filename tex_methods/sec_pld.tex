Pulsed laser deposition (PLD) is a thin film growth method that utilizes high energy laser photons. A pulsed laser beam is focused and transmitted through a window into a high vacuum chamber. The laser beam ablates a target of the source material, generating a so called ``plasma plume''. The ``plasma plume'' re-solidifies on a heated substrate a distance away, forming the desired thin film sample (fig.~\ref{fig:PLD_schematic}).%
\begin{figure}[ht]%
	\centering%
    \includegraphics[width=0.618\columnwidth]{figs_methods/PLD_schematic.png}%
    \caption[Schematic Illustration of a PLD apparatus]{\label{fig:PLD_schematic}A schematic illustration of a pulsed laser deposition apparatus, licensed from Wikimedia Commons under Creative Commons CC0 1.0 Universal Public Domain Dedication.}%
\end{figure}%

While a PLD apparatus is relatively simple, the physical processes involved in PLD thin film growth are complex. A typical KrF Excimer pulsed laser used for PLD 