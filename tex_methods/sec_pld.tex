Pulsed laser deposition (PLD) is a thin film growth method that utilizes high energy laser photons. A pulsed laser beam is focused and transmitted through a window into a high vacuum chamber. The laser beam ablates a solid piece of source material (the target), generating a plume of energized species that is typically referred to as the ``plasma plume''. The plasma plume solidifies on a heated substrate a distance away, forming the desired thin film sample (fig.~\ref{fig:PLD_schematic}). %
\begin{figure}[ht]%
	\centering%
    \includegraphics[width=0.618\columnwidth]{figs_methods/PLD_schematic.png}%
    \caption[Schematic Illustration of a PLD apparatus]{\label{fig:PLD_schematic}A schematic illustration of a pulsed laser deposition apparatus, licensed from Wikimedia Commons under Creative Commons CC0 1.0 Universal Public Domain Dedication.}%
\end{figure}%
While a PLD apparatus is relatively simple, the physical processes involved in PLD thin film growth are typically complex. Because of the photon energy of the laser, and the time scales involved in the ablation, the plume's propagation, and the nucleation, each of these stages often involves non-equilibrium processes and coexistence of multiple phases, depending on both the deposition parameters and the source material. While the entirety of physics and novel techniques involved in PLD is still a subject of ongoing research and outside the scope of this thesis, key characters and complications of PLD's various components and processes are outlined in this section, whereas detailed treatise and review can be found in refs.~\cite{PLD_book, PLD_review}.

The typical wavelength used for PLD, achieved by either Nd$^{3+}$:YAG or excimer lasers, is ultraviolet and ranges between \SI{150}{nm} and \SI{400}{nm}. The work documented in this thesis was done with a KrF excimer laser, which has the highest gain among the available choices. Its \SI{248}{nm} wavelength corresponds to a photon energy of \SI{5}{eV}, similar to the first ionization energy of alkaline metals (\SI{5.1}{eV} for Na $\to$ Na$^+$), and equivalent to a temperature $T~=~6 \times~10^5~\mathrm{K}$. The laser pulses have \SI{25}{ns} width and are typically generated at repetition rates on the order of \SI{10}{Hz}. Possible ablation mechanisms resulted from the interaction between the laser photons and the target surface include well understood phenomena, such as collisional sputtering, evaporation, electronic sputtering, exfoliation, and hydrodynamic formation of droplets, as well as less understood secondary phenomena documented in sec.~3.3 of ref.~\cite{PLD_book}.

The ``plasma plume'' is typically constituted of both charge-neutral vapor and charged plasma, often exhibiting a visible color. The plasma temperatures measured by emission spectroscopy during the initial expansion are typically on the order of $\sim$\SI{10000}{K}, considerably higher than the boiling temperature of the source material. Such phenomena is currently attributed to the inverse Bremsstrahlung absorption of the laser photons.

%In addition to producing the plasma plumes, the ablation resulted from a laser pulse also modifies the target surface for all the subsequent pulses that irradiate on the same area.

