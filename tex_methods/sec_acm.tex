In a constant external magnetic field, $\vec{H}$, the magnetization, $\vec{M}$, in a material is determined by its DC magnetic susceptibility, $\chi$:
\begin{equation}%
    \vec{M} = \chi\vec{H}~.%
\end{equation}%
%
In general, since $\vec{M}$ may have different direction from $\vec{H}$, $\chi$ is a tensor. Within the scope of this thesis, however, only the component of $\vec{M}$ that is parallel to $\vec{H}$ is measured, therefore $\vec{H}$, $\vec{M}$, and $\chi$ are effectively all treated as scalars:
\begin{equation}%
    M = \chi{}H~.%
\end{equation}%
%
The DC magnetic susceptibility of a sample, therefore, can simply be calculated by the measured magnetic moment, \textit{e.g.}, with a SQUID magnetometer, and the applied magnetic field strength.

When the applied magnetic field strength contains a constant component, $\overline{H}$, and a sinusoidal component, $\widetilde{H}(t)=\left|\widetilde{H}\right|e^{i\omega{}t}$, the magnetization generally also displays an oscillating component in response: $M(t) = \overline{M} + \widetilde{M}(t)$. Similarly to the DC magnetic susceptibility, the oscillating components of the the magnetization and of the applied field strength are in turn related by the AC magnetic susceptibility $\chi_{AC}$:
\begin{equation}%
    \widetilde{M}(t) = \chi_{AC}\widetilde{H}(t)~.%
\end{equation}%

When the oscillating field $\widetilde{H}(t)$ has a small amplitude and a low frequency, the AC susceptibility takes real values, and measures the local derivative of $\overline{M}$ with respect to $\overline{H}$:%
\begin{equation}%
    \lim\limits_{\omega \to 0}\chi_{AC} = \frac{\diff\overline{M}}{\diff\overline{H}}~.%
\end{equation}%
%
On the other hand, at higher frequencies, the dynamic properties of the sample plays an important role, and the magnetization may respond to the applied field in a partially out-of-sync manner. In such case, the AC susceptibility takes a complex form:%
\begin{equation}%
    \chi_{AC} = \chi' + i\chi''~,%
\end{equation}%
where $\chi'$ and $\chi''$ take real values. These are often referred to as the dynamic magnetic susceptibility.

Consider the work done to the local magnetization by the applied external magnetic field, the instantaneous power being $P = \mu_0H\frac{\diff M}{\diff t}$, the complex power per unit volume is therefore%
\begin{align}%
    \widetilde{P} &= \frac{1}{2}\mu_0 \widetilde{H}^* \frac{\diff \widetilde{M}}{\diff t}\nonumber\\
        &= \frac{1}{2}\mu_0\chi_{AC}\widetilde{H}^*\frac{\diff \widetilde{H}}{\diff t}\nonumber\\
        &= \frac{1}{2}i\omega\mu_0\chi_{AC}\left|\widetilde{H}\right|^2\nonumber\\
        &= \frac{1}{2}\omega\mu_0\left|\widetilde{H}\right|^2(-\chi''+i\chi')\label{eq:acm_power}~.
\end{align}%
The real part of $\widetilde{P}$, corresponding to dissipation, is determined by $\chi''$; whereas the imaginary part of $\widetilde{P}$, corresponding to the reflection of electromagnetic waves, is detwemined by $\chi'$.

A home-made mutual inductance device (figure~\ref{fig:mi_sketch}) based on refs.~\cite{Jeanneret1989} was used to measure the AC susceptibility in thin film samples.\footnote{The mutual inductance device was fabricated by Alan Fang.} %
\begin{figure}[ht]%
	\centering%
    \includegraphics[width=0.65\columnwidth]{figs_methods/mi}%
    \caption[Mutual inductance device for AC susceptibility measurements]{\label{fig:mi_sketch}Schematic illustration of the mutual inductance device for AC susceptibility measurements. An AC current is applied through the drive coil, resulting in an AC applied magnetic field. The pickup coil is wound in a gradiometer configuration, therefore insensitive to the applied field. Since the magnetization in the sample produces an uneven AC magnetic flux outside the sample, it induces an electromotive force in the pickup coil.}%
\end{figure}%
%
A solenoid $\sim$\SI{1.5}{mm} in diameter is used as the drive coil. A pickup coil is wound in a gradiometer configuration and placed concentric inside the drive coil. Both coils are wound with copper wires, which exhibit negligible electric resistance at low temperatures. To provide mechanical integrity, Stycast 2850-FT epoxy was applied on the coils and molded into a cylinder. One end of the cylinder was then polished to allow the sample to be in close proximity to the top of the two concentric coils (also see ref.~\cite{YazdaniThesis}).

An AC current is passed through the drive coil to produce an AC applied magnetic field. Since the magnetic field is homogeneous within the drive coil, it does not generate a signal in the pickup coil. The magnetization in the sample, on the other hand, produces an uneven AC magnetic flux outside the sample, therefore induces an electromotive force (EMF) in the pickup coil.

The applied magnetic field strength at the end of the drive coil is proportional to the AC current in the drive coil:%
\begin{equation}%
    \widetilde{H} = \frac{1}{2}N_d\widetilde{I}~,\label{eq:mi_i}
\end{equation}%
where $N_d$ is the number of turns per unit length in the drive coil. The induced EMF in the pickup coil, namely the measured voltage, is related to the distribution of magnetic flux generated by the sample, hence in turn related to the magnetization in the sample. Approximate to the first order:%
\begin{align}%
    \widetilde{V} &\approx N_p\left(\frac{l}{2}\right)^2 \frac{\diff^2 \Phi}{\diff z \diff t}\nonumber\\
        &= \frac{1}{4}l^2N_p\alpha \frac{\diff M}{\diff t}\nonumber\\
        &= \frac{1}{4}i\omega l^2 N_p \alpha \widetilde{M}~,\label{eq:mi_v}
\end{align}%
where $N_p$ is the number of turns per unit length in the pickup coil, $l$ is the length of the pickup coil, $z$ is the spatial coordinate along the length, and $\alpha$ is a real constant related to the sample geometry and its distance to the coils. Combining equations \ref{eq:mi_i} \& \ref{eq:mi_v}, the impedance due to the mutual inductance in the coils is therefore%
\begin{align}%
    \widetilde{Z} &= \frac{\widetilde{V}}{\widetilde{I}}\nonumber\\
        &= \frac{1}{8}i\omega l^2 N_d N_p \alpha \frac{\widetilde{M}}{\widetilde{H}}\nonumber\\
        &= i\omega\beta\chi_{AC}\nonumber\\
        &= -\omega\beta\chi'' + i\omega\beta\chi'\label{eq:mi_z}~,
\end{align}%
where $\beta$ is a real constant related to the geometry both of the sample and of the coils. The real and imaginary parts of the impedance are, therefore, proportional to the imaginary and real parts of the AC susceptibility, respectively. Furthermore, the sensitivity of the mutual inductance measurement should beproportional to the frequency of the driving current.

Ideally, the driving coil and the pickup coil are connected to room-temperature electronics each by a coaxial cable, which exhibits finite series resistance, series self-inductance and shunt capacitance, but negligible mutual inductance. These characteristics impose a complex multiplier on each of the applied current and the measured voltage, therefore affect both the amplitude and the phase of measured mutual inductance:%
\begin{align}%
    \widetilde{Z}_{measured} &= Ae^{i\phi}\widetilde{Z}\nonumber\\
        &= \omega\beta'e^{i\phi'}\chi_{AC}~.\label{eq:mi_coax}
\end{align}%
The amplitude and phase, $\beta'$ and $\phi'$, relating the measured mutual inductance and the AC susceptibility can be determined by measuring a sample with well known AC susceptibility (e.g., a superconductor)~\cite{Jeanneret1989, HahnThesis}. Thus a mutual inductance device can be calibrated, to adjust for each cryostat and each combination of coaxial cables, to measure precisely the AC susceptibility of samples.

In practice, for the AC susceptibility measurements discussed in this thesis, the mutual inductance device was installed in a Quantum Design PPMS system, in which the coils were connected to room-temperature electronics by twisted wires~\cite{ppms_ac}. Twisted wires, compared to coaxial cables, however, also has finite mutual inductance between the current wires and the voltage wires. Therefore the wire characteristics not only impose a complex multiplier to the measured mutual inductance, but also (approximating to the first order) a complex offset:%
\begin{equation}%
    \widetilde{Z}_{measured} \approx \omega\beta'e^{i\phi'}\chi_{AC} + \widetilde{Z}_0~.\label{eq:mi_twisted}
\end{equation}%

Similarly to SQUID magnetometry, a mutual inductance measurement not only measures the AC susceptibility of the sample, but also the surrounding materials. These include the epoxy enclosing the coils, the structure that holds together the sample and the mutual inductance device, and, in the case of thin film samples, the substrate on which the sample was grown. Therefore, with or without calibration, the exact AC susceptibility of the sample can be difficult to separate from the background, unless its contribution is much greater than the surrounding materials, such as in the cases of superconducting or ferromagnetic samples.

When surface and interface effects are the subjects of investigation, due to the small volumes in which these effects occur, their contribution to the mutual inductance signals might not be much greater than the surrounding materials, therefore might not be accurately determined. Nonetheless, if these effects manifest as sharp changes in the temperature-dependence or field-dependence measurements, such sharp changes can be distinguished from the background signals. While the electric and magnetic properties of the wiring and the surrounding materials may change with the temperature or the applied magnetic field, these changes are gradual, since the materials used (typically copper and Stycast epoxies) do not exhibit phase transitions at low temperatures. Indeed, AC magnetic susceptibility measurements have been proven to be very sensitive to thermodynamic transitions as well as surface and local phenomena. Such utility has been demonstrated in a variety of material systems, including 2D ferromagnetism, spin-glass, superparamagnetism, heavy fermions, and superconductivity~\cite{ac_nitroxide, ac_spin_glass, ac_superpara, Ando1994, Gegenwart2005, Schemm2014}.

Such sensitivity to thermodynamic transitions, and surface, interface or local phenomena, is the primary reason for AC susceptibility measurements to be employed in addition to SQUID magnetometry in our work. SQUID magnetometry, as discussed in section~\ref{sec:squid}, assumes a single magnetic dipole moment in the data-fitting process, therefore requires the effects under investigation either exhibiting signals dominating over the surrounding materials, or occurring in close proximity to the dominant magnetic signal in the surrounding materials. Such requirement may not always be satisfied for surface and interface effects. The AC susceptibility measurements, on the other hand, do not have such single-dipole restriction, therefore well complement the SQUID measurements.