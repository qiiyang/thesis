\footnote[2]{A part of this chapter is adapted from ~\cite{bilayer2018}~\fullcite{bilayer2018} and its supplemental material, with permission of the publisher. Copyright (2018) by the American Physical Society.}%
%
The first generation of TI--ferromagnet bilayers used bismuth selenide (Bi$_2$Se$_3$) as the TI platform, and EuS~\cite{bilayer2014,Moodera2013} or GdN~\cite{Samarth2013} for the insulating ferromagnet. Relevant to the present study, we previously reported magneto-transport measurements on bilayer samples with europium sulfide (EuS) as the insulating ferromagnet, where a crossover between positive and negative magnetoresistance suggested a proximity effect occurring at the Curie temperature ($T_C$) of EuS~\cite{bilayer2014}. Investigating a similar material system, Wei {\it et al.} further detected a low temperature weak hysteresis as a signature for a developing ferromagnetic phase~\cite{Moodera2013}. Further investigations by this group, using spin-polarized neutron reflectivity experiments, revealed interfacial magnetism that extended $\sim$2 nm into a $\sim$20 nm Bi$_2$Se$_3$ system, which persisted to temperatures much higher than the $T_C$ of EuS itself~\cite{Moodera2016}. While a small increase in $T_C$ of EuS has been reported before, and was attributed to the presence of free bulk carriers~\cite{EuS_ntype, EuS_thin_film_Keller}, the much larger increase in $T_C$ was attributed solely to an interface effect. However, progress in this bilayer material system has been slow, primarily because of Bi$_2$Se$_3$ quality problems such as interstitials and vacancies, which lift the Fermi level to the bulk conduction band, resulting in n-type bulk conductivity~\cite{TI_ARPES1, zhangli2013, Zhanybek3, Fisher2010}, thereby complicate the interpretation of experimental results.

A variety of other 3D-TI materials have been studied in search for an optimal TI platform. In particular, like Bi$_2$Se$_3$, both Bi$_2$Te$_3$ and Sb$_2$Te$_3$ share the same quintuple-layer (QL) crystalline structures with similar lattice constants~\cite{SbStructure, BiStructure}. However, unlike Bi$_2$Se$_3$, the Dirac point of either compound is not well exposed in the bulk band gap~\cite{TI_electronic_structure_zhang}. This was resolved by using the alloy (Bi$_x$Sb$_{1-x}$)$_2$Te$_3$ (BST), introducing a further advantage that electric conduction can be tuned between n-type and p-type by changing the Bi to Sb ratios~\cite{ZhangJS2011}. The realization of QAHE by Cr-doping of BST~\cite{Chang2013, Kou2014}, exhibiting high sample quality and robust magnetism at low temperatures, which persists even when the film thickness is beyond the 2D hybridization limit, suggests that it should also be tried with a bilayer configuration.

In this chapter we will discuss further results on magnetic behavior in the BST--EuS bilayer thin film system. In addition to reproducing similar results as in the Bi$_2$Se$_3$--EuS bilayer system in Chap.~\ref{ch:bilayer2014}, namely a positive to negative magnetoresistance crossover at the Curie temperature of EuS $T_C\approx15~\mathrm{K}$, novel magnetic order was observed at the interface between BST and EuS, which persists to $\sim 60~\mathrm{K}$, much higher than the bulk $T_C$ of EuS. Anomalies in the resistivity and AC Susceptibility suggest a two-stage magnetic proximity induced gap-opening mechanism. In the rest of this chapter, the magnetic and transport properties of four representative samples are reported and compared.


Samples of (Bi$_{x}$Sb$_{1-x}$)$_2$Te$_3$--EuS bilayer thin films were fabricated by pulsed laser deposition (PLD). For the ferromagnet layer, 40~nm EuS thin films were grown on intrinsic Si (100) substrates with native oxide. With the previously reported recipe~\cite{EuS_PLD}, high quality EuS thin films were consistently obtained, characterized by single (100) orientations, atomically smooth surfaces, immeasurably high sheet resistance, and magnetic anisotropy exhibiting an out-of plane component of the magnetizations. To optimize the quality of the interfaces, the TI layer was subsequently grown by PLD \textit{in situ}. Based on existing reports on both Sb$_2$Te$_3$ and Bi$_2$Te$_3$~\cite{telluride_PLD1, telluride_PLD2}, we established a procedure to deposit (Bi$_{x}$Sb$_{1-x}$)$_2$Te$_3$ by alternating the targets of the two compounds~\cite{PLD_alt_target2, PLD_alt_target}. Following each EuS deposition carried out in high vacuum of $\sim10^{-7}~\mathrm{torr}$,  the sample was allowed to cool to $\SI{300}{\degreeCelsius}$ before 200~millitorr of argon gas mixed with 2\% hydrogen was introduced to diffuse the plasma plumes and to prevent potential oxidation from residual gases. Sputtering targets (Kurt J. Lesker, 99.999\%) were ablated $\SI{5}{\cm}$ away from the sample with 25~ns 248~nm KrF excimer laser pulses at 0.55 $\mathrm{J\cdot{}cm^{-2}}$ fluence and 5~Hz repetition rate. The thickness of a $\sim\SI{40}{nm}$ film was measured by atomic force profiliometry and the average deposition rate was $\SI{0.22}{\angstrom}$/pulse. To achieve the optimal composition (5$\%$ bismuth doping), where the Fermi level is inside the bulk band gap and closest to the Dirac point~\cite{ZhangJS2011}, the Bi$_2$Te$_3$ target was ablated by one pulse once per 19 pulses on Sb$_2$Te$_3$. 


Here we focus on two thin and optimally doped samples (S1 \& S2), where the surface state should dominate the electric conduction~\cite{ZhangJS2011}; and, to contrast, two thicker and undoped samples (S3 \& S4), where the Fermi levels intersect the bulk valance band, hence a large contribution of p-type bulk conduction is expected (table.~\ref{tab:bl2018_samples}). 
\begin{table}[ht]
    \centering
    \begin{tabularx}{0.7\columnwidth}[t]{l|l|l|X}
		\hline\hline
        Samples & Ferromagnet & TI Compositions & TI Thicknesses\\
        \hline%
        S1 & EuS (100) & (Bi$_{0.05}$Sb$_{0.95}$)$_2$Te$_3$ & 4~nm\\
        S2 & EuS (100) & (Bi$_{0.05}$Sb$_{0.95}$)$_2$Te$_3$ & 6.5~nm\\
        S3 & EuS (100) & Sb$_2$Te$_3$ & 13~nm\\
        S4 & EuS (100) & Sb$_2$Te$_3$ & 65~nm\\
		\hline\hline
    \end{tabularx}
    \caption[Summary of (Bi$_x$Sb$_{1-x}$)$_2$Te$_3$ / EuS bilayer samples presented in Chap.~\ref{ch:bilayer2018}]{\label{tab:bl2018_samples}Summary of samples. S1 \& S2 are thin and optimally doped and therefore should have dominant surface conduction; whereas S3 \& S4 are undoped and thicker therefore should have large contribution from the bulk. Composition and thickness are calculated from numbers of laser pulses.}
\end{table}
The X-ray diffraction (XRD) spectra of these samples (fig.~\ref{fig:bilayer2018_xrd}) indicate clear (001) orientations of the (Bi$_{x}$Sb$_{1-x}$)$_2$Te$_3$ layers. %
%
%
\begin{figure}[ht]%
    \centering%
    \includegraphics[width=0.7\columnwidth]{figs_bilayer2018/material.eps}%
    \subfloat{\label{fig:bilayer2018_xrd}}%
    \subfloat{\label{fig:bilayer2018_mvt}}%
    \subfloat{\label{fig:bilayer2018_mvh}}%
    \caption{\label{fig:bilayer2018_material}(Color online)~(a)~Semi-log X-ray diffraction spectra of the four (Bi$_{x}$Sb$_{1-x}$)$_2$Te$_3$--EuS bilayer samples, with thickness of the TI layer increasing from the top to the bottom, compared to a EuS-only thin film of similar thickness. K-$\beta$ spectral contamination exists in the spectrum of sample S4 due to unavailable monochromator. Dashed lines mark the expected positions of the (Bi,Sb)$_2$Te$_3$ [001] peaks.\protect\cite{SbStructure} Magnetization of sample S2 as functions of (b)~temperatures and (c)~perpendicular magnetic fields (arrows indication field sweep directions). A fitting to the Curie-Weiss law is shown as the black solid curve.}%
\end{figure}%
%
%
%
Their bulk magnetic properties were studied with a superconducting quantum interference device (SQUID) magnetometer. Examples are presented in figs.~\ref{fig:bilayer2018_mvt}~\&~\ref{fig:bilayer2018_mvh} for sample S2. The sample was cooled in zero magnetic field to $T=2~\mathrm{K}$, at which the centering procedure of SQUID was carried out. Subsequently, the field-dependence of the magnetization was measured in an external magnetic field perpendicular to the thin film sweeping between $\mu_0H = +2~\mathrm{T}$ and $\mu_0H = -2~\mathrm{T}$ (fig.~\ref{fig:bilayer2018_mvh}). Finally the external field was reduced to zero from $\mu_0H = +2~\mathrm{T}$, and the temperature dependence of the magnetization was measured during warming up in zero field (fig.~\ref{fig:bilayer2018_mvt}). The component of magnetization perpendicular to the films behaves similarly to EuS thin films without the TI layer~\cite{EuS_PLD}. By fitting to the Curie-Weiss law in the paramagnetic r\'egime, the Curie temperature was determined to be $T_C = 14.5\pm0.3\mathrm{K}$.