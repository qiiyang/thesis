\footnote[2]{A part of this chapter is adapted from ~\cite{bilayer2018}~\fullcite{bilayer2018}, with permission of the publisher. Copyright (2018) by the American Physical Society.}%
%
The first generation of TI--ferromagnet bilayers used bismuth selenide (Bi$_2$Se$_3$) as the TI platform, and EuS~\cite{bilayer2014,Moodera2013} or GdN~\cite{Samarth2013} for the insulating ferromagnet. Relevant to the present study, we previously reported magneto-transport measurements on bilayer samples with europium sulfide (EuS) as the insulating ferromagnet, where a crossover between positive and negative magnetoresistance suggested a proximity effect occurring at the Curie temperature ($T_C$) of EuS~\cite{bilayer2014}. Investigating a similar material system, Wei {\it et al.} further detected a low temperature weak hysteresis as a signature for a developing ferromagnetic phase~\cite{Moodera2013}. Further investigations by this group, using spin-polarized neutron reflectivity experiments, revealed interfacial magnetism that extended $\sim$2 nm into a $\sim$20 nm Bi$_2$Se$_3$ system, which persisted to temperatures much higher than the $T_C$ of EuS itself~\cite{Moodera2016}. While a small increase in $T_C$ of EuS has been reported before, and was attributed to the presence of free bulk carriers~\cite{EuS_ntype, EuS_thin_film_Keller}, the much larger increase in $T_C$ was attributed solely to an interface effect. However, progress in this bilayer material system has been slow, primarily because of Bi$_2$Se$_3$ quality problems such as interstitials and vacancies, which lift the Fermi level to the bulk conduction band, resulting in n-type bulk conductivity~\cite{TI_ARPES1, zhangli2013, Zhanybek3, Fisher2010}, thereby complicate the interpretation of experimental results. 