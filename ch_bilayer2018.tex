\footnote[2]{A part of this chapter is adapted from ~\cite{bilayer2018}~\fullcite{bilayer2018}, with permission of the publisher. Copyright (2018) by the American Physical Society.}%
%
The first generation of TI--ferromagnet bilayers used bismuth selenide (Bi$_2$Se$_3$) as the TI platform, and EuS~\cite{bilayer2014,Moodera2013} or GdN~\cite{Samarth2013} for the insulating ferromagnet. Relevant to the present study, we previously reported magneto-transport measurements on bilayer samples with europium sulfide (EuS) as the insulating ferromagnet, where a crossover between positive and negative magnetoresistance suggested a proximity effect occurring at the Curie temperature ($T_C$) of EuS~\cite{bilayer2014}. Investigating a similar material system, Wei {\it et al.} further detected a low temperature weak hysteresis as a signature for a developing ferromagnetic phase~\cite{Moodera2013}. Further investigations by this group, using spin-polarized neutron reflectivity experiments, revealed interfacial magnetism that extended $\sim$2 nm into a $\sim$20 nm Bi$_2$Se$_3$ system, which persisted to temperatures much higher than the $T_C$ of EuS itself~\cite{Moodera2016}. While a small increase in $T_C$ of EuS has been reported before, and was attributed to the presence of free bulk carriers~\cite{EuS_ntype, EuS_thin_film_Keller}, the much larger increase in $T_C$ was attributed solely to an interface effect. However, progress in this bilayer material system has been slow, primarily because of Bi$_2$Se$_3$ quality problems such as interstitials and vacancies, which lift the Fermi level to the bulk conduction band, resulting in n-type bulk conductivity~\cite{TI_ARPES1, zhangli2013, Zhanybek3, Fisher2010}, thereby complicate the interpretation of experimental results.

A variety of other 3D-TI materials have been studied in search for an optimal TI platform. In particular, like Bi$_2$Se$_3$, both Bi$_2$Te$_3$ and Sb$_2$Te$_3$ share the same quintuple-layer (QL) crystalline structures with similar lattice constants~\cite{SbStructure, BiStructure}. However, unlike Bi$_2$Se$_3$, the Dirac point of either compound is not well exposed in the bulk band gap~\cite{TI_electronic_structure_zhang}. This was resolved by using the alloy (Bi$_x$Sb$_{1-x}$)$_2$Te$_3$ (BST), introducing a further advantage that electric conduction can be tuned between n-type and p-type by changing the Bi to Sb ratios~\cite{ZhangJS2011}. The realization of QAHE by Cr-doping of BST~\cite{Chang2013, Kou2014}, exhibiting high sample quality and robust magnetism at low temperatures, which persists even when the film thickness is beyond the 2D hybridization limit, suggests that it should also be tried with a bilayer configuration.

In this chapter we will discuss further results on magnetic behavior in the BST--EuS bilayer thin film system. In addition to reproducing similar results as in the Bi$_2$Se$_3$--EuS bilayer system in Ch.~\ref{ch:bilayer2014}, namely a positive to negative magnetoresistance crossover at the Curie temperature of EuS $T_C\approx15~\mathrm{K}$, novel magnetic order was observed at the interface between BST and EuS, which persists to $\sim 60~\mathrm{K}$, much higher than the bulk $T_C$ of EuS. Anomalies in the resistivity and AC Susceptibility suggest a two-stage magnetic proximity induced gap-opening mechanism. In the rest of this chapter, the magnetic and transport properties of four representative samples are reported and compared.

